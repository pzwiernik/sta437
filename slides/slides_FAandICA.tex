\documentclass[11pt,handout,aspectratio=169]{beamer}


%%%%%%%%% GENERAL PACKAGES
%\usepackage{xcolor}
%\usepackage{pdfpages}
%\usetheme[progressbar=frametitle]{metropolis}
%\setbeamercolor{background canvas}{bg=white}
%\usepackage{appendixnumberbeamer}
%\usepackage{booktabs}
%\usepackage[scale=2]{ccicons}
%\usepackage{pgfplots}
%\usepgfplotslibrary{dateplot}
%\usepackage{xspace}
%\newcommand{\themename}{\textbf{\textsc{metropolis}}\xspace}
%\usepackage[absolute,overlay]{textpos}

%%%%%%%%% COLOR THEME

% Define some colors:
\definecolor{DarkFern}{HTML}{407428}
\definecolor{DarkCharcoal}{HTML}{4D4944}
\definecolor{AlertColor}{RGB}{89,124,158}
\definecolor{HighLight}{RGB}{96,95,134}
\definecolor{Important}{RGB}{234,122,133}
\definecolor{Yellow}{HTML}{00539C}
\colorlet{Fern}{DarkFern!85!white}
\colorlet{Charcoal}{DarkCharcoal!85!white}
\colorlet{LightCharcoal}{Charcoal!50!white}
\colorlet{HighLight2}{AlertColor}
\colorlet{DarkRed}{red!70!black}
\colorlet{DarkBlue}{blue!70!black}
\colorlet{DarkGreen}{green!70!black}
\definecolor{RoyalBlue}{HTML}{00539C}
\definecolor{Peach}{HTML}{EEA47F}
\definecolor{ForestGreen}{HTML}{2C5F2D}
\definecolor{MossGreen}{HTML}{E8FCC9}
% Use the colors:
\setbeamercolor{title}{fg=Fern}
\setbeamercolor{frametitle}{fg=MossGreen,bg=ForestGreen}
\setbeamercolor{normal text}{fg=Charcoal!70!black}
\setbeamercolor{block title}{fg=black,bg=Fern!25!white}
\setbeamercolor{block body}{fg=black,bg=Fern!10!white}
\setbeamercolor{block title alerted}{fg=black,bg=DarkRed!25!white}
\setbeamercolor{block body alerted}{fg=black,bg=DarkRed!10!white}
\setbeamercolor{alerted text}{fg=DarkRed}
\setbeamercolor{itemize item}{fg=Charcoal}



%%%%%%%%% OTHER COMMANDS
\newcommand{\indep}{\perp\!\!\! \perp}
\newcommand{\comment}[1]{}
\newcommand{\bs}{\boldsymbol}
\newcommand{\tr}{\text{trace}}
\newcommand{\sgn}{{\rm sgn}}
\def\T{\top}
%\newcommand{\det}{\text{det}}
\newcommand{\var}{\mathrm{var}}
\newcommand{\cC}{{\cal C}}
\newcommand{\cG}{{\cal G}}
\newcommand{\cV}{{\cal V}}
\newcommand{\cE}{{\cal E}}
\newcommand{\cM}{{\cal M}}
\newcommand{\cP}{{\cal P}}
\newcommand{\cX}{{\cal X}}
\newcommand{\cY}{{\cal Y}}
\newcommand{\X}{\mathbf{X}}
\newcommand{\Y}{\mathbf{Y}}
\newcommand{\x}{\mathbf{x}}
\newcommand{\y}{\mathbf{y}}
\newcommand{\z}{\mathbf{z}}

\newcommand{\argmin}{\operatornamewithlimits{argmin}}
\newcommand{\eps}{\varepsilon}
\newcommand{\<}{\langle}
\renewcommand{\>}{\rangle}


\setbeamertemplate{itemize subitem}{\tiny\raise1.5pt\hbox{\donotcoloroutermaths$\blacktriangleright$}}
\setbeamertemplate{itemize subsubitem}{\tiny\raise1.5pt\hbox{\donotcoloroutermaths$\blacktriangleright$}}
\setbeamertemplate{enumerate item}{\insertenumlabel.}
\setbeamertemplate{enumerate subitem}{\insertenumlabel.\insertsubenumlabel}
\setbeamertemplate{enumerate subsubitem}{\insertenumlabel.\insertsubenumlabel.\insertsubsubenumlabel}
\setbeamertemplate{enumerate mini template}{\insertenumlabel}

\newcommand{\TODO}[1]{{\color{red}{[TODO: #1]}}}


\newcommand{\R}{\mathbb R}
\newcommand{\E}{\mathbb E}
\renewcommand{\P}{\mathbb P}


\DeclareMathOperator*{\cov}{cov}


\newsavebox{\zerobox}
\newenvironment{nospace}
{\par\edef\theprevdepth{\the\prevdepth}\nointerlineskip
  \setbox\zerobox=\vtop to 0pt\bgroup
  \hrule height0pt\kern\dimexpr\baselineskip-\topskip\relax
}
{\par\vss\egroup\ht\zerobox=0pt \wd\zerobox=0pt \dp\zerobox=0pt
  \box\zerobox}

\usepackage{soul}
\makeatletter
\let\HL\hl
\renewcommand\hl{%
  \let\set@color\beamerorig@set@color
  \let\reset@color\beamerorig@reset@color
  \HL}
  \makeatother


\title[STA437-Week1]{STA 437/2005: \\ Methods for Multivariate Data}
\subtitle[]{Weeks 10: Factor Analysis and Independent Component Analysis}
\author[Piotr Zwiernik]{Piotr Zwiernik}
\institute[UofT]{University of Toronto}
\date{}


%\usepackage{Sweave}

\begin{document}

\maketitle
\section{Introduction}
\begin{frame}{Low-dimensional structures in multivariate statistics}
	In probabilistic PCA we had ...\\[5mm]
	
	Our goal in this lecture is to discuss other related techniques widely used in practice.\\[5mm]
	In all this cases the covariance matrix $\Sigma$ or higher order moments of $X$ have a special strutcture,  \\[5mm]
	Our goal is to explain why these models are important and how they can be learned.
\end{frame}

\section{Factor Analysis }

\begin{frame}{}
	\begin{center}
		{\Huge \alert{Factor Analysis}}
	\end{center}
\end{frame}

\begin{frame}{Factor Analysis: Motivating Examples}
\begin{block}{Example: Capital Asset Pricing Model (CAPM)}
	Models stock returns based on a common factor; the \alert{market return}. For each stock:
	\[ X_i \;=\; \mu_i + w_i Z + \eps_i \]
This is one of the most basic models in finance.
\end{block}
\begin{alertblock}{Example: Human Intelligence}
	Cognitive abilities modeled by latent \alert{intelligence factor}.\\[3mm]
	This could be further generalized to account for multiple types of intelligence.
\end{alertblock}
\end{frame}

\section{Factor Analysis Model}

\begin{frame}{Factor Analysis Model}
The model assumes the following stochastic representation of $X=(X_1,\ldots,X_m)$:
\[ X = \mu + WZ + \eps, \]
where
        \begin{itemize}
            \item $Z \sim N_r(0, I_r)$,
            \item $\eps \sim N_m(0, \Psi)$ (diagonal covariance matrix),
            \item $Z\indep \eps$.
        \end{itemize}
        \bigskip 
        \begin{block}{The latent factors $Z$}
        	As the two examples suggest, often in this context $Z$ has a specific interpretation.
        \end{block}
\end{frame}

\begin{frame}{Parametrization and identifiability}
	$X=\mu+WZ+\eps$ is Gaussian with the induced covariance structure: \[ \Sigma \;=\; \textcolor{blue}{WW^T} + \alert{\Psi}. \]
	Note that this model equals to Probabilistic PCA if $\Psi=\sigma^2I_m$.\\[4mm]
	The general model does not give an analytic formula for the MLE. \\[4mm]
\begin{alertblock}{Lack of identifiability}
	As for PPCA, $W$ is not uniquely identified.
	\begin{itemize}
		\item Replacing $W$ with $WU$ for $U\in O(m)$ does not change the distribution.
		\item This has important consequences for model intepretability.
	\end{itemize}
\end{alertblock}
\end{frame}

\begin{frame}{Dealing with non-uniqueness of $W$}
\textbf{Approach 1: } Constraint $W$ so that $W^T\Psi^{-1}W$ diagonal.
    \begin{itemize}
        \item Multiply $X=\mu+WZ+\eps$ by $\Psi^{-1/2}$ to get $\tilde X=\tilde \mu+\tilde WZ+\tilde \eps$ with $\tilde\eps\sim N(0,I_m)$.
        \item We have $W^T\Psi^{-1}W=\tilde W^\top \tilde W$ so this corresponds to orthogonality of the columns of $\tilde W^\top$.
    \end{itemize}
    \bigskip
    
    \textbf{Approach 2: } Apply varimax rotation for interpretability.
    \begin{itemize}
    	\item Consider any $W\in \R^{m\times r}$. We find $U$ such that $WU$ more interpretable. 
    	\item Define $M\in \R^{m\times r}$ by $M_{ij}=\frac{(WU)_{ij}^2}{\sum_{k=1}^r (WU)_{ij}^2}$ then find the appropriate $U$ maximizing: 
    	$$\|M-\tfrac1m \bs 1_m\bs 1_m^\top M\|^2_F.$$
    	\item This results with solutions such that each column of $M$ has a bunch of big entries and the remaining ones are negligible.
    \end{itemize}
\end{frame}

\begin{frame}{Number of factors}
	ss
\end{frame}

\begin{frame}{Application}
	sss
\end{frame}

\section{Independent Component Analysis (ICA)}

\begin{frame}{ICA: Motivating Examples}
    \textbf{Example: Cocktail Party Problem}
    \begin{itemize}
        \item Separate mixed audio signals into independent sources.
        \item Modeled as: \[ X = WZ \]
    \end{itemize}
    \textbf{Example: EEG/MEG Analysis}
    \begin{itemize}
        \item Extract independent brain signals from mixed recordings.
    \end{itemize}
\end{frame}



\begin{frame}{ICA Model}
    \begin{itemize}
        \item Observed signal: \[ X = WZ \]
        \item Assumes $Z$ has independent non-Gaussian components.
        \item Identifiability result: Unique up to permutation and scaling.
    \end{itemize}
\end{frame}

\begin{frame}{ICA: Non-Gaussianity Principle}
    \begin{itemize}
        \item Gaussian components remain Gaussian under rotation.
        \item Non-Gaussian components allow identification.
        \item Theorem (Comon): If at most one component is Gaussian, $W$ is identifiable up to permutation and scaling.
    \end{itemize}
\end{frame}

\section{Estimation Methods}

\begin{frame}{Factor Analysis Estimation: MLE}
    \begin{itemize}
        \item Log-likelihood maximization.
        \item EM algorithm: iteratively estimates $W$ and $\Psi$.
        \item J\"oreskog's method: Spectral decomposition approach.
    \end{itemize}
\end{frame}

\begin{frame}{ICA Estimation: Method of Moments}
    \begin{itemize}
        \item Uses higher-order statistics (kurtosis, skewness).
        \item JADE algorithm: Estimates mixing matrix via cumulants.
        \item Contrast functions: Measures independence.
    \end{itemize}
\end{frame}

\section{Choosing Number of Components}

\begin{frame}{Choosing the Number of Factors}
    \begin{itemize}
        \item Eigenvalue-based criteria (e.g., Kaiser's rule, Scree plot).
        \item Information-theoretic approaches (AIC, BIC).
        \item Parallel analysis: Compare to random eigenvalues.
        \item Likelihood ratio tests for nested models.
    \end{itemize}
\end{frame}

\begin{frame}{Conclusion}
    \begin{itemize}
        \item Factor Analysis captures shared variance via latent factors.
        \item ICA decomposes signals into independent components.
        \item Estimation methods include MLE for FA and cumulant-based techniques for ICA.
    \end{itemize}
    \centering{\textbf{Thank you!}}
\end{frame}

\end{document}

