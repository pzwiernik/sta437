\documentclass[11pt,handout,aspectratio=169]{beamer}


%%%%%%%%% GENERAL PACKAGES
%\usepackage{xcolor}
%\usepackage{pdfpages}
%\usetheme[progressbar=frametitle]{metropolis}
%\setbeamercolor{background canvas}{bg=white}
%\usepackage{appendixnumberbeamer}
%\usepackage{booktabs}
%\usepackage[scale=2]{ccicons}
%\usepackage{pgfplots}
%\usepgfplotslibrary{dateplot}
%\usepackage{xspace}
%\newcommand{\themename}{\textbf{\textsc{metropolis}}\xspace}
%\usepackage[absolute,overlay]{textpos}

%%%%%%%%% COLOR THEME

% Define some colors:
\definecolor{DarkFern}{HTML}{407428}
\definecolor{DarkCharcoal}{HTML}{4D4944}
\definecolor{AlertColor}{RGB}{89,124,158}
\definecolor{HighLight}{RGB}{96,95,134}
\definecolor{Important}{RGB}{234,122,133}
\definecolor{Yellow}{HTML}{00539C}
\colorlet{Fern}{DarkFern!85!white}
\colorlet{Charcoal}{DarkCharcoal!85!white}
\colorlet{LightCharcoal}{Charcoal!50!white}
\colorlet{HighLight2}{AlertColor}
\colorlet{DarkRed}{red!70!black}
\colorlet{DarkBlue}{blue!70!black}
\colorlet{DarkGreen}{green!70!black}
\definecolor{RoyalBlue}{HTML}{00539C}
\definecolor{Peach}{HTML}{EEA47F}
\definecolor{ForestGreen}{HTML}{2C5F2D}
\definecolor{MossGreen}{HTML}{E8FCC9}
% Use the colors:
\setbeamercolor{title}{fg=Fern}
\setbeamercolor{frametitle}{fg=MossGreen,bg=ForestGreen}
\setbeamercolor{normal text}{fg=Charcoal!70!black}
\setbeamercolor{block title}{fg=black,bg=Fern!25!white}
\setbeamercolor{block body}{fg=black,bg=Fern!10!white}
\setbeamercolor{block title alerted}{fg=black,bg=DarkRed!25!white}
\setbeamercolor{block body alerted}{fg=black,bg=DarkRed!10!white}
\setbeamercolor{alerted text}{fg=DarkRed}
\setbeamercolor{itemize item}{fg=Charcoal}



%%%%%%%%% OTHER COMMANDS
\newcommand{\indep}{\perp\!\!\! \perp}
\newcommand{\comment}[1]{}
\newcommand{\bs}{\boldsymbol}
\newcommand{\tr}{\text{trace}}
\newcommand{\sgn}{{\rm sgn}}
\def\T{\top}
%\newcommand{\det}{\text{det}}
\newcommand{\var}{\mathrm{var}}
\newcommand{\cC}{{\cal C}}
\newcommand{\cG}{{\cal G}}
\newcommand{\cV}{{\cal V}}
\newcommand{\cE}{{\cal E}}
\newcommand{\cM}{{\cal M}}
\newcommand{\cP}{{\cal P}}
\newcommand{\cX}{{\cal X}}
\newcommand{\cY}{{\cal Y}}
\newcommand{\X}{\mathbf{X}}
\newcommand{\Y}{\mathbf{Y}}
\newcommand{\x}{\mathbf{x}}
\newcommand{\y}{\mathbf{y}}
\newcommand{\z}{\mathbf{z}}

\newcommand{\argmin}{\operatornamewithlimits{argmin}}
\newcommand{\eps}{\varepsilon}
\newcommand{\<}{\langle}
\renewcommand{\>}{\rangle}


\setbeamertemplate{itemize subitem}{\tiny\raise1.5pt\hbox{\donotcoloroutermaths$\blacktriangleright$}}
\setbeamertemplate{itemize subsubitem}{\tiny\raise1.5pt\hbox{\donotcoloroutermaths$\blacktriangleright$}}
\setbeamertemplate{enumerate item}{\insertenumlabel.}
\setbeamertemplate{enumerate subitem}{\insertenumlabel.\insertsubenumlabel}
\setbeamertemplate{enumerate subsubitem}{\insertenumlabel.\insertsubenumlabel.\insertsubsubenumlabel}
\setbeamertemplate{enumerate mini template}{\insertenumlabel}

\newcommand{\TODO}[1]{{\color{red}{[TODO: #1]}}}


\newcommand{\R}{\mathbb R}
\newcommand{\E}{\mathbb E}
\renewcommand{\P}{\mathbb P}


\DeclareMathOperator*{\cov}{cov}


\newsavebox{\zerobox}
\newenvironment{nospace}
{\par\edef\theprevdepth{\the\prevdepth}\nointerlineskip
  \setbox\zerobox=\vtop to 0pt\bgroup
  \hrule height0pt\kern\dimexpr\baselineskip-\topskip\relax
}
{\par\vss\egroup\ht\zerobox=0pt \wd\zerobox=0pt \dp\zerobox=0pt
  \box\zerobox}

\usepackage{soul}
\makeatletter
\let\HL\hl
\renewcommand\hl{%
  \let\set@color\beamerorig@set@color
  \let\reset@color\beamerorig@reset@color
  \HL}
  \makeatother


\title[STA437-Week1]{STA 437/2005: \\ Methods for Multivariate Data}
\subtitle[]{Week 1: Introduction and Preliminaries}
\author[Prob Learning]{Piotr Zwiernik}
\institute[UofT]{University of Toronto}
\date{}


\usepackage{Sweave}


\begin{document}

\maketitle

\begin{frame}{Table of contents}
  \setbeamertemplate{section in toc}[sections numbered]
  \tableofcontents%[hideallsubsections]
\end{frame}



\section{Introduction to Gaussian Processes (GPs)}

\begin{frame}{What are Gaussian Processes?}
\begin{itemize}
    \item A \textbf{Gaussian Process (GP)} is a generalization of the multivariate normal distribution to a collection of random variables indexed by a set \( T \).
    \item A GP defines a distribution over functions, characterized by:
    \begin{itemize}
        \item A \textbf{mean function}: \( m(t) = \mathbb{E}[X_t] \)
        \item A \textbf{kernel function}: \( k(t, t') = \text{Cov}(X_t, X_{t'}) \)
    \end{itemize}
    \item For any finite set of points \( \{t_1, \dots, t_n\} \subset T \), the corresponding vector \( (X_{t_1}, \dots, X_{t_n}) \) follows a multivariate normal distribution.
\end{itemize}
\end{frame}


\begin{frame}{Key Properties of GPs}
\begin{itemize}
    \item \textbf{Mean Function:} Specifies the average behavior of the process:
    \[
    m(t) = \mathbb{E}[X_t].
    \]
    \item \textbf{Kernel Function:} Defines the covariance structure:
    \[
    k(t, t') = \text{Cov}(X_t, X_{t'}).
    \]
    \item \textbf{Positive Semi-Definiteness:} For any finite set \( \{t_1, \dots, t_n\} \), the covariance matrix \( \Sigma \) with entries \( \Sigma_{ij} = k(t_i, t_j) \) is positive semi-definite.
\end{itemize}
\end{frame}

\begin{frame}{Common Kernels in GPs}
\begin{itemize}
    \item \textbf{Squared Exponential (RBF) Kernel:}
    \[
    k_{\textsc{e}}(t, t') = \sigma^2 \exp\left(-\frac{\|t - t'\|^2}{2\ell^2}\right).
    \]
    \begin{itemize}
        \item Controls smoothness of the functions sampled from the GP.
        \item Length scale \( \ell \): Correlation distance.
        \item Signal variance \( \sigma^2 \): Scale of the output.
    \end{itemize}
    \item \textbf{Mat\'{e}rn Kernel:}
    \[
    k_{\textsc{m}}(t, t') = \sigma^2 \frac{2^{1-\nu}}{\Gamma(\nu)} \left(\sqrt{2\nu} \frac{\|t - t'\|}{\ell}\right)^\nu K_\nu\left(\sqrt{2\nu} \frac{\|t - t'\|}{\ell}\right).
    \]
    \begin{itemize}
        \item \( \nu \): Smoothness parameter.
        \item More flexible than the RBF kernel for modeling rough functions.
    \end{itemize}
\end{itemize}
\end{frame}

\section{Gaussian Processes for Spatial Data}

\begin{frame}{Example: Modeling Spatial Data with GPs}
\begin{itemize}
    \item Gaussian Processes are widely used in spatial statistics to model correlated observations over geographic regions.
    \item Example: Modeling temperature across a grid of locations.
\end{itemize}

%\begin{minted}[frame=single,fontsize=\small]{r}
%# Generate spatially correlated data
%set.seed(42)
%n <- 100
%locs <- expand.grid(x = seq(0, 1, length.out = n), y = seq(0, 1, length.out = n))
%true_cov <- exp(-rdist(locs) / 2)  # Exponential kernel
%temp <- t(chol(true_cov)) %*% rnorm(n * n)
%image(matrix(temp, n, n), main = "Simulated Temperature Data")
%\end{minted}
\end{frame}

\begin{frame}{Spatial GP: Prediction}
\begin{enumerate}
    \item Combine training and test locations.
    \item Compute the covariance matrix using the kernel function.
    \item Use Gaussian conditioning formulas:
    \begin{align*}
        \mathbb{E}[\mathbf{y}_\text{test} | \mathbf{y}_\text{train}] &= \mathbf{k}_\text{test,train}^\top \mathbf{K}_\text{train,train}^{-1} \mathbf{y}_\text{train}, \\
        \text{Cov}(\mathbf{y}_\text{test} | \mathbf{y}_\text{train}) &= \mathbf{K}_\text{test,test} - \mathbf{k}_\text{test,train}^\top \mathbf{K}_\text{train,train}^{-1} \mathbf{k}_\text{test,train}.
    \end{align*}
\end{enumerate}
\end{frame}

\section{Nonparametric Regression with GPs}

\begin{frame}{Nonparametric Regression}
\begin{itemize}
    \item GPs can be used for nonparametric regression:
    \[
    y_i = f(x_i) + \epsilon, \quad \epsilon \sim \mathcal{N}(0, \sigma^2).
    \]
    \item Prior over \( f(x) \): GP defined by \( m(x) \) and \( k(x, x') \).
    \item Prediction involves computing the posterior GP.
\end{itemize}
\end{frame}

%\begin{frame}{Code Example: Nonparametric Regression}
%\begin{minted}[frame=single,fontsize=\small]{r}
%# Generate training data
%x_train <- seq(0, 10, length.out = 10)
%y_train <- sin(x_train) + rnorm(length(x_train), sd = 0.1)
%
%# Test points
%x_test <- seq(0, 10, length.out = 100)
%
%# RBF kernel
%k <- function(x, x_prime, length_scale = 1) {
%    exp(-0.5 * (outer(x, x_prime, "-")^2) / length_scale^2)
%}
%
%# Covariance matrices
%K <- k(x_train, x_train)
%K_s <- k(x_train, x_test)
%K_ss <- k(x_test, x_test)
%
%# Posterior mean and covariance
%mu_s <- t(K_s) %*% solve(K) %*% y_train
%\end{minted}
%\end{frame}

\begin{frame}{Summary}
\begin{itemize}
    \item Gaussian Processes are a versatile tool for regression and spatial modeling.
    \item Key components:
    \begin{itemize}
        \item Mean function.
        \item Kernel function.
    \end{itemize}
    \item Next: Applications of GPs in high-dimensional data.
\end{itemize}
\end{frame}

\end{document}
