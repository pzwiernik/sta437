\documentclass[11pt,handout,aspectratio=169]{beamer}


%%%%%%%%% GENERAL PACKAGES

\usepackage{pdfpages}
\usetheme[progressbar=frametitle]{metropolis}
\setbeamercolor{background canvas}{bg=white}
\usepackage{appendixnumberbeamer}
\usepackage{booktabs}
\usepackage[scale=2]{ccicons}
\usepackage{pgfplots}
\usepgfplotslibrary{dateplot}
\usepackage{xspace}
\newcommand{\themename}{\textbf{\textsc{metropolis}}\xspace}
\usepackage[absolute,overlay]{textpos}

%%%%%%%%% COLOR THEME

% Define some colors:
\definecolor{DarkFern}{HTML}{407428}
\definecolor{DarkCharcoal}{HTML}{4D4944}
\definecolor{AlertColor}{RGB}{89,124,158}
\definecolor{HighLight}{RGB}{96,95,134}
\definecolor{Important}{RGB}{234,122,133}
\definecolor{Yellow}{HTML}{00539C}
\colorlet{Fern}{DarkFern!85!white}
\colorlet{Charcoal}{DarkCharcoal!85!white}
\colorlet{LightCharcoal}{Charcoal!50!white}
\colorlet{HighLight2}{AlertColor}
\colorlet{DarkRed}{red!70!black}
\colorlet{DarkBlue}{blue!70!black}
\colorlet{DarkGreen}{green!70!black}
\definecolor{RoyalBlue}{HTML}{00539C}
\definecolor{Peach}{HTML}{EEA47F}
\definecolor{ForestGreen}{HTML}{2C5F2D}
\definecolor{MossGreen}{HTML}{E8FCC9}
% Use the colors:
\setbeamercolor{title}{fg=Fern}
\setbeamercolor{frametitle}{fg=MossGreen,bg=ForestGreen}
\setbeamercolor{normal text}{fg=Charcoal!70!black}
\setbeamercolor{block title}{fg=black,bg=Fern!25!white}
\setbeamercolor{block body}{fg=black,bg=Fern!10!white}
\setbeamercolor{block title alerted}{fg=black,bg=DarkRed!25!white}
\setbeamercolor{block body alerted}{fg=black,bg=DarkRed!10!white}
\setbeamercolor{alerted text}{fg=DarkRed}
\setbeamercolor{itemize item}{fg=Charcoal}



%%%%%%%%% OTHER COMMANDS
\newcommand{\Perp}{\perp\!\!\! \perp}
\newcommand{\Ep}[2]{\ensuremath{E_{#1}\left[{#2}\right]}}
\def\hpY{\mathbf{\bar{\beta}}}
\newcommand{\gaus}[2]{\mathcal{N}\left({#1};\,{#2}\right)}
\newcommand{\comment}[1]{}

\newcommand{\trace}{\text{trace}}
\def\T{\top}
%\newcommand{\det}{\text{det}}
\newcommand{\variance}{\mathrm{Var}}
\def\reals{{\mathbb R}}
\newcommand{\prob}{\mathrm{Pr}}
\newcommand{\zeroVec}{\mathbf{0}}
\newcommand{\zeroMat}{\mathbf{0}}
\newcommand{\onesVec}{\mathbf{1}}
\newcommand{\ident}{\mathbf{I}}
\newcommand{\deriv}{\mathrm{d}}
\newcommand{\transpose}{\top}
\newcommand{\costDeriv}[1]{\overline{#1}}
\newcommand{\lossDeriv}{\costDeriv}
\newcommand{\normal}{\mathcal{N}}
\newcommand{\data}{\mathcal{D}}
%\newcommand{\bm}{{\mathbf{m}}}
\newcommand{\loss}{{\cal L}}
\newcommand{\cG}{{\cal G}}
\newcommand{\cV}{{\cal V}}
\newcommand{\cE}{{\cal E}}
\newcommand{\cP}{{\cal P}}
\newcommand{\X}{{\cal X}}
\newcommand{\Y}{{\cal Y}}
\newcommand{\bK}{\mathbf{K}}
\newcommand{\bX}{\mathbf{X}}
\newcommand{\bY}{\mathbf{Y}}
\newcommand{\bk}{\mathbf{k}}
\newcommand{\bx}{\mathbf{x}}
\newcommand{\by}{\mathbf{y}}
\newcommand{\bhy}{\hat{\mathbf{y}}}
\newcommand{\bty}{\tilde{\mathbf{y}}}
\newcommand{\bG}{\mathbf{G}}
\newcommand{\bI}{\mathbf{I}}
\newcommand{\bg}{\mathbf{g}}
\newcommand{\bS}{\mathbf{S}}
\newcommand{\bs}{\mathbf{s}}
\newcommand{\bM}{\mathbf{M}}
\newcommand{\bw}{\mathbf{w}}
\newcommand{\eye}{\mathbf{I}}
\newcommand{\bU}{\mathbf{U}}
\newcommand{\bV}{\mathbf{V}}
\newcommand{\bW}{\mathbf{W}}
\newcommand{\bn}{\mathbf{n}}
\newcommand{\bv}{\mathbf{v}}
\newcommand{\bq}{\mathbf{q}}
\newcommand{\bR}{\mathbf{R}}
\newcommand{\bi}{\mathbf{i}}
\newcommand{\bj}{\mathbf{j}}
\newcommand{\bp}{\mathbf{p}}
\newcommand{\bt}{\mathbf{t}}
\newcommand{\bJ}{\mathbf{J}}
\newcommand{\bu}{\mathbf{u}}
\newcommand{\bB}{\mathbf{B}}
\newcommand{\bD}{\mathbf{D}}
\newcommand{\bz}{\mathbf{z}}
\newcommand{\bP}{\mathbf{P}}
\newcommand{\bC}{\mathbf{C}}
\newcommand{\bA}{\mathbf{A}}
\newcommand{\bZ}{\mathbf{Z}}
\newcommand{\bff}{\mathbf{f}}
\newcommand{\bF}{\mathbf{F}}
\newcommand{\bo}{\mathbf{o}}
\newcommand{\bc}{\mathbf{c}}
\newcommand{\bm}{\mathbf{m}}
\newcommand{\bT}{\mathbf{T}}
\newcommand{\bQ}{\mathbf{Q}}
\newcommand{\bL}{\mathbf{L}}
\newcommand{\bl}{\mathbf{l}}
\newcommand{\ba}{\mathbf{a}}
\newcommand{\bE}{\mathbf{E}}
\newcommand{\bH}{\mathbf{H}}
\newcommand{\bN}{\mathbf{N}}
\newcommand{\bd}{\mathbf{d}}
\newcommand{\br}{\mathbf{r}}
\newcommand{\be}{\mathbf{e}}
\newcommand{\bb}{\mathbf{b}}
\newcommand{\bh}{\mathbf{h}}
\newcommand{\bhh}{\hat{\mathbf{h}}}

\newcommand{\graph}{{\cal H}}
\newcommand{\bayes}{{\cal B}}
\newcommand{\cx}{{\cal X}}
\newcommand{\cg}{{\cal G}}
\newcommand{\cm}{{\cal M}}
\newcommand{\ci}{{\cal I}}
\newcommand{\ct}{{\cal T}}
\newcommand{\co}{{\cal O}}
\newcommand{\ck}{{\cal K}}
\newcommand{\cu}{{\cal U}}
\newcommand{\cv}{{\cal V}}
\newcommand{\ce}{{\cal E}}
\newcommand{\cf}{{\cal F}}
\newcommand{\cb}{{\cal B}}
\newcommand{\cq}{{\cal Q}}
\newcommand{\cd}{{\cal D}}

\newcommand{\btheta}{\boldsymbol{\theta}}
\newcommand{\bpi}{\boldsymbol{\pi}}
\newcommand{\bphi}{\boldsymbol{\phi}}
\newcommand{\bPhi}{\boldsymbol{\Phi}}
\newcommand{\bmu}{\boldsymbol{\mu}}
\newcommand{\bSigma}{\boldsymbol{\Sigma}}
\newcommand{\bGamma}{\boldsymbol{\Gamma}}
\newcommand{\bbeta}{\boldsymbol{\beta}}
\newcommand{\bomega}{\boldsymbol{\omega}}
\newcommand{\blambda}{\boldsymbol{\lambda}}
\newcommand{\bkappa}{\boldsymbol{\kappa}}
\newcommand{\btau}{\boldsymbol{\tau}}
\newcommand{\balpha}{\boldsymbol{\alpha}}
\def\bgamma{\boldsymbol\gamma}

\newcommand{\argmin}{\operatornamewithlimits{argmin}}

%\newcommand{\animal}[2]{\item[\bf #1] {\em #2}}
 \newcommand{\ikron}[1] {\bI\otimes #1}
  %\newcommand{\val}{\bar{\bx}}
    \newcommand{\train}[1]{{\phi(\bx_{#1})}}
    \newcommand{\ikronval}[1]{(\ikron{\phi(\val_{#1}))}}
\newcommand{\ikronvalT}[1]{(\ikron{\phi(\val_{#1})^T)}}
\newcommand{\ikrontrainT}{(\ikron{\train{i}^T)}}
\newcommand{\ikrontrain}[1]{(\ikron{\train{#1})}}
\newcommand{\ikrontrainAT}{(\ikron{\phi(\bx)^T)}}
\newcommand{\ikrontrainA}{(\ikron{\phi(\bx))}}
  \newcommand{\half}{\frac{1}{2}}
  \newcommand{\con}{C^{(c)}}
    \newcommand{\ig}{\frac{1}{\gamma}}
      \newcommand{\Bi}{\bB^{-1}}
 \newcommand{\kernel}{\hat{\bK}}    
 \newcommand{\ikrontestT}{(\ikron{\test^T)}}
   \newcommand{\test}{\phi(\bx_*)}

% partial derivatives
 \newcommand{\pardev}[2]{\frac{\partial #1}{\partial #2}}
  \newcommand{\dev}[2]{\frac{d #1}{d #2}}
  \newcommand{\dw}{\delta\bw}
  
    \newcommand{\lab}{\mathcal{L}}
      \newcommand{\unlab}{\mathcal{U}}
      
      
  \newcommand{\ind}{1{\hskip -2.5 pt}\hbox{I}}
  
 \newcommand{\ff}[2]{   \cf_{\prec (#1 \rightarrow #2)}}
 \newcommand{\vv}[2]{   \cv_{\prec (#1 \rightarrow #2)}}
  \newcommand{\dd}[2]{   \delta_{#1 \rightarrow #2}}
    \newcommand{\ld}[2]{   \lambda_{#1 \rightarrow #2}}
    \newcommand{\en}[2]{  \bD(#1|| #2)}
       \newcommand{\ex}[3]{  \bE_{#1 \sim #2}\left[ #3\right]} 
       \newcommand{\exd}[2]{  \bE_{#1 }\left[ #2\right]} 
  
%  \newtheorem{theorem}{Theorem}
%\newtheorem{proposition}{Prop}
%\newtheorem{lemma}{Lemma}
%\newtheorem{lemma-ap}{Lemma}
%\newtheorem{definition}{Definition}
%\newtheorem{corollary}{Corollary}
%\newtheorem{claim}{Claim}
%\newtheorem{claim-ap}{Claim}
%\newcommand{\argmin}[1]{\underset{#1}{\mathrm{argmin}} \:}
\newcommand{\argmax}[1]{\underset{#1}{\mathrm{argmax}} \:}
\DeclareMathOperator*{\Max}{max}
\def\eop {{\noindent\framebox[0.5em]{\rule[0.25ex]{0em}{0.75ex}}}}

\newcommand{\tr}[1]{\ensuremath{\mathrm{tr}\left(#1\right)}}
\def\Xdim{{d}}
\def\Ydim{{D}}
\def\Zdim{{S}}

\setbeamertemplate{itemize subitem}{\tiny\raise1.5pt\hbox{\donotcoloroutermaths$\blacktriangleright$}}
\setbeamertemplate{itemize subsubitem}{\tiny\raise1.5pt\hbox{\donotcoloroutermaths$\blacktriangleright$}}
\setbeamertemplate{enumerate item}{\insertenumlabel.}
\setbeamertemplate{enumerate subitem}{\insertenumlabel.\insertsubenumlabel}
\setbeamertemplate{enumerate subsubitem}{\insertenumlabel.\insertsubenumlabel.\insertsubsubenumlabel}
\setbeamertemplate{enumerate mini template}{\insertenumlabel}

\newcommand{\book}[1]{{\it{#1}}}

\newcommand{\high}[1]{{\color{blue}{#1}}}
\newcommand{\raquel}[1]{{\color{red}{#1}}}

\newcommand{\Real}{\mathbb{R}}
\newcommand{\TODO}[1]{{\color{red}{[TODO: #1]}}}

\newcommand{\dataGenDist}{p_{\rm sample}}
\newcommand{\trainDist}{p_{\rm dataset}}
\DeclareMathOperator*{\Var}{Var}
\newcommand{\given}{\,|\,}
\newcommand{\expect}{\mathbb{E}}

\newcommand{\dataIdx}{i}
\newcommand{\featIdx}{j}
\newcommand{\dimIdx}{\featIdx}
\newcommand{\paramIdx}{\dimIdx}
\newcommand{\hidIdx}{i}
\newcommand{\classIdx}{k}
\newcommand{\outputIdx}{k}
\newcommand{\classIdxTwo}{\ell}
\newcommand{\featIdxTwo}{j^\prime}
\newcommand{\nfeat}{D}
\newcommand{\ndim}{\nfeat}
\newcommand{\ndata}{N}
\newcommand{\numClasses}{K}
\newcommand{\nout}{\numClasses}
\newcommand{\layerIdx}{\ell}
\newcommand{\numLayers}{L}
\newcommand{\nhid}{M}
\newcommand{\timeIdx}{t}
\newcommand{\ntime}{T}
\newcommand{\contextLen}{K}


\newcommand{\inputIJ}[2]{x^{(#1)}_{#2}}
\newcommand{\inputI}[1]{{\bf x}^{(#1)}}
\newcommand{\inputJ}[1]{x_{#1}}
\newcommand{\inputVec}{{\bf x}}
\newcommand{\inputVecT}[1]{\inputVec^{(#1)}}
\newcommand{\inputVecI}[1]{\inputVec^{(#1)}}
\newcommand{\inputUni}{x}
\newcommand{\inputUniI}[1]{x^{(#1)}}
\newcommand{\inputUniT}[1]{x^{(#1)}}
\newcommand{\inputMatrix}{\mathbf{X}}
\newcommand{\inputMatrixT}[1]{\inputMatrix^{(#1)}}
\newcommand{\targetI}[1]{t^{(#1)}}
\newcommand{\target}{t}
\newcommand{\targetK}[1]{\target_{#1}}
\newcommand{\targets}{\mathbf{t}}
\newcommand{\prediction}{y}
\newcommand{\predictionI}[1]{y^{(#1)}}
\newcommand{\predictionK}[1]{y_{#1}}
\newcommand{\predictionT}[1]{y^{(#1)}}
\newcommand{\predictions}{\mathbf{y}}
\newcommand{\predictionMatrix}{\mathbf{Y}}
\newcommand{\predictionMatrixT}[1]{\predictionMatrix^{(#1)}}
\newcommand{\intermediate}{z}
\newcommand{\intermediateI}[1]{\intermediate^{(#1)}}
\newcommand{\intermediateT}[1]{\intermediate^{(#1)}}
\newcommand{\intermediateK}[1]{\intermediate_{#1}}
\newcommand{\intermediates}{\mathbf{z}}
\newcommand{\intermediateMatrix}{\mathbf{Z}}
\newcommand{\intermediateMatrixT}[1]{\intermediateMatrix^{(#1)}}
\newcommand{\outIntermediate}{r}
\newcommand{\outIntermediateT}[1]{r^{(#1)}}
\newcommand{\outIntermediateK}[1]{\outIntermediate_{#1}}
\newcommand{\outIntermediates}{\mathbf{r}}
\newcommand{\outIntermediateMat}{\mathbf{R}}
\newcommand{\outIntermediateMatrix}{\mathbf{R}}
\newcommand{\outIntermediateMatrixT}[1]{\outIntermediateMatrix^{(#1)}}
\newcommand{\hiddenI}[1]{h_{#1}}
\newcommand{\hiddenT}[1]{h^{(#1)}}
\newcommand{\hiddenIT}[2]{h_{#1}^{(#2)}}
\newcommand{\hiddenLI}[2]{h_{#2}^{(#1)}}
\newcommand{\hiddens}{\mathbf{h}}
\newcommand{\hiddensL}[1]{\hiddens^{(#1)}}
\newcommand{\hiddensT}[1]{\hiddens^{(#1)}}
\newcommand{\hiddenMatrix}{\mathbf{H}}
\newcommand{\hiddenMat}{\hiddenMatrix}
\newcommand{\hiddenMatrixT}[1]{\hiddenMatrix^{(#1)}}
\newcommand{\hiddenMatL}[1]{\hiddenMat^{(#1)}}
\newcommand{\weights}{{\bf w}}
\newcommand{\weightsLS}{{\bf w}^{\text{LS}}}
\newcommand{\weightsMLE}{{\bf w}^{\text{MLE}}}
\newcommand{\weightsHat}{\hat{\weights}}
\newcommand{\weightsL}[1]{\weights^{(#1)}}
\newcommand{\weightJ}[1]{w_{#1}}
\newcommand{\weightLIJ}[3]{w^{(#1)}_{#2 #3}}
\newcommand{\weightLKI}[3]{w^{(#1)}_{#2 #3}}
\newcommand{\weightLJ}[2]{w^{(#1)}_{#2}}
\newcommand{\weightKJ}[2]{w_{#1 #2}}
\newcommand{\weightIJ}{\weightKJ}
\newcommand{\weightUni}{w}
\newcommand{\weightMat}{\mathbf{W}}
\newcommand{\weightMatL}[1]{\weightMat^{(#1)}}
\newcommand{\bias}{b}
\newcommand{\biasLI}[2]{\bias^{(#1)}_{#2}}
\newcommand{\biasLK}{\biasLI}
\newcommand{\biasL}[1]{\bias^{(#1)}}
\newcommand{\biasK}[1]{\bias_{#1}}
\newcommand{\biasJ}[1]{\bias_{#1}}
\newcommand{\biases}{\mathbf{b}}
\newcommand{\biasesL}[1]{\biases^{(#1)}}
\newcommand{\threshold}{r}
\newcommand{\featureJ}[1]{\psi_{#1}}
\newcommand{\featureVec}{{\boldsymbol \psi}}
\newcommand{\lossI}[1]{\loss^{(#1)}}
\newcommand{\zeroOneLoss}{\loss_{\rm 0-1}}
\newcommand{\squaredErrorLoss}{\loss_{\rm SE}}
\newcommand{\crossEntropyLoss}{\loss_{\rm CE}}
\newcommand{\logisticCrossEntropyLoss}{\loss_{\rm LCE}}
\newcommand{\softmaxCrossEntropyLoss}{\loss_{\rm SCE}}
\newcommand{\hingeLoss}{\loss_{\rm H}}
\newcommand{\cost}{\mathcal{J}}
\newcommand{\regularizer}{\mathcal{R}}
\newcommand{\lrate}{\alpha}
\newcommand{\learningRate}{\lrate}
\newcommand{\featureMap}{{\boldsymbol \psi}}
\newcommand{\featureMapJ}[1]{\psi_{#1}}
\newcommand{\sigmoid}{\sigma}
\newcommand{\logistic}{\sigmoid}
\newcommand{\activationFunction}{\phi}
\newcommand{\activationFunctionL}[1]{\activationFunction^{(#1)}}
\newcommand{\activationFunctionTwo}{\psi}
\newcommand{\parityFunction}{f_{\rm par}}
\newcommand{\function}{f}
\newcommand{\functionL}[1]{\function^{(#1)}}
\newcommand{\indicatorOf}[1]{\mathbbm{1}_{#1}}
\newcommand{\softmax}{\mathrm{softmax}}
\newcommand{\weightCost}{\lambda}
\newcommand{\genCost}{\mathcal{C}}
\newcommand{\momentumVec}{\mathbf{p}}
\newcommand{\momentumJ}[1]{p_{#1}}
\newcommand{\momentumParam}{\mu}
\newcommand{\genParams}{{\boldsymbol \theta}}
\newcommand{\genParamJ}[1]{\theta_{#1}}
\newcommand{\pData}{p_{\mathcal{D}}}
\newcommand{\bestPrediction}{\prediction_\star}

\newcommand{\obs}{\mathbf{x}}
\newcommand{\obsJ}[1]{x_{#1}}
\newcommand{\obsI}[1]{\obs^{(#1)}}
\newcommand{\pfn}{\mathcal{Z}}
\newcommand{\happiness}{H}
\newcommand{\latents}{\mathbf{z}}

\newcommand{\state}{\mathbf{s}}
\newcommand{\stateT}[1]{\state_{#1}}
\newcommand{\act}{\mathbf{a}}
\newcommand{\actT}[1]{\act_{#1}}
\newcommand{\reward}{r}
\newcommand{\policy}{\pi}
\newcommand{\policyParams}{\boldsymbol{\theta}}
\newcommand{\policyTh}{{\policy_{\policyParams}}}
\newcommand{\MDP}{\mathcal{M}}
\newcommand{\rollout}{\tau}
\newcommand{\expectedReturn}{R}

\newcommand{\discReturn}{G}
\newcommand{\discFactor}{\gamma}
\newcommand{\valueFunc}{V}
\newcommand{\valueFuncPi}{\valueFunc^{\policy}}
\newcommand{\valueFuncPiTh}{\valueFunc^{\policyTh}}
\newcommand{\qFunc}{Q}
\newcommand{\qFuncPi}{\qFunc^{\policy}}
\newcommand{\optPolicy}{\policy^*}
\newcommand{\optQ}{\qFunc^*}

\newcommand{\subspace}{\mathcal{S}}
\newcommand{\projectedInput}{\tilde{\inputVec}}
\newcommand{\projectedInputI}[1]{\projectedInput^{(#1)}}
\newcommand{\codeVec}{\mathbf{z}}
\newcommand{\codeVecI}[1]{\codeVec^{(#1)}}
\newcommand{\dataMean}{\boldsymbol{\mu}}
\newcommand{\dataCov}{\boldsymbol{\Sigma}}
\newcommand{\pcaVec}{\mathbf{u}}

\newcommand{\featureMatrix}{{\boldsymbol \Psi}}
\newcommand{\smootherMatrix}{{\boldsymbol \Omega}}
\newcommand{\smootherMatrixEntry}{\Omega}
\newcommand{\hypothesis}{\mathcal{H}}
\newcommand{\priorMean}{\mathbf{m}}
\newcommand{\priorCov}{\mathbf{S}}
\newcommand{\priorVar}{\eta}
\newcommand{\postMean}{\boldsymbol{\mu}}
\newcommand{\postCov}{\boldsymbol{\Sigma}}
\newcommand{\predMean}{\mu_{\rm pred}}
\newcommand{\predVar}{\sigma^2_{\rm pred}}
\newcommand{\predStd}{\sigma_{\rm pred}}

\newcommand{\R}{\mathbb R}
\newcommand{\E}{\mathbb E}
\renewcommand{\P}{\mathbb P}

\newcommand{\One}[1]{{\mathbb I}{\{#1\}}}
\newcommand{\norm}[1]{\left\Vert#1\right\Vert}
\newcommand{\expLoss}{\loss_{\rm E}}
\newcommand{\eqdef}{\triangleq}
\newcommand{\param}{\theta}
\newcommand{\params}{{\boldsymbol \theta}}
\newcommand{\coinParam}{\param}
\newcommand{\numHeads}{{N_H}}
\newcommand{\numTails}{{N_T}}
\newcommand{\likelihood}{L}
\newcommand{\loglik}{\ell}
\newcommand{\mean}{\mu}
\newcommand{\std}{\sigma}
\newcommand{\coinParamML}{\hat{\theta}_{\rm ML}}
\newcommand{\coinParamPred}{\theta_{\rm pred}}
\newcommand{\coinParamMAP}{\hat{\theta}_{\rm MAP}}
\newcommand{\priorStd}{\std_{\rm pri}}
\newcommand{\postStd}{\std_{\rm post}}
\newcommand{\meanML}{\hat{\mu}_{\rm ML}}
\newcommand{\stdML}{\hat{\sigma}_{\rm ML}}
\newcommand{\meanMAP}{\hat{\mu}_{\rm MAP}}
\newcommand{\stdMAP}{\hat{\sigma}_{\rm MAP}}
\newcommand{\paramsML}{\hat{\params}_{\rm ML}}
\newcommand{\paramsMAP}{\hat{\params}_{\rm MAP}}
\def\bW{{\mathbf{u}}}
\renewcommand{\bm}{\mathbf{m}}
\renewcommand{\bA}{\mathbf{A}}
\renewcommand{\bb}{\mathbf{b}}
\renewcommand{\bU}{\mathbf{U}}
\renewcommand{\bW}{\mathbf{W}}
\renewcommand{\bw}{\mathbf{w}}
\newcommand{\beps}{\mathbf{\epsilon}}
\DeclareMathOperator*{\Cov}{Cov}

\newcommand\cceq{\stackrel{\mathclap{\tiny\mbox{for $c=1$}}}{=}}

\newsavebox{\zerobox}
\newenvironment{nospace}
{\par\edef\theprevdepth{\the\prevdepth}\nointerlineskip
  \setbox\zerobox=\vtop to 0pt\bgroup
  \hrule height0pt\kern\dimexpr\baselineskip-\topskip\relax
}
{\par\vss\egroup\ht\zerobox=0pt \wd\zerobox=0pt \dp\zerobox=0pt
  \box\zerobox}

\usepackage{soul}
\makeatletter
\let\HL\hl
\renewcommand\hl{%
  \let\set@color\beamerorig@set@color
  \let\reset@color\beamerorig@reset@color
  \HL}
  \makeatother


\title[STA437-Week1]{STA 437/2005: \\ Methods for Multivariate Data}
\subtitle[]{Week 4: Gaussian Processes}
\author[Piotr Zwiernik]{Piotr Zwiernik}
\institute[UofT]{University of Toronto}
\date{}


%\usepackage{Sweave}

\begin{document}

\maketitle

\begin{frame}{Table of contents}
\setbeamertemplate{section in toc}[sections numbered]
\tableofcontents%[hideallsubsections]
\end{frame}

\section{Introduction to Gaussian Processes (GPs)}

\begin{frame}{}
	\begin{center}
		{\Huge \alert{Introduction to GPs}}
	\end{center}
\end{frame}

\begin{frame}{Marginal distribution of MVN}
Consider the following reformulation of the earlier result:\\[3mm]

Suppose $X\sim N_m(\mu,\Sigma)$. Let \alert{$T:=\{1,\ldots,m\}$} and define
\begin{itemize}
	\item  $m:T \to \R$ such that $m(i):=\mu_i$ (mean function)
	\item $k:T\times T\to \R$ such that $k(i,j):=\Sigma_{ij}$ (kernel function)
\end{itemize}
\medskip

Then for every $A=\{t_1,\ldots,t_n\}\subseteq T$, the vector $X_A=(X_{t_1},\ldots,X_{t_n})$ is Gaussian with
\begin{itemize}
	\item The mean $\mu_A$ whose $i$-th entry is $m(t_i)$.
	\item The covariance matrix $\Sigma_{AA}$ whose $(i,j)$-th entry is $k(t_i,t_j)$.
\end{itemize}
\begin{block}{}
	The set $T$ indexes all random variables in the system. \\[2mm]
	For every $A=\{t_1,\ldots,t_n\}\subseteq T$, $(X_{t_1},\ldots,X_{t_n})$ is Gaussian.
\end{block} 
\end{frame}

\begin{frame}{Gaussian Processes - an immediate generalization}
A \textbf{Gaussian Process (GP)} is a generalization of the multivariate normal distribution to a collection of random variables indexed by an \alert{arbitrary} set \( T \).

\begin{alertblock}{Definition}
A Gaussian Process is a collection of random variables $\{X_t\}_{t\in T}$ such that for any finite set of points \( \{t_1, \dots, t_n\} \subset T \), the corresponding vector \( (X_{t_1}, \dots, X_{t_n}) \) follows a multivariate normal distribution.
\end{alertblock}

In what follows we assume $T\subseteq \R^d$ with the Euclidean distance metric.\\[2mm]

Often, the correlation between two variables $X_s$ and $X_t$ will depend on the distance $\|t-s\|$.
\end{frame}

\begin{frame}{The mean and the kernel functions}
	A Gaussian Process is characterized  by:
    \begin{itemize}
        \item A \textbf{mean function} $m:T\to \R$:\quad  \( m(t) = \mathbb{E}[X_t] \)
        \item A \textbf{kernel function} $k:T\times T\to \R$:\quad  \( k(t, t') = \text{Cov}(X_t, X_{t'}) \)
    \end{itemize}
 	Note that $m$ is pretty much arbitrary (often set to be zero) but $k$ is highly constrained:\\[2mm]
 	\begin{alertblock}{Positive semi-definitness:}
 		For any finite set \( \{t_1, \dots, t_n\}\subset T \), the covariance matrix \( \Sigma \) with entries \( \Sigma_{ij} = k(t_i, t_j) \) is positive semi-definite.
 	\end{alertblock}
 We can use feature maps $\psi:\R^d\to \R^p$ to define kernels: 
  $$k(s,t) = \psi(s)^\top\psi(t).$$ 
Feature maps define kernels but not all kernels are like that (this can be generalized to ``infinite dimensional'' feature maps).
\end{frame}


%\begin{frame}{Feature map defines a kernel}
%\begin{itemize}
%\item Let $k(\x,\x') =  \psi(\x)^\top \psi(\x') $
%  \item The kernel matrix is given as $\Sigma_{ij} = k(\x^{(i)},\x^{(j)})$, $\Sigma=\y\y^\top$.
%  \item
%  We show that this matrix is positive semi-definite, $\forall \mathbf u \in \R^N$, 
%  $$
%    \bs u^\top \Sigma \bs u = \bs u^\top \y\y^\top\bs u=(\y^\top \bs u)^\top \y^\top \bs u=\|\y^\top \bs u\|^2\geq0.
%  $$	
%\end{itemize}
%Main points:
%
%\begin{itemize}
%\item Forget the feature map.
%	\item We can directly choose a kernel and work with it!
%	\item The dimension of the feature space does not matter anymore.
%  \item Kernels provide a measure of proximity between $\x$ and $\x'$.
%\end{itemize}
%\end{frame}


%\begin{frame}{Kernels: Examples}
%Example 1:  \\[-.1cm]
%\begin{itemize}
%  \item $D$-dimensional inputs: $\x=(x_1,x_2,...,x_D)^\top$ and $\z =(z_1,z_2,...z_D)^\top$
%  \begin{align*}
%  k(\x,\z) =& (\x^\top\z)^2 = (x_1z_1 + x_2z_2 + ...)^2\\
%  =& x_1^2z_1^2 +2x_1 z_1x_2z_2  + x_2^2z_2^2 + ...\\
%  =& (x_1^2, x_2^2,...,\sqrt{2}x_1x_2, ...)^\top(z_1^2, z_2^2,...,\sqrt{2}z_1z_2,...)\\
%  =&\psi(\x)^\top\psi(\z)
%  \end{align*}
%%  \item Explicit computation $\featureMap(\inputVec)^\top\featureMap(\z)$ takes $O(D^2)$ time, whereas implicit computation $k(\x,\z)$ takes $O(D)$ time.
%\end{itemize}
%\vspace{.3cm}
%Example 2 (Gaussian kernel):
%  $ k(\x,\z) =\exp(- \|\x-\z\|^2/2\sigma^2)$.\\[.1cm]  
%  \begin{itemize}
%  	\item The feature vector has infinite dimension here! (a bit of functional analysis)
%  \end{itemize}
%\end{frame}
%



\begin{frame}{Common Kernels in GPs}
\begin{itemize}
    \item \textbf{Squared Exponential (RBF) Kernel:}
    \[
    k_{\textsc{e}}(t, t') = \sigma^2 \exp\left(-\frac{\|t - t'\|^2}{2\ell^2}\right).
    \]
    \begin{itemize}
        \item Controls smoothness of the functions sampled from the GP.
        \item Length scale \( \ell \): Correlation distance.
        \item Signal variance \( \sigma^2 \): Scale of the output.
    \end{itemize}
    \item \textbf{Mat\'{e}rn Kernel:}
    \[
    k_{\textsc{m}}(t, t') = \sigma^2 \frac{2^{1-\nu}}{\Gamma(\nu)} \left(\sqrt{2\nu} \frac{\|t - t'\|}{\ell}\right)^\nu K_\nu\left(\sqrt{2\nu} \frac{\|t - t'\|}{\ell}\right).
    \]
    \begin{itemize}
        \item \( \nu \): Smoothness parameter.
        \item More flexible than the RBF kernel for modeling rough functions.
    \end{itemize}
\end{itemize}
\end{frame}


\begin{frame}{Constructing kernels from kernels}
Given valid kernels $k_1(\x,\x')$ and $k_2(\x,\x')$, the following kernels will also be valid:
\begin{align*}
	k(\x,\x')&=ck_1(\x,\x')\quad\mbox{for}\;\; c>0,\\
	k(\x,\x')&=f(\x)k_1(\x,\x')f(\x')\\
	k(\x,\x')&=k_1(\x,\x')+k_2(\x,\x')\\
	k(\x,\x')&=k_1(\x,\x')\cdot k_2(\x,\x')\\
	k(\x,\x')&=\x^\top A\x'\qquad (A\mbox{ PSD})\\
	k(\x,\x')&=\exp(k_1(\x,\x'))\\
	k(\x,\x')&=q(k_1(\x,\x'))
\end{align*}
where $q$ polynomial with $\geq 0$ coefficients. 
\end{frame}


\begin{frame}{Modelling with Gaussian processes}
Working with Gaussian Processes we 	fix a kernel function.\\[3mm]

Data: Suppose we observed $(X_{t_1},\ldots,X_{t_n})$ for some $t_1,\ldots,t_n\in T$. \\[3mm]
If the kernel function comes with some hyperparameters $\alpha$, we can learn them maximizing the log-likelihood.
	\begin{itemize}
	\item By definition, $(X_{t_1},\ldots,X_{t_n})$ is MVN with covariance that depends on $\alpha$. 
	\item This may be a complicated optimization procedure.\\[3mm]
	\end{itemize}

Suppose we want to predict the value of the process at some point $t_{n+1}$
	\begin{itemize}
	\item By definition  $(X_{t_1},\ldots,X_{t_n},\alert{X_{t_{n+1}}})$ is jointly Gaussian so simply compute the conditional distribution: $\alert{X_{t_{n+1}}}|X_{t_1},\ldots, X_{t_n}$.
	\item This gives both the point prodiction (the conditional mean) and uncertainty quantification (conditional variance).
\end{itemize}	
\end{frame}


\section{GPs for Spatial Data}

\begin{frame}{}
	\begin{center}
		{\Huge \alert{GPs for Spatial Data}}
	\end{center}
\end{frame}

\begin{frame}{Example: Modeling Spatial Data with GPs}
GPs are widely used in spatial statistics, e.g. temperature across a grid of locations.\\[3mm]
\begin{minipage}{7cm}
	\begin{figure}
	\includegraphics[width=
	\textwidth]{pics/GP_temp.pdf}
\end{figure}
\end{minipage}\begin{minipage}{8cm}
$\bullet$	Grid of $100^2$ points. \\[2mm]
$\bullet$ Fix the exponential kernel $\exp\{-\tfrac12\|\x-\x'\|\}$\\[2mm]
$\bullet$ Compute the $100^2\times 100^2$ covariance matrix\\[2mm]
$\bullet$ Get 1 sample from the corresponding distr.\\[2mm]
\end{minipage}
\medskip

Handling a $10000$-dimensional Gaussian comes with its own computational challenges.  
\end{frame}

\begin{frame}{Spatial GP: Prediction}
We explained how to make a prediction for $X_{t_{n+1}}$. This easily generalizes.\\[3mm]


Suppose we observed the mean zero GP over some locations $\x_{\rm train}$.\\[3mm] 

Our goal is to make predictions over some other points $\x_{\rm test}$
\begin{enumerate}
    \item Combine training and test locations.
    \item Compute the covariance matrix using the kernel function.
    \item Use Gaussian conditioning formulas:
    \begin{align*}
        \mathbb{E}[\x_\text{test} | \x_\text{train}] &= \Sigma_\text{test,train} \Sigma_\text{train,train}^{-1} \x_\text{train}, \\
        \text{Cov}(\x_\text{test} | \x_\text{train}) &= \Sigma_\text{test,test} - \Sigma_\text{test,train} \Sigma_\text{train,train}^{-1} \Sigma_\text{test,train}.
    \end{align*}
\end{enumerate}
\end{frame}




\section{Nonparametric Regression with GPs}


\begin{frame}{}
	\begin{center}
		{\Huge \alert{Nonparametric Regression with GPs}}
	\end{center}
\end{frame}

\begin{frame}{Nonparametric Regression}
GPs can be used for nonparametric regression:
    \[
    y_i = f(\x_i) + \eps_i, \quad \eps_i \sim {N}(0, \sigma^2),\quad i=1,\ldots, n.
    \]
    
Prior over \( f:\R^d\to \R \): GP defined by \( m(\x) \) and \( k(\x, \x') \).
    \begin{itemize}
    \item In this sense GP defines a distribution over (random) functions $f: \R^d\to \R$.
    \end{itemize}    
    \medskip 
    
We have $(f(\x_1),\ldots,f(\x_n))\sim N_n(\mu,\Sigma)$ 
    \begin{itemize}
    \item $\mu_i=m(\x_i)$
    \item $\Sigma_{ij}=k(\x_i,\x_j)$
    \end{itemize}
    
    \begin{alertblock}{}
    	Say $d=1$. Given \( m(x) \) and \( k(x, x') \), how would you plot random samples of the corresponding random functions on $\R$?
    \end{alertblock}

\end{frame}

\begin{frame}{Nonparametric Regression}
Note that $\y=(y_1,\ldots,y_n)=(f(\x_1)+\eps_1,\ldots,f(\x_n)+\eps_n)$.
\bigskip

Consider the underlying Gaussian Process $y(\x)$:
\medskip 

$\bullet$ The mean is $m(\x)$.
\begin{itemize}
	\item  $\E[y(\x_i)]=\E[f(\x_i)+\eps_i]=m(\x_i)$.
\end{itemize}

$\bullet$  The kernel is $k(\x,\x')+\sigma^2\bs 1\{\x=\x'\}$. 
\begin{itemize}
	\item  $\cov[y(\x_i),y(\x_j)]=\cov(f(\x_i)+\eps_i,f(\x_j)+\eps_j)=k(\x_i,\x_j)+\sigma^2\bs 1\{\x_i=\x_j\}$.
\end{itemize}

\begin{alertblock}{}
Given data $(y_1,\x_1)$, \ldots, $(y_n,\x_n)$ we can now easily predict $y$ at any other point $\x$.	
\end{alertblock}
\end{frame}


%\begin{frame}{Code Example: Nonparametric Regression}
%\begin{minted}[frame=single,fontsize=\small]{r}
%# Generate training data
%x_train <- seq(0, 10, length.out = 10)
%y_train <- sin(x_train) + rnorm(length(x_train), sd = 0.1)
%
%# Test points
%x_test <- seq(0, 10, length.out = 100)
%
%# RBF kernel
%k <- function(x, x_prime, length_scale = 1) {
%    exp(-0.5 * (outer(x, x_prime, "-")^2) / length_scale^2)
%}
%
%# Covariance matrices
%K <- k(x_train, x_train)
%K_s <- k(x_train, x_test)
%K_ss <- k(x_test, x_test)
%
%# Posterior mean and covariance
%mu_s <- t(K_s) %*% solve(K) %*% y_train
%\end{minted}
%\end{frame}

\begin{frame}{Illustration}
	\begin{center}
		\includegraphics[width=0.8\textwidth]{pics/GP_nonpreg.pdf}
	\end{center}
\end{frame}

\begin{frame}{Summary}
\begin{itemize}
    \item Gaussian Processes are a versatile tool for regression and spatial modeling.\\[3mm]
    \item Key components:
    \begin{itemize}
        \item Mean function.
        \item Kernel function.\\[3mm]
    \end{itemize}
    \item Takeaway: Conceptually it is not harder than MVNs and the same formulas apply.\\[3mm]
    \item Computational issues can be significant.
\end{itemize}
\end{frame}

\end{document}

