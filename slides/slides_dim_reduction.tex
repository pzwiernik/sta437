\documentclass[11pt,handout,aspectratio=169]{beamer}


%%%%%%%%% GENERAL PACKAGES
%\usepackage{xcolor}
%\usepackage{pdfpages}
%\usetheme[progressbar=frametitle]{metropolis}
%\setbeamercolor{background canvas}{bg=white}
%\usepackage{appendixnumberbeamer}
%\usepackage{booktabs}
%\usepackage[scale=2]{ccicons}
%\usepackage{pgfplots}
%\usepgfplotslibrary{dateplot}
%\usepackage{xspace}
%\newcommand{\themename}{\textbf{\textsc{metropolis}}\xspace}
%\usepackage[absolute,overlay]{textpos}

%%%%%%%%% COLOR THEME

% Define some colors:
\definecolor{DarkFern}{HTML}{407428}
\definecolor{DarkCharcoal}{HTML}{4D4944}
\definecolor{AlertColor}{RGB}{89,124,158}
\definecolor{HighLight}{RGB}{96,95,134}
\definecolor{Important}{RGB}{234,122,133}
\definecolor{Yellow}{HTML}{00539C}
\colorlet{Fern}{DarkFern!85!white}
\colorlet{Charcoal}{DarkCharcoal!85!white}
\colorlet{LightCharcoal}{Charcoal!50!white}
\colorlet{HighLight2}{AlertColor}
\colorlet{DarkRed}{red!70!black}
\colorlet{DarkBlue}{blue!70!black}
\colorlet{DarkGreen}{green!70!black}
\definecolor{RoyalBlue}{HTML}{00539C}
\definecolor{Peach}{HTML}{EEA47F}
\definecolor{ForestGreen}{HTML}{2C5F2D}
\definecolor{MossGreen}{HTML}{E8FCC9}
% Use the colors:
\setbeamercolor{title}{fg=Fern}
\setbeamercolor{frametitle}{fg=MossGreen,bg=ForestGreen}
\setbeamercolor{normal text}{fg=Charcoal!70!black}
\setbeamercolor{block title}{fg=black,bg=Fern!25!white}
\setbeamercolor{block body}{fg=black,bg=Fern!10!white}
\setbeamercolor{block title alerted}{fg=black,bg=DarkRed!25!white}
\setbeamercolor{block body alerted}{fg=black,bg=DarkRed!10!white}
\setbeamercolor{alerted text}{fg=DarkRed}
\setbeamercolor{itemize item}{fg=Charcoal}



%%%%%%%%% OTHER COMMANDS
\newcommand{\indep}{\perp\!\!\! \perp}
\newcommand{\comment}[1]{}
\newcommand{\bs}{\boldsymbol}
\newcommand{\tr}{\text{trace}}
\newcommand{\sgn}{{\rm sgn}}
\def\T{\top}
%\newcommand{\det}{\text{det}}
\newcommand{\var}{\mathrm{var}}
\newcommand{\cC}{{\cal C}}
\newcommand{\cG}{{\cal G}}
\newcommand{\cV}{{\cal V}}
\newcommand{\cE}{{\cal E}}
\newcommand{\cM}{{\cal M}}
\newcommand{\cP}{{\cal P}}
\newcommand{\cX}{{\cal X}}
\newcommand{\cY}{{\cal Y}}
\newcommand{\X}{\mathbf{X}}
\newcommand{\Y}{\mathbf{Y}}
\newcommand{\x}{\mathbf{x}}
\newcommand{\y}{\mathbf{y}}
\newcommand{\z}{\mathbf{z}}

\newcommand{\argmin}{\operatornamewithlimits{argmin}}
\newcommand{\eps}{\varepsilon}
\newcommand{\<}{\langle}
\renewcommand{\>}{\rangle}


\setbeamertemplate{itemize subitem}{\tiny\raise1.5pt\hbox{\donotcoloroutermaths$\blacktriangleright$}}
\setbeamertemplate{itemize subsubitem}{\tiny\raise1.5pt\hbox{\donotcoloroutermaths$\blacktriangleright$}}
\setbeamertemplate{enumerate item}{\insertenumlabel.}
\setbeamertemplate{enumerate subitem}{\insertenumlabel.\insertsubenumlabel}
\setbeamertemplate{enumerate subsubitem}{\insertenumlabel.\insertsubenumlabel.\insertsubsubenumlabel}
\setbeamertemplate{enumerate mini template}{\insertenumlabel}

\newcommand{\TODO}[1]{{\color{red}{[TODO: #1]}}}


\newcommand{\R}{\mathbb R}
\newcommand{\E}{\mathbb E}
\renewcommand{\P}{\mathbb P}


\DeclareMathOperator*{\cov}{cov}


\newsavebox{\zerobox}
\newenvironment{nospace}
{\par\edef\theprevdepth{\the\prevdepth}\nointerlineskip
  \setbox\zerobox=\vtop to 0pt\bgroup
  \hrule height0pt\kern\dimexpr\baselineskip-\topskip\relax
}
{\par\vss\egroup\ht\zerobox=0pt \wd\zerobox=0pt \dp\zerobox=0pt
  \box\zerobox}

\usepackage{soul}
\makeatletter
\let\HL\hl
\renewcommand\hl{%
  \let\set@color\beamerorig@set@color
  \let\reset@color\beamerorig@reset@color
  \HL}
  \makeatother


\title[STA437-Week1]{STA 437/2005: \\ Methods for Multivariate Data}
\subtitle[]{Week 9: Non-linear Dimension Reduction Techniques}
\author[Piotr Zwiernik]{Piotr Zwiernik}
\institute[UofT]{University of Toronto}
\date{}


%\usepackage{Sweave}

\begin{document}

\maketitle


\begin{frame}{Why Principal Component Analysis may not be enough?}
  \textbf{Why go beyond PCA?} \newline
  PCA captures variance through linear projections but struggles with:
  \begin{itemize}
    \item Non-linear relationships.
    \item Complex manifolds.\\[6mm]
  \end{itemize}
  {We explore MDS, UMAP and its relationship with PCA.}
\end{frame}

\begin{frame}{}
	\begin{center}
		\alert{\Huge Multi-dimensional Scaling (MDS)}
	\end{center}
\end{frame}

% Slide: Problem Setup
\begin{frame}{Problem Setup}
Consider a \alert{dissimilarity} matrix $\Delta=(\delta_{ij})\in \R^{n\times n}$: $\delta_{ii}=0$ for all $i$, $\delta_{ij}\geq 0$ for all $i\neq j$. \\[5mm]

In classical MDS: there exist $\x_1,\ldots,\x_n\in \R^m$ such that $\delta_{ij}=\|\x_i-\x_j\|$.\\[5mm]
In general we have $n$ objects and $\delta_{ij}$ is a measure of their dissimilarity (small if similar). There need not a Euclidean distance defining this metric. 
\begin{alertblock}{Multidimensional Scaling}
	Find a configuration of points $\y_1,\ldots,\y_n$ in $\R^d$ ($d<\!\!<n$) such that:
	$$
	\|\y_i-\y_j\|\;\approx\;\delta_{ij}.
	$$ 
\end{alertblock}
\end{frame}

% Slide: Distance Matrix Algebra
\begin{frame}{Classical MDS: $\delta_{ij}=\|\x_i-\x_j\|$}
If $\delta_{ij}=\|\x_i-\x_j\|$, we have:
\[\delta_{ij}^2 \;=\; (\mathbf{x}_i - \mathbf{x}_j)^{\top}(\mathbf{x}_i - \mathbf{x}_j)\;=\;(\X\X^\top)_{i,i}+(\X\X^\top)_{j,j}-2(\X\X^\top)_{i,j}.\]
The Hadamard product $\Delta\odot\Delta=[\delta_{ij}^2]$ can be written as:
	\[\Delta \odot \Delta = \mathrm{diag}(\mathbf{X}\mathbf{X}^{\top}) \mathbf{1} \mathbf{1}^{\top} + \mathbf{1} \mathbf{1}^{\top} \mathrm{diag}(\mathbf{X}\mathbf{X}^{\top}) - 2\mathbf{X}\mathbf{X}^{\top}\]
	Reintroducing the centering matrix $H = I_n - \frac{1}{n}\mathbf{1}\mathbf{1}^{\top}$, we obtain
	\[B \;:=\; -\frac{1}{2} H (\Delta \odot \Delta) H\;=\;H\X (H\X)^\top \;=\;\tilde\X\tilde\X^\top.\]
	This matrix contains all inner products $\tilde\x_i^\top\tilde\x_j$ for $1\leq i,j\leq n$.
\end{frame}

% Slide: Centering the Matrix
\begin{frame}{}

\end{frame}

% Slide: Eigen-decomposition
\begin{frame}{Eigen-decomposition for Dimensionality Reduction}
\[B = V \Lambda V^{\top}\]
\[\mathbf{Y} = U_d \Lambda_d^{1/2}\]
$U_d$: top $d$ eigenvectors.
$\Lambda_d$: top $d$ eigenvalues.
\end{frame}

\begin{frame}{Duality Between MDS and PCA}
Classical MDS and PCA are closely connected. Here is the key insight:
\begin{itemize}
    \item \textbf{PCA}: Finds principal components from the eigenvectors of $(H\mathbf{X})^{\top}H\mathbf{X}$.
    \item \textbf{MDS}: Finds embeddings from the eigenvectors of $H\mathbf{X}(H\mathbf{X})^{\top}$.
\end{itemize}
Both methods rely on the singular value decomposition (SVD) of $H\mathbf{X}$.
\end{frame}

% Slide: Detailed Explanation of Duality
\begin{frame}{Detailed Explanation of Duality}
\textbf{Singular Value Decomposition (SVD):}
\[ H\mathbf{X} = U \tilde{\Lambda}^{1/2} V^{\top} \]
\begin{itemize}
    \item MDS uses $U$ (left singular vectors) and $\tilde{\Lambda}$ (singular values).
    \item PCA uses $V$ (right singular vectors) and $\tilde{\Lambda}$ (singular values).
\end{itemize}
This shows that MDS and PCA are dual methods, analyzing complementary covariance structures.
\end{frame}

% Slide: Key Result
\begin{frame}{Key Result}
\textbf{Theorem:} Classical MDS on distances is equivalent to PCA on the centered data matrix. 
\[ H\mathbf{X}(H\mathbf{X})^{\top} = U \tilde{\Lambda} U^{\top} \]
\[ (H\mathbf{X})^{\top}H\mathbf{X} = V \tilde{\Lambda} V^{\top} \]
\textbf{Conclusion:} The MDS embedding and PCA scores are both derived from $H\mathbf{X}$ but use different components of the SVD.
\end{frame}

\end{document}

