\documentclass[11pt,handout,aspectratio=169]{beamer}


%%%%%%%%% GENERAL PACKAGES

\usepackage{pdfpages}
\usetheme[progressbar=frametitle]{metropolis}
\setbeamercolor{background canvas}{bg=white}
\usepackage{appendixnumberbeamer}
\usepackage{booktabs}
\usepackage[scale=2]{ccicons}
\usepackage{pgfplots}
\usepgfplotslibrary{dateplot}
\usepackage{xspace}
\newcommand{\themename}{\textbf{\textsc{metropolis}}\xspace}
\usepackage[absolute,overlay]{textpos}

%%%%%%%%% COLOR THEME

% Define some colors:
\definecolor{DarkFern}{HTML}{407428}
\definecolor{DarkCharcoal}{HTML}{4D4944}
\definecolor{AlertColor}{RGB}{89,124,158}
\definecolor{HighLight}{RGB}{96,95,134}
\definecolor{Important}{RGB}{234,122,133}
\definecolor{Yellow}{HTML}{00539C}
\colorlet{Fern}{DarkFern!85!white}
\colorlet{Charcoal}{DarkCharcoal!85!white}
\colorlet{LightCharcoal}{Charcoal!50!white}
\colorlet{HighLight2}{AlertColor}
\colorlet{DarkRed}{red!70!black}
\colorlet{DarkBlue}{blue!70!black}
\colorlet{DarkGreen}{green!70!black}
\definecolor{RoyalBlue}{HTML}{00539C}
\definecolor{Peach}{HTML}{EEA47F}
\definecolor{ForestGreen}{HTML}{2C5F2D}
\definecolor{MossGreen}{HTML}{E8FCC9}
% Use the colors:
\setbeamercolor{title}{fg=Fern}
\setbeamercolor{frametitle}{fg=MossGreen,bg=ForestGreen}
\setbeamercolor{normal text}{fg=Charcoal!70!black}
\setbeamercolor{block title}{fg=black,bg=Fern!25!white}
\setbeamercolor{block body}{fg=black,bg=Fern!10!white}
\setbeamercolor{block title alerted}{fg=black,bg=DarkRed!25!white}
\setbeamercolor{block body alerted}{fg=black,bg=DarkRed!10!white}
\setbeamercolor{alerted text}{fg=DarkRed}
\setbeamercolor{itemize item}{fg=Charcoal}



%%%%%%%%% OTHER COMMANDS
\newcommand{\Perp}{\perp\!\!\! \perp}
\newcommand{\Ep}[2]{\ensuremath{E_{#1}\left[{#2}\right]}}
\def\hpY{\mathbf{\bar{\beta}}}
\newcommand{\gaus}[2]{\mathcal{N}\left({#1};\,{#2}\right)}
\newcommand{\comment}[1]{}

\newcommand{\trace}{\text{trace}}
\def\T{\top}
%\newcommand{\det}{\text{det}}
\newcommand{\variance}{\mathrm{Var}}
\def\reals{{\mathbb R}}
\newcommand{\prob}{\mathrm{Pr}}
\newcommand{\zeroVec}{\mathbf{0}}
\newcommand{\zeroMat}{\mathbf{0}}
\newcommand{\onesVec}{\mathbf{1}}
\newcommand{\ident}{\mathbf{I}}
\newcommand{\deriv}{\mathrm{d}}
\newcommand{\transpose}{\top}
\newcommand{\costDeriv}[1]{\overline{#1}}
\newcommand{\lossDeriv}{\costDeriv}
\newcommand{\normal}{\mathcal{N}}
\newcommand{\data}{\mathcal{D}}
%\newcommand{\bm}{{\mathbf{m}}}
\newcommand{\loss}{{\cal L}}
\newcommand{\cG}{{\cal G}}
\newcommand{\cV}{{\cal V}}
\newcommand{\cE}{{\cal E}}
\newcommand{\cP}{{\cal P}}
\newcommand{\X}{{\cal X}}
\newcommand{\Y}{{\cal Y}}
\newcommand{\bK}{\mathbf{K}}
\newcommand{\bX}{\mathbf{X}}
\newcommand{\bY}{\mathbf{Y}}
\newcommand{\bk}{\mathbf{k}}
\newcommand{\bx}{\mathbf{x}}
\newcommand{\by}{\mathbf{y}}
\newcommand{\bhy}{\hat{\mathbf{y}}}
\newcommand{\bty}{\tilde{\mathbf{y}}}
\newcommand{\bG}{\mathbf{G}}
\newcommand{\bI}{\mathbf{I}}
\newcommand{\bg}{\mathbf{g}}
\newcommand{\bS}{\mathbf{S}}
\newcommand{\bs}{\mathbf{s}}
\newcommand{\bM}{\mathbf{M}}
\newcommand{\bw}{\mathbf{w}}
\newcommand{\eye}{\mathbf{I}}
\newcommand{\bU}{\mathbf{U}}
\newcommand{\bV}{\mathbf{V}}
\newcommand{\bW}{\mathbf{W}}
\newcommand{\bn}{\mathbf{n}}
\newcommand{\bv}{\mathbf{v}}
\newcommand{\bq}{\mathbf{q}}
\newcommand{\bR}{\mathbf{R}}
\newcommand{\bi}{\mathbf{i}}
\newcommand{\bj}{\mathbf{j}}
\newcommand{\bp}{\mathbf{p}}
\newcommand{\bt}{\mathbf{t}}
\newcommand{\bJ}{\mathbf{J}}
\newcommand{\bu}{\mathbf{u}}
\newcommand{\bB}{\mathbf{B}}
\newcommand{\bD}{\mathbf{D}}
\newcommand{\bz}{\mathbf{z}}
\newcommand{\bP}{\mathbf{P}}
\newcommand{\bC}{\mathbf{C}}
\newcommand{\bA}{\mathbf{A}}
\newcommand{\bZ}{\mathbf{Z}}
\newcommand{\bff}{\mathbf{f}}
\newcommand{\bF}{\mathbf{F}}
\newcommand{\bo}{\mathbf{o}}
\newcommand{\bc}{\mathbf{c}}
\newcommand{\bm}{\mathbf{m}}
\newcommand{\bT}{\mathbf{T}}
\newcommand{\bQ}{\mathbf{Q}}
\newcommand{\bL}{\mathbf{L}}
\newcommand{\bl}{\mathbf{l}}
\newcommand{\ba}{\mathbf{a}}
\newcommand{\bE}{\mathbf{E}}
\newcommand{\bH}{\mathbf{H}}
\newcommand{\bN}{\mathbf{N}}
\newcommand{\bd}{\mathbf{d}}
\newcommand{\br}{\mathbf{r}}
\newcommand{\be}{\mathbf{e}}
\newcommand{\bb}{\mathbf{b}}
\newcommand{\bh}{\mathbf{h}}
\newcommand{\bhh}{\hat{\mathbf{h}}}

\newcommand{\graph}{{\cal H}}
\newcommand{\bayes}{{\cal B}}
\newcommand{\cx}{{\cal X}}
\newcommand{\cg}{{\cal G}}
\newcommand{\cm}{{\cal M}}
\newcommand{\ci}{{\cal I}}
\newcommand{\ct}{{\cal T}}
\newcommand{\co}{{\cal O}}
\newcommand{\ck}{{\cal K}}
\newcommand{\cu}{{\cal U}}
\newcommand{\cv}{{\cal V}}
\newcommand{\ce}{{\cal E}}
\newcommand{\cf}{{\cal F}}
\newcommand{\cb}{{\cal B}}
\newcommand{\cq}{{\cal Q}}
\newcommand{\cd}{{\cal D}}

\newcommand{\btheta}{\boldsymbol{\theta}}
\newcommand{\bpi}{\boldsymbol{\pi}}
\newcommand{\bphi}{\boldsymbol{\phi}}
\newcommand{\bPhi}{\boldsymbol{\Phi}}
\newcommand{\bmu}{\boldsymbol{\mu}}
\newcommand{\bSigma}{\boldsymbol{\Sigma}}
\newcommand{\bGamma}{\boldsymbol{\Gamma}}
\newcommand{\bbeta}{\boldsymbol{\beta}}
\newcommand{\bomega}{\boldsymbol{\omega}}
\newcommand{\blambda}{\boldsymbol{\lambda}}
\newcommand{\bkappa}{\boldsymbol{\kappa}}
\newcommand{\btau}{\boldsymbol{\tau}}
\newcommand{\balpha}{\boldsymbol{\alpha}}
\def\bgamma{\boldsymbol\gamma}

\newcommand{\argmin}{\operatornamewithlimits{argmin}}

%\newcommand{\animal}[2]{\item[\bf #1] {\em #2}}
 \newcommand{\ikron}[1] {\bI\otimes #1}
  %\newcommand{\val}{\bar{\bx}}
    \newcommand{\train}[1]{{\phi(\bx_{#1})}}
    \newcommand{\ikronval}[1]{(\ikron{\phi(\val_{#1}))}}
\newcommand{\ikronvalT}[1]{(\ikron{\phi(\val_{#1})^T)}}
\newcommand{\ikrontrainT}{(\ikron{\train{i}^T)}}
\newcommand{\ikrontrain}[1]{(\ikron{\train{#1})}}
\newcommand{\ikrontrainAT}{(\ikron{\phi(\bx)^T)}}
\newcommand{\ikrontrainA}{(\ikron{\phi(\bx))}}
  \newcommand{\half}{\frac{1}{2}}
  \newcommand{\con}{C^{(c)}}
    \newcommand{\ig}{\frac{1}{\gamma}}
      \newcommand{\Bi}{\bB^{-1}}
 \newcommand{\kernel}{\hat{\bK}}    
 \newcommand{\ikrontestT}{(\ikron{\test^T)}}
   \newcommand{\test}{\phi(\bx_*)}

% partial derivatives
 \newcommand{\pardev}[2]{\frac{\partial #1}{\partial #2}}
  \newcommand{\dev}[2]{\frac{d #1}{d #2}}
  \newcommand{\dw}{\delta\bw}
  
    \newcommand{\lab}{\mathcal{L}}
      \newcommand{\unlab}{\mathcal{U}}
      
      
  \newcommand{\ind}{1{\hskip -2.5 pt}\hbox{I}}
  
 \newcommand{\ff}[2]{   \cf_{\prec (#1 \rightarrow #2)}}
 \newcommand{\vv}[2]{   \cv_{\prec (#1 \rightarrow #2)}}
  \newcommand{\dd}[2]{   \delta_{#1 \rightarrow #2}}
    \newcommand{\ld}[2]{   \lambda_{#1 \rightarrow #2}}
    \newcommand{\en}[2]{  \bD(#1|| #2)}
       \newcommand{\ex}[3]{  \bE_{#1 \sim #2}\left[ #3\right]} 
       \newcommand{\exd}[2]{  \bE_{#1 }\left[ #2\right]} 
  
%  \newtheorem{theorem}{Theorem}
%\newtheorem{proposition}{Prop}
%\newtheorem{lemma}{Lemma}
%\newtheorem{lemma-ap}{Lemma}
%\newtheorem{definition}{Definition}
%\newtheorem{corollary}{Corollary}
%\newtheorem{claim}{Claim}
%\newtheorem{claim-ap}{Claim}
%\newcommand{\argmin}[1]{\underset{#1}{\mathrm{argmin}} \:}
\newcommand{\argmax}[1]{\underset{#1}{\mathrm{argmax}} \:}
\DeclareMathOperator*{\Max}{max}
\def\eop {{\noindent\framebox[0.5em]{\rule[0.25ex]{0em}{0.75ex}}}}

\newcommand{\tr}[1]{\ensuremath{\mathrm{tr}\left(#1\right)}}
\def\Xdim{{d}}
\def\Ydim{{D}}
\def\Zdim{{S}}

\setbeamertemplate{itemize subitem}{\tiny\raise1.5pt\hbox{\donotcoloroutermaths$\blacktriangleright$}}
\setbeamertemplate{itemize subsubitem}{\tiny\raise1.5pt\hbox{\donotcoloroutermaths$\blacktriangleright$}}
\setbeamertemplate{enumerate item}{\insertenumlabel.}
\setbeamertemplate{enumerate subitem}{\insertenumlabel.\insertsubenumlabel}
\setbeamertemplate{enumerate subsubitem}{\insertenumlabel.\insertsubenumlabel.\insertsubsubenumlabel}
\setbeamertemplate{enumerate mini template}{\insertenumlabel}

\newcommand{\book}[1]{{\it{#1}}}

\newcommand{\high}[1]{{\color{blue}{#1}}}
\newcommand{\raquel}[1]{{\color{red}{#1}}}

\newcommand{\Real}{\mathbb{R}}
\newcommand{\TODO}[1]{{\color{red}{[TODO: #1]}}}

\newcommand{\dataGenDist}{p_{\rm sample}}
\newcommand{\trainDist}{p_{\rm dataset}}
\DeclareMathOperator*{\Var}{Var}
\newcommand{\given}{\,|\,}
\newcommand{\expect}{\mathbb{E}}

\newcommand{\dataIdx}{i}
\newcommand{\featIdx}{j}
\newcommand{\dimIdx}{\featIdx}
\newcommand{\paramIdx}{\dimIdx}
\newcommand{\hidIdx}{i}
\newcommand{\classIdx}{k}
\newcommand{\outputIdx}{k}
\newcommand{\classIdxTwo}{\ell}
\newcommand{\featIdxTwo}{j^\prime}
\newcommand{\nfeat}{D}
\newcommand{\ndim}{\nfeat}
\newcommand{\ndata}{N}
\newcommand{\numClasses}{K}
\newcommand{\nout}{\numClasses}
\newcommand{\layerIdx}{\ell}
\newcommand{\numLayers}{L}
\newcommand{\nhid}{M}
\newcommand{\timeIdx}{t}
\newcommand{\ntime}{T}
\newcommand{\contextLen}{K}


\newcommand{\inputIJ}[2]{x^{(#1)}_{#2}}
\newcommand{\inputI}[1]{{\bf x}^{(#1)}}
\newcommand{\inputJ}[1]{x_{#1}}
\newcommand{\inputVec}{{\bf x}}
\newcommand{\inputVecT}[1]{\inputVec^{(#1)}}
\newcommand{\inputVecI}[1]{\inputVec^{(#1)}}
\newcommand{\inputUni}{x}
\newcommand{\inputUniI}[1]{x^{(#1)}}
\newcommand{\inputUniT}[1]{x^{(#1)}}
\newcommand{\inputMatrix}{\mathbf{X}}
\newcommand{\inputMatrixT}[1]{\inputMatrix^{(#1)}}
\newcommand{\targetI}[1]{t^{(#1)}}
\newcommand{\target}{t}
\newcommand{\targetK}[1]{\target_{#1}}
\newcommand{\targets}{\mathbf{t}}
\newcommand{\prediction}{y}
\newcommand{\predictionI}[1]{y^{(#1)}}
\newcommand{\predictionK}[1]{y_{#1}}
\newcommand{\predictionT}[1]{y^{(#1)}}
\newcommand{\predictions}{\mathbf{y}}
\newcommand{\predictionMatrix}{\mathbf{Y}}
\newcommand{\predictionMatrixT}[1]{\predictionMatrix^{(#1)}}
\newcommand{\intermediate}{z}
\newcommand{\intermediateI}[1]{\intermediate^{(#1)}}
\newcommand{\intermediateT}[1]{\intermediate^{(#1)}}
\newcommand{\intermediateK}[1]{\intermediate_{#1}}
\newcommand{\intermediates}{\mathbf{z}}
\newcommand{\intermediateMatrix}{\mathbf{Z}}
\newcommand{\intermediateMatrixT}[1]{\intermediateMatrix^{(#1)}}
\newcommand{\outIntermediate}{r}
\newcommand{\outIntermediateT}[1]{r^{(#1)}}
\newcommand{\outIntermediateK}[1]{\outIntermediate_{#1}}
\newcommand{\outIntermediates}{\mathbf{r}}
\newcommand{\outIntermediateMat}{\mathbf{R}}
\newcommand{\outIntermediateMatrix}{\mathbf{R}}
\newcommand{\outIntermediateMatrixT}[1]{\outIntermediateMatrix^{(#1)}}
\newcommand{\hiddenI}[1]{h_{#1}}
\newcommand{\hiddenT}[1]{h^{(#1)}}
\newcommand{\hiddenIT}[2]{h_{#1}^{(#2)}}
\newcommand{\hiddenLI}[2]{h_{#2}^{(#1)}}
\newcommand{\hiddens}{\mathbf{h}}
\newcommand{\hiddensL}[1]{\hiddens^{(#1)}}
\newcommand{\hiddensT}[1]{\hiddens^{(#1)}}
\newcommand{\hiddenMatrix}{\mathbf{H}}
\newcommand{\hiddenMat}{\hiddenMatrix}
\newcommand{\hiddenMatrixT}[1]{\hiddenMatrix^{(#1)}}
\newcommand{\hiddenMatL}[1]{\hiddenMat^{(#1)}}
\newcommand{\weights}{{\bf w}}
\newcommand{\weightsLS}{{\bf w}^{\text{LS}}}
\newcommand{\weightsMLE}{{\bf w}^{\text{MLE}}}
\newcommand{\weightsHat}{\hat{\weights}}
\newcommand{\weightsL}[1]{\weights^{(#1)}}
\newcommand{\weightJ}[1]{w_{#1}}
\newcommand{\weightLIJ}[3]{w^{(#1)}_{#2 #3}}
\newcommand{\weightLKI}[3]{w^{(#1)}_{#2 #3}}
\newcommand{\weightLJ}[2]{w^{(#1)}_{#2}}
\newcommand{\weightKJ}[2]{w_{#1 #2}}
\newcommand{\weightIJ}{\weightKJ}
\newcommand{\weightUni}{w}
\newcommand{\weightMat}{\mathbf{W}}
\newcommand{\weightMatL}[1]{\weightMat^{(#1)}}
\newcommand{\bias}{b}
\newcommand{\biasLI}[2]{\bias^{(#1)}_{#2}}
\newcommand{\biasLK}{\biasLI}
\newcommand{\biasL}[1]{\bias^{(#1)}}
\newcommand{\biasK}[1]{\bias_{#1}}
\newcommand{\biasJ}[1]{\bias_{#1}}
\newcommand{\biases}{\mathbf{b}}
\newcommand{\biasesL}[1]{\biases^{(#1)}}
\newcommand{\threshold}{r}
\newcommand{\featureJ}[1]{\psi_{#1}}
\newcommand{\featureVec}{{\boldsymbol \psi}}
\newcommand{\lossI}[1]{\loss^{(#1)}}
\newcommand{\zeroOneLoss}{\loss_{\rm 0-1}}
\newcommand{\squaredErrorLoss}{\loss_{\rm SE}}
\newcommand{\crossEntropyLoss}{\loss_{\rm CE}}
\newcommand{\logisticCrossEntropyLoss}{\loss_{\rm LCE}}
\newcommand{\softmaxCrossEntropyLoss}{\loss_{\rm SCE}}
\newcommand{\hingeLoss}{\loss_{\rm H}}
\newcommand{\cost}{\mathcal{J}}
\newcommand{\regularizer}{\mathcal{R}}
\newcommand{\lrate}{\alpha}
\newcommand{\learningRate}{\lrate}
\newcommand{\featureMap}{{\boldsymbol \psi}}
\newcommand{\featureMapJ}[1]{\psi_{#1}}
\newcommand{\sigmoid}{\sigma}
\newcommand{\logistic}{\sigmoid}
\newcommand{\activationFunction}{\phi}
\newcommand{\activationFunctionL}[1]{\activationFunction^{(#1)}}
\newcommand{\activationFunctionTwo}{\psi}
\newcommand{\parityFunction}{f_{\rm par}}
\newcommand{\function}{f}
\newcommand{\functionL}[1]{\function^{(#1)}}
\newcommand{\indicatorOf}[1]{\mathbbm{1}_{#1}}
\newcommand{\softmax}{\mathrm{softmax}}
\newcommand{\weightCost}{\lambda}
\newcommand{\genCost}{\mathcal{C}}
\newcommand{\momentumVec}{\mathbf{p}}
\newcommand{\momentumJ}[1]{p_{#1}}
\newcommand{\momentumParam}{\mu}
\newcommand{\genParams}{{\boldsymbol \theta}}
\newcommand{\genParamJ}[1]{\theta_{#1}}
\newcommand{\pData}{p_{\mathcal{D}}}
\newcommand{\bestPrediction}{\prediction_\star}

\newcommand{\obs}{\mathbf{x}}
\newcommand{\obsJ}[1]{x_{#1}}
\newcommand{\obsI}[1]{\obs^{(#1)}}
\newcommand{\pfn}{\mathcal{Z}}
\newcommand{\happiness}{H}
\newcommand{\latents}{\mathbf{z}}

\newcommand{\state}{\mathbf{s}}
\newcommand{\stateT}[1]{\state_{#1}}
\newcommand{\act}{\mathbf{a}}
\newcommand{\actT}[1]{\act_{#1}}
\newcommand{\reward}{r}
\newcommand{\policy}{\pi}
\newcommand{\policyParams}{\boldsymbol{\theta}}
\newcommand{\policyTh}{{\policy_{\policyParams}}}
\newcommand{\MDP}{\mathcal{M}}
\newcommand{\rollout}{\tau}
\newcommand{\expectedReturn}{R}

\newcommand{\discReturn}{G}
\newcommand{\discFactor}{\gamma}
\newcommand{\valueFunc}{V}
\newcommand{\valueFuncPi}{\valueFunc^{\policy}}
\newcommand{\valueFuncPiTh}{\valueFunc^{\policyTh}}
\newcommand{\qFunc}{Q}
\newcommand{\qFuncPi}{\qFunc^{\policy}}
\newcommand{\optPolicy}{\policy^*}
\newcommand{\optQ}{\qFunc^*}

\newcommand{\subspace}{\mathcal{S}}
\newcommand{\projectedInput}{\tilde{\inputVec}}
\newcommand{\projectedInputI}[1]{\projectedInput^{(#1)}}
\newcommand{\codeVec}{\mathbf{z}}
\newcommand{\codeVecI}[1]{\codeVec^{(#1)}}
\newcommand{\dataMean}{\boldsymbol{\mu}}
\newcommand{\dataCov}{\boldsymbol{\Sigma}}
\newcommand{\pcaVec}{\mathbf{u}}

\newcommand{\featureMatrix}{{\boldsymbol \Psi}}
\newcommand{\smootherMatrix}{{\boldsymbol \Omega}}
\newcommand{\smootherMatrixEntry}{\Omega}
\newcommand{\hypothesis}{\mathcal{H}}
\newcommand{\priorMean}{\mathbf{m}}
\newcommand{\priorCov}{\mathbf{S}}
\newcommand{\priorVar}{\eta}
\newcommand{\postMean}{\boldsymbol{\mu}}
\newcommand{\postCov}{\boldsymbol{\Sigma}}
\newcommand{\predMean}{\mu_{\rm pred}}
\newcommand{\predVar}{\sigma^2_{\rm pred}}
\newcommand{\predStd}{\sigma_{\rm pred}}

\newcommand{\R}{\mathbb R}
\newcommand{\E}{\mathbb E}
\renewcommand{\P}{\mathbb P}

\newcommand{\One}[1]{{\mathbb I}{\{#1\}}}
\newcommand{\norm}[1]{\left\Vert#1\right\Vert}
\newcommand{\expLoss}{\loss_{\rm E}}
\newcommand{\eqdef}{\triangleq}
\newcommand{\param}{\theta}
\newcommand{\params}{{\boldsymbol \theta}}
\newcommand{\coinParam}{\param}
\newcommand{\numHeads}{{N_H}}
\newcommand{\numTails}{{N_T}}
\newcommand{\likelihood}{L}
\newcommand{\loglik}{\ell}
\newcommand{\mean}{\mu}
\newcommand{\std}{\sigma}
\newcommand{\coinParamML}{\hat{\theta}_{\rm ML}}
\newcommand{\coinParamPred}{\theta_{\rm pred}}
\newcommand{\coinParamMAP}{\hat{\theta}_{\rm MAP}}
\newcommand{\priorStd}{\std_{\rm pri}}
\newcommand{\postStd}{\std_{\rm post}}
\newcommand{\meanML}{\hat{\mu}_{\rm ML}}
\newcommand{\stdML}{\hat{\sigma}_{\rm ML}}
\newcommand{\meanMAP}{\hat{\mu}_{\rm MAP}}
\newcommand{\stdMAP}{\hat{\sigma}_{\rm MAP}}
\newcommand{\paramsML}{\hat{\params}_{\rm ML}}
\newcommand{\paramsMAP}{\hat{\params}_{\rm MAP}}
\def\bW{{\mathbf{u}}}
\renewcommand{\bm}{\mathbf{m}}
\renewcommand{\bA}{\mathbf{A}}
\renewcommand{\bb}{\mathbf{b}}
\renewcommand{\bU}{\mathbf{U}}
\renewcommand{\bW}{\mathbf{W}}
\renewcommand{\bw}{\mathbf{w}}
\newcommand{\beps}{\mathbf{\epsilon}}
\DeclareMathOperator*{\Cov}{Cov}

\newcommand\cceq{\stackrel{\mathclap{\tiny\mbox{for $c=1$}}}{=}}

\newsavebox{\zerobox}
\newenvironment{nospace}
{\par\edef\theprevdepth{\the\prevdepth}\nointerlineskip
  \setbox\zerobox=\vtop to 0pt\bgroup
  \hrule height0pt\kern\dimexpr\baselineskip-\topskip\relax
}
{\par\vss\egroup\ht\zerobox=0pt \wd\zerobox=0pt \dp\zerobox=0pt
  \box\zerobox}

\usepackage{soul}
\makeatletter
\let\HL\hl
\renewcommand\hl{%
  \let\set@color\beamerorig@set@color
  \let\reset@color\beamerorig@reset@color
  \HL}
  \makeatother


\title[STA437-Week1]{STA 437/2005: \\ Methods for Multivariate Data}
\subtitle[]{Week 5: Non-Gaussian Distributions}
\author[Piotr Zwiernik]{Piotr Zwiernik}
\institute[UofT]{University of Toronto}
\date{}


%\usepackage{Sweave}

\begin{document}

\maketitle

\begin{frame}{Modelling non-Gaussian distributions}
	Gaussian distribution has many properties that makes it very appealing.\\[2mm]
	
	It does however has some limitations:
	\begin{itemize}
		\item Cannot be used to model multimodal populations.
		\item Is not suitable for modelling processes with extreme events. 
		\item Not-suitable for modelling asymmetric distributions. \\[3mm]
	\end{itemize}
	
	Often we want to retain some of the advantages of the Gaussian distribution removing one or all of its limitations.\\[3mm]
	We focus on three some approaches: 
	\begin{itemize}
		\item spherical and elliptical distributions
		\item copula modelling
		\item Gaussian mixtures
	\end{itemize}
\end{frame}


\begin{frame}{Table of contents}
\setbeamertemplate{section in toc}[sections numbered]
\tableofcontents%[hideallsubsections]
\end{frame}


\section{Elliptical distributions}

\begin{frame}{}
	\begin{center}
		{\Huge \alert{Elliptical distributions}}
	\end{center}
\end{frame}

\subsection{Spherical distributions}

\begin{frame}{Why Study Elliptical Distributions?}
    \begin{itemize}
        \item Generalize the multivariate normal distribution.\\[5mm]
        \item Model data with heavy tails or outliers.
        \begin{itemize}
        \item higher probability of extreme events\\[5mm]
        \end{itemize}
        \item Maintain symmetry and linear correlation structures.\\[5mm]
        \item Applications in finance, insurance, and environmental studies.
    \end{itemize}
\end{frame}

% Slide: Spherical Distributions Definition
\begin{frame}{Spherical Distributions}
\textbf{Orthogonal Matrices:} $O(m) = \{ U \in \mathbb{R}^{m \times m} : U^\top U = I_m \}$.
%  \begin{itemize}
%    \item $U \in O(m)$ implies $U^{-1} = U^\top$.
%    \item Determinants satisfy $|\det(U)| = 1$.
%  \end{itemize}
\begin{alertblock}{Spherical distribution}
A random vector $X \in \mathbb{R}^m$ has a \emph{spherical distribution} if for any $U \in O(m)$:
  \begin{equation*}
    X \overset{d}{=} U X.
  \end{equation*}	
\end{alertblock}
\textbf{\alert{Example:}} $X\sim N_m(\bs 0,I_m)$ or more generally $X\sim N_m(\bs 0,\sigma^2 I_m)$.
\begin{block}{Density generator and dependence on the norm}
	Characteristic function satisfies: $\psi_X(\bs t)=\psi_{UX}(\bs t)=\psi_X(U^\top \bs t)$ and so \textbf{equivalently} $\psi_X(\bs t)$ depends only on $\|\bs t\|$. The same applies to the density: $$f_X(\x)\;=\;h(\|\bs x\|)\qquad\mbox{for some }h\mbox{ (generator)}.$$
\end{block}
%e.g. $X\sim N_m(\bs 0_m,I_m)$ then $h(s)=\tfrac{1}{(2\pi)^{m/2}}e^{-\tfrac12 s}$
  \end{frame}

% Slide: Examples of Spherical Distributions
\begin{frame}{Examples of Spherical Distributions}
The case $X\sim N_m(\bs 0,\sigma^2 I_m)$ has a simple generalization.\\[3mm]
\begin{block}{Spherical scale mixture of normals}
	If $Z \sim N_m(0, I_m)$ and a random variable $\tau > 0$ is independent of $Z$, then:
      \begin{equation*}
        X = \frac{1}{\sqrt{\tau}} Z
      \end{equation*}
      has a spherical distribution.
\end{block}
  \textbf{Indeed:} Let $U \in O(m)$, then 
$$    UX \;=\; \frac{1}{\sqrt{\tau}} UZ \;\overset{d}{=}\; \frac{1}{\sqrt{\tau}} Z \;=\; X.$$
\end{frame}

% Slide: Moment Structure of Spherical Distributions
\begin{frame}{Moment Structure of Spherical Distributions}
\begin{alertblock}{Spherical symmetry implies:}
  $\bullet$\quad      $\mu\;=\;\mathbb{E}[X] \;=\; 0$, \\
  $\bullet$\quad      $\Sigma\;=\;\var(X) \;=\; c I_m$, \quad for some $c \geq 0$. 
      \end{alertblock}
      \medskip
      
      \textbf{Indeed:} Let $\Sigma=\textcolor{blue}{U\Lambda U^\top} $ be the spectral decomposition.
      \begin{itemize}
      	\item $\Sigma=\var(X)=\var(VX)=V \var(X)V^\top=V\textcolor{blue}{U\Lambda U^\top} V^\top$ for any $V\in O(m)$.
      	\item take $V=U^\top$ to show that $\Sigma$ must be diagonal, $\Sigma=\Lambda$.
      	\item take $V$ to be all the \alert{permutation matrices} to conclude that $\Lambda=c I_m$.  
      \end{itemize} 
%\bigskip
%
%For $X = \frac{1}{\sqrt{\tau}} Z$ with $Z \sim N(0, I_m)$, $\tau>0$, $\tau\indep Z$:
%      \begin{equation*}
%        \var(X) = \mathbb{E}[\tau^{-1}] I_m.
%      \end{equation*}
%\textbf{Indeed:} $\E[X]=\E[\tfrac{1}{\sqrt{\tau}}Z]=\E[\tfrac{1}{\sqrt{\tau}}]\E[Z]=\bs 0_m$ and so 
%$$\var(X)\;=\;\E XX^\top -\E[X]\E[X]^\top\;=\;\E[\tfrac{1}{\tau}ZZ^\top]\;=\;\E[\tfrac{1}{\tau}]\E[ZZ^\top]\;=\;\E[\tfrac{1}{\tau}]I_m$$
\end{frame}

% Slide: Independence of Norm and Direction
\begin{frame}{Independence of $\|X\|$ and $\frac{X}{\|X\|}$}
\begin{alertblock}{Key Property}
	If $X$ is spherical, the norm $\|X\| = \sqrt{X^\top X}$ is independent of the direction $\frac{X}{\|X\|}$.
\end{alertblock}
  \textbf{Proof Sketch:}
Let $U \in O(m)$. Then:
$$
        \frac{X}{\|X\|} \;\overset{d}{=}\; \frac{UX}{\|UX\|} \;=\; U \frac{X}{\|X\|}.
$$
The vector $\frac{X}{\|X\|}$ is rotationally invariant $\Longrightarrow$ has uniform distribution on the unit sphere (independent of what $\|X\|$ is).
\bigskip

A formal proof uses polar coordinates, see the notes.
\end{frame}

%% Slide: Polar Coordinates
%\begin{frame}{Polar Coordinates}
%  \textbf{Definition:} In $\mathbb{R}^m$, polar coordinates represent $\mathbf{x}$ as:
%  \begin{equation*}
%    \mathbf{x} = r \mathbf{u}(\boldsymbol{\theta}),
%  \end{equation*}
%  where $r = \|\mathbf{x}\|$ is the radial coordinate, and $\boldsymbol{\theta}$ are angular coordinates.
%  \vspace{0.5cm}
%  \textbf{Jacobian Determinant:}
%  \begin{equation*}
%    J(r, \boldsymbol{\theta}) = r^{m-1} \prod_{i=2}^{m-1} \sin^{m-i}(\theta_{i-1}).
%  \end{equation*}
%  \vspace{0.5cm}
%  \textbf{Implication:} If $f(\mathbf{x}) = g(\|\mathbf{x}\|^2)$, then:
%  \begin{equation*}
%    f(\mathbf{x}) d\mathbf{x} = g(r^2) r^{m-1} dr d\boldsymbol{\theta}.
%  \end{equation*}
%\end{frame}

\subsection{Elliptical distributions}

% Slide: Elliptical Distributions
\begin{frame}{Elliptical Distribution $E(\mu,\Sigma)$}
Recall that $Z\sim N_m(\bs 0_m,I_m)$ then $X=\mu+\Sigma^{1/2}Z\sim N_m(\mu,\Sigma)$.
\begin{block}{Elliptical distribution}
A random vector $X \in \mathbb{R}^m$ has an elliptical distribution \alert{$E(\mu,\Sigma)$} if:
  \begin{equation*}
    X = \mu + \Sigma^{1/2} Z,
  \end{equation*}
  where $Z$ is a spherical random vector.	
\end{block}
The density of $X\sim E(\mu,\Sigma)$ is of the form $$f_X(\x)\;=\;c_m \sqrt{\det{\Sigma^{-1}}}h\big((\x-\mu)^\top\Sigma^{-1}(\x-\mu)\big).$$
The generator $h$ controls the shape of the distribution (and its tails in particular). 
\end{frame}

% Slide: Covariance and Correlation
\begin{frame}{Covariance and Correlation in Elliptical Distributions}
$\Sigma$ is called the \textbf{scale matrix}. It is generally not equal to the covariance matrix.
\medskip
      \begin{equation*}
        \mathrm{Var}(X) = c \Sigma, \quad c > 0.
      \end{equation*}

Correlation structure is still governed by $\Sigma$:
$$
R_{ij}\;=\;\frac{c\Sigma_{ij}}{\sqrt{c\Sigma_{ii}c\Sigma_{jj}}}\;=\;\frac{\Sigma_{ij}}{\sqrt{\Sigma_{ii}\Sigma_{jj}}}.
$$
Similarly, if $X\sim E(\mu,\Sigma)$ and $X=(X_A,X_B)$ then 
$$
\E(X_A|X_B=\x_B)\;=\;\E(X_A)-\Sigma_{A,B}\Sigma_{B,B}^{-1}(\x_B-\mu_B)
$$
exactly as in the Gaussian case.
\end{frame}


% Slide: Why Elliptical Distributions?
\begin{frame}{Again: Why Elliptical Distributions?}
    \begin{itemize}
        \item Generalize the multivariate normal distribution.\\[5mm]
        \item Model data with heavy tails or outliers.\\[5mm]
        \item Maintain symmetry and linear correlation structures.\\[5mm]
        \item Applications in finance, insurance, and environmental studies.
    \end{itemize}
\end{frame}

% Slide: Scale Mixtures of Normals
\begin{frame}{Scale Mixtures of Normals (particularly tractable subclass)}
Scale mixture of normals is a special class of elliptical distributions. 
\bigskip

  \textbf{Stochastic representation:}
  \begin{equation*}
    X = \mu + \textcolor{blue}{\frac{1}{\sqrt{\tau}}} \Sigma^{1/2} \textcolor{blue}{Z},
  \end{equation*}
  where $Z \sim N_m(0, I_m)$ and $\tau > 0$ is independent of $Z$.
  \vspace{0.5cm}
  \begin{block}{Special Cases of Scale Mixture of Normals}
  	  \begin{itemize}
    \item $\tau \equiv 1$: Multivariate normal.
    \item $\tau \sim \frac{1}{k} \chi^2_k$: Multivariate $t$-distribution with $k$ degrees of freedom.
    \begin{itemize}
    \item Smaller $k$ means heavier tails. Gaussian is the limit $k\to \infty$.
    \end{itemize}
    \item $\tau \sim \text{Exp}(1)$: Multivariate Laplace.
  \end{itemize}
    \end{block}
\end{frame}

\begin{frame}{Its about the tails (say $m=10$)}
	For scale mixture of normals:
	$$
	Y:=\|X-\mu\|_\Sigma^2\;=\;(X-\mu)^\top \Sigma^{-1}(X-\mu)\;=\;\tfrac{1}{\tau}\|Z\|^2\;\overset{d}{=}\;\tfrac{1}{\tau} \chi^2_m.
	$$
	\begin{center}
		\includegraphics[scale=.7]{pics/SNMtails.pdf}
	\end{center}
\end{frame}

\begin{frame}{}
Some tails are \alert{much} heavier than Gaussian. \\[3mm]

In the plot above we study $Y=\|X-\mu\|_\Sigma^2$ for $X$: normal, multivariate t, Laplace. 
	\begin{table}[h]
    \centering
    \begin{tabular}{lcccc}
        \hline
        \textbf{Case} & $P(Y > 75)$ & $P(Y > 500)$ & $P(Y > 1000)$ & $P(Y > 10000)$ \\
        \hline
        Gaussian & 0.000 & 0.000 & 0.000 & 0.000 \\
        $t_{100}$ & 0.000 & 0.000 & 0.000 & 0.000 \\
        $t_{20}$ & 0.000 & 0.000 & 0.000 & 0.000 \\
        $t_{5}$ & 0.019 & 0.000 & 0.000 & 0.000 \\
        Laplace & 0.124 & 0.020 & 0.010 & 0.001 \\
        $t_{1}$ & 0.277 & 0.109 & 0.077 & 0.024 \\
        \hline
    \end{tabular}
    \caption{Proportion of Samples Exceeding Thresholds}
    \label{tab:proportion_thresholds}
\end{table}
\end{frame}

\begin{frame}[fragile]{Simple illustration}
In the notes we provide an example of four stocks: Apple, Microsoft, Google, Amazon.\\[3mm]
Compare the empirical distribution of the Mahalanobis distance with $\chi^2_4$ (Gaussian).\\[3mm]
\begin{minipage}{7cm}
	\includegraphics[scale=.5]{pics/SMN2.pdf}
\end{minipage}\begin{minipage}{8cm}
	Empirical density seems to be more concentrated around zero.\\[3mm]
	But it has much heavier tails. 
	\begin{itemize}
		\item $\P(\chi^2_4>20)\approx 0$. 
		\item $\P(Y>20)\approx 0.03$
	\end{itemize}
\end{minipage}
\smallskip

This may be much more dramatic for smaller companies.
\end{frame}



\section{Copula models}

\begin{frame}{}
	\begin{center}
		{\Huge \alert{Copula models}}
	\end{center}
\end{frame}

\begin{frame}{Cumulative Distribution Function (CDF)}
	Let $X=(X_1,\ldots,X_m)$ be a random vector.  Its \textbf{CDF} is: $$F(x_1,\ldots,x_m)=\P(X_1\leq x_1,X_2\leq x_2,\ldots,X_m\leq x_m).$$
	Marginal CDF: $F_1(x_1)=\P(X_1\leq x_1)=\lim_{x_2\to \infty}\cdots\lim_{x_m\to\infty}F(x_1,x_2,\ldots,x_m)$.
	(similar for any other margin)\\[5mm]	
	\begin{block}{}
	If $f$ is the corresponding density of $X$, then:
	$$
	f(x_1,\ldots,x_m)\;=\;\frac{\partial^m}{\partial x_1\cdots\partial x_m}F(x_1,\ldots,x_m)
	$$
	
	$$
	F(x_1,\ldots,x_m)\;=\;\int_{-\infty}^{x_1}\cdots \int_{-\infty}^{x_m} f(y_1,\ldots,y_m){\rm d}y_1\cdots {\rm d} y_m. 
	$$		
	\end{block}
		If $U\sim U[0,1]$ then $F(u)=u$ for all $u\in [0,1]$.

\end{frame}


\begin{frame}{What is a Copula?}
\begin{itemize}
    \item A \textbf{copula} is a function that captures the \textbf{dependence structure} between random variables, separate from their marginal distributions.
\end{itemize}
\begin{alertblock}{Definition}
	A function $C : [0, 1]^m \to [0, 1]$ is a \textbf{copula} if it is a CDF with uniform marginals, that is, $C_1(u_1)=u_1$, \ldots, $C_m(u_m)=u_m$, where $C_i$ are the marginal CDF's.
\end{alertblock}
For example, the copula $C(\bs u)=u_1\cdots u_m$ corresponds to a $m$ independent $U[0,1]$.
\begin{block}{Why use copulas?}
	\begin{itemize}
        \item To model non-Gaussian dependencies.
        \item To analyze dependence independently of marginal behaviors.
    \end{itemize}
\end{block}
\end{frame}

% Slide 2: Sklar's Theorem
\begin{frame}{Sklar's Theorem}
\begin{block}{Theorem (Sklar, 1959)}
Let $X = (X_1, \ldots, X_m)$ be a \textbf{continuous} random vector with joint CDF $F$ and marginals $F_1, \ldots, F_m$. There exists a unique copula $C$ such that:
\begin{equation}\label{eq:copula}
	F(x_1, \dots, x_m) = C(F_1(x_1), \dots, F_m(x_m)). 
\end{equation}
Conversely, given marginals $F_1, \ldots, F_m$ and a copula $C$, $F$ in \eqref{eq:copula} is a CDF of a multivariate distribution with given margins.
\end{block}
\begin{itemize}
    \item $C$ captures \textbf{dependence structure}.
    \item $F_1, \ldots, F_m$ capture marginal behaviors.
\end{itemize}
\end{frame}

% Slide 3: Understanding Sklar's Theorem
\begin{frame}{Understanding Sklar's Theorem}
%    \item When $m = 1$, $C(u) = u$, the identity function on $[0, 1]$.\\[5mm]
\begin{alertblock}{}
If $X$ is continuous with CDF $F$, then $F(X) \sim U(0, 1)$. 	
\end{alertblock}
    \textbf{Proof:} If $X$ is continuous, $F$ is strictly increasing on the support. Hence
    \[
    \mathbb{P}(F(X) \leq u) = \mathbb{P}(X \leq F^{-1}(u)) = F(F^{-1}(u)) = u.
    \]
    \vspace{-7mm}
\begin{block}{}
	Let $X = (X_1, \ldots, X_m)$ with CDF $F$ and margins $F_i$. Define $U_i := F_i(X_i)$.
	\begin{itemize}
		\item The transformed variables $U = (U_1, \ldots, U_m)$ have uniform marginals.
    \[ \mathbb{P}(U_1 \leq u_1, \ldots, U_m \leq u_m) =: C(u_1, \ldots, u_m). \]
    \item Also $C(\bs u)$ is given explicitly in terms of $F$ and $F_i$'s:
    \begin{equation}\label{eq:cop1}
    C(\bs u)=\mathbb{P}(F_1(X_1) \leq u_1, \ldots, F_m(X_m) \leq u_m) = F(F_1^{-1}(u_1),\ldots,F_m^{-1}(u_m))	
\end{equation}
	\end{itemize}
\end{block}
\end{frame}

% Slide 5: Example of a Copula
\begin{frame}{Simple Example of a Copula}
\begin{itemize}
    \item Joint CDF:
    \[ F_{X,Y}(x, y) =
    \begin{cases} 
    0 & x < 0 \text{ or } y < 0, \\
    x^2 y^2 & 0 \leq x, y \leq 1, \\
    1 & x > 1 \text{ and } y > 1, \\
    \min(x^2, y^2) & \text{otherwise}.
    \end{cases} \]
    \item Marginal CDFs:
    \[ F_X(x) = x^2, \quad F_Y(y) = y^2 \quad \text{for } 0 \leq x, y \leq 1. \]
    \item Copula:
    \[ C(u, v) = uv \quad \text{if } u, v \leq 1. \]
\end{itemize}
\end{frame}


\begin{frame}{Sampling}
	Fix a copula $C(\bs u)$ and suppose we can sample from it.
	\begin{block}{Transform the copula sample}
		Consider a sample $\bs u^{(1)}$, \ldots, $\bs u^{(n)}$ from the copula.\\[3mm]
		
		Transform the data to have the right marginals $F_1,\ldots,F_m$:
		$$
		\x^{(t)}_i \;:=\; F_i^{-1}(\bs u^{(t)}_i)\qquad\mbox{for all }i=1,\ldots,m, t=1,\ldots,n.
		$$
		The sample $\x^{(1)},\ldots,\x^{(n)}$ has the right marginals and the right dependence structure. 
		$$
		\P(\x_i^{(t)}\leq s_i)=\P(F_i^{-1}(\bs u_i^{(t)})\leq s_i)=\P(\bs u_i^{(t)}\leq F_i(s_i))=F_i(s_i).
		$$
	\end{block}
	\medskip
	We will later show how to sample from some popular copula models. 
\end{frame}


% Slide 6: Invariance under Transformations
\begin{frame}{Invariance under Monotone Transformations}
\begin{alertblock}{Copulas are invariant under monotone transformations. }
	Consider $Y_i := f_i(X_i)$, where $f_i$ are strictly increasing transformations. Then the copula of $X$ is the same as the copula of $Y$. 
\end{alertblock}
Proof outline: Let $G$ be the CDF of $Y$ and $G_i$ the marginal CDF of $Y_i$
    \begin{itemize}
    \item By \eqref{eq:cop1}, equiv. show $F(F_1^{-1}(u_1), \ldots, F_m^{-1}(u_m))=G(G_1^{-1}(u_1), \ldots, G_m^{-1}(u_m))$
        \item $G_i(y_i)=\P(Y_i\leq y_i)=\P(f_i(X_i)\leq y_i)=\P(X_i\leq f_i^{-1}(y_i))=F_i(f_i^{-1}(y_i))$ and, in particular, $G_i^{-1}=f_i\circ F_i^{-1}$.
        \item Thus, $\{Y_i\leq G_i^{-1}(u_i)\}=\{f(X_i)\leq f_i(F^{-1}_i(u_i))\}=\{X_i\leq F^{-1}_i(u_i)\}$ and so  
      \begin{eqnarray*}
      	&&G(G_1^{-1}(u_1),\ldots, G_m^{-1}(u_m))= \P\left(\bigcap_{i=1}^m \{Y_i\leq G^{-1}_i(u_i)\}\right)\\
      	&=&\P\left(\bigcap_{i=1}^m \{X_i\leq F^{-1}_i(u_i)\}\right)=F(F_1^{-1}(u_1), \ldots, F_m^{-1}(u_m)).
      \end{eqnarray*} 
    \end{itemize}
\end{frame}

% Slide 7: Copula Density
\begin{frame}{Density of a Copula}
The PDF of a copula $C$ is obtained by differentiating its CDF:
    \[ c(\mathbf{u}) = \frac{\partial^m C(\mathbf{u})}{\partial u_1 \cdots \partial u_m}. \]
    
    Recall $C(\bs u)=F(F_1^{-1}(u_1),\ldots, F_m^{-1}(u_m))$. Using the chain rule:
    \[ c(\mathbf{u}) = \frac{f(\mathbf{x})}{\prod_{i=1}^m f_i(x_i)},\qquad \mbox{where } x_i=F_i^{-1}(u_i)\mbox{ for all }i \]
    where $f$ is the joint density and $f_i$ are marginal densities.
    \bigskip
    
    e.g. $C(\bs u)=u_1\cdots u_m$ is the CDF of independent $U_i\sim U(0,1)$. The density is uniform on $[0,1]^m$. Given margins $f_i$, we get $f(\x)=\prod_i f_i(x_i)$.
\end{frame}

% Slide 8: Gaussian Copula
\begin{frame}{Gaussian Copula}
Gaussian copula is derived from the multivariate normal distribution $X\sim N_m(\mu,\Sigma)$.\\[3mm]
By monotone invariance, we can assume $\E X_i=0$, $\var(X_i)=1$
\begin{itemize}
	\item  $\mu=0$, $\Sigma$ is a correlation matrix,
	\item each $X_i\sim N(0,1)$. 
\end{itemize}
 

Let $\Phi$ be the CDF of $N(0,1)$ with PDF $\phi$. Let $f(\x;\Sigma)$ be the PDF of $N_m(\bs 0,\Sigma)$.
\begin{alertblock}{The density of the Gaussian copula $C(\bs u;\Sigma)$}
	Using the general formula, we get:
	\[ c(\mathbf{u}; \Sigma) \;=\; \frac{f(\x;\Sigma)}{\prod_{i=1}^m \phi(x_i)}\;=\; \det(\Sigma)^{-1/2} \exp\left(-\frac{1}{2} \mathbf{x}^\top (\Sigma^{-1} - I_m) \mathbf{x}\right), \]
	where $\x=(\Phi^{-1}(u_1),\ldots,\Phi^{-1}(u_m))$.
\end{alertblock}
 \end{frame}

\begin{frame}{Sampling from the Gaussian copula $C(\bs u;\Sigma)$}
Let $\Sigma$ be a correlation matrix.\\[3mm]
\begin{itemize}
	\item Sample $\z^{(1)},\ldots,\z^{(n)}\sim N_m(\bs 0,\Sigma)$.\\[3mm]
	\item Transform $\bs u^{(t)}_i=\Phi(\z^{(t)}_i)$ for all $i=1,\ldots,m$ and $t=1,\ldots,n$. \\[3mm]
	\item The sample $\bs u^{(1)},\ldots,\bs u^{(n)}$ comes from the Gaussian copula. \\[3mm]
\end{itemize}	
	As described earlier, we can now transform this sample to get arbitrary margins.\\[3mm]
	\begin{block}{}
		The Gaussian copula model can still handle quite general distributions. Yet, it retains some of the computational advantages of the Gaussian distribution.
	\end{block}

\end{frame}

\begin{frame}{Steps to Estimate a Copula: normalize data}
	Given data $\x^{(1)},\ldots,\x^{(n)}$, start by fixing a copula model (e.g. Gaussian).\\[3mm]
	
	We assume the CDF $F$ of the data satisfies $F(\x)=C(F_1(x_1),\ldots,F_m(x_m))$.\\[3mm]
	
	However, \alert{the margins $F_i$ are not known!}\\[3mm]
	
	Given a sample $\x_i^{(1)},\ldots,\x_i^{(n)}$ of $X_i$ we compute the \textbf{empirical CDF} (proxy for $F_i$)
	$$
	\widehat F_i(x_i)\;:=\;\frac1n \sum_{t=1}^n \bs 1\{\x_i^{(t)}\leq x_i\}.
	$$
	\vspace{-5mm}
\begin{alertblock}{}
	Transform, the data using the empirical CDFs
$$
\bs u^{(t)}_i\;=\;\widehat F_i(\x^{(t)}_i).
$$
This transforms the data matrix $\X$ to $\bs U$ with uniform marginals. 
\end{alertblock}\end{frame}


\begin{frame}{Steps to Estimate a Copula: Fit the copula family}
In the next step, we fit the data to the given copula family. \\[3mm]

Often this is done by maximizing the log-likelihood $\sum_{t=1}^n \log c(\bs u^{(t)})$.\\[3mm]

In the case of the Gaussian copula $C(\bs u;\Sigma)$:
\begin{itemize}
	\item Transform the data to standard Gaussian margins: $\y^{(t)}_i=\Phi^{-1}(\bs u^{(t)}_i)$.
	\item Fit the Gaussian likelihood for $N_m(\bs 0,\Sigma)$ with the sample covariance  $S_n=\tfrac1n \Y^\top \Y$.
\end{itemize}
\end{frame}


\begin{frame}{Steps to Estimate a Copula: Evaluate the fit}
As the last step, compare the fitted copula model with the observed data. Check whether the copula captures the dependence structure accurately.\\[3mm]
We can generate samples from the fitted Gaussian copula.
\end{frame}




% Slide 9: Applications of Copulas
\begin{frame}{Applications of Copulas}
\begin{itemize}
    \item \textbf{Finance:} Modeling dependencies in asset returns.
    \item \textbf{Insurance:} Understanding risks in correlated claims.
    \item \textbf{Environmental Science:} Joint modeling of extreme events (e.g., floods).
    \item \textbf{Medical Statistics:} Modeling dependence in survival times.
\end{itemize}
\end{frame}


\section{Gaussian mixture models}

\begin{frame}{}
	\begin{center}
		{\Huge \alert{Gaussian mixtures}}
	\end{center}
\end{frame}


% Slide 1: Definition of GMMs
\begin{frame}{Mixture of Gaussians}
We combine simple models into a complex model by taking a mixture of $K$ multivariate Gaussian densities of the form:
$$
p(x)\;=\;\sum_{k=1}^K \pi_k N_m(x|\mu_k,\Sigma_k),
$$
where $\pi_k\geq 0$, $\sum_{k=1}^K\pi_k=1$, and $N_m(x|\mu_k,\Sigma_k)$ is the $m$-dim Gaussian density.
\begin{itemize}
	\item Each Gaussian component has its own mean vector $\mu_k$ and covariance matrix $\Sigma_k$.
	\item The parameters $\pi_k$ are called the mixing coefficients.
\end{itemize}
\pause
\begin{minipage}{7cm}{}
Example:
\begin{itemize}
	\item $K=3$ (three Gaussian components)
	\item $m=1$ (univariate Gaussians)
\end{itemize}
\end{minipage}\begin{minipage}{5cm}{}
	\begin{figure}
\includegraphics[scale=.25]{./pics/multimodal.jpg}
\end{figure}
\end{minipage}
\end{frame}

\begin{frame}{The crabs from Naples bay}

%	{In 1892, scientists collected data on populations of the crab and observed that the ratio of forehead width to the body length  actually showed a highly skewed distribution.}\\[.2cm]

%	\begin{figure*}
\begin{minipage}{5.5cm}{}
	\includegraphics[scale=.29]{pics/crabs.png}
		\end{minipage}\begin{minipage}{9cm}
			 In 1892, scientists collected data on populations of the crab and observed that the ratio of forehead width to the body length  actually showed a highly skewed distribution.\\[2mm] {\small Source: \textit{On Certain Correlated Variations in Carcinus maenas} (1893) W. F.  Weldon.}	
		\end{minipage}

{ They wondered whether this distribution could be the result of the population being a mix of two different normal distributions (two sub-species).}
\smallskip 

{ In \textbf{1894}, Karl Pearson proposed a method to fit this model (\href{https://archive.org/details/philtrans02543681}{\textcolor{blue}{read here}}), whose modern version is the ``method of moments''. The method involved solving a higher order polynomial.}
%	\end{figure*}
\end{frame}

\begin{frame}{}
%\frametitle{Mixture of Gaussians: 2D example}
\begin{figure}
\includegraphics[page=2,width=4.8in,trim={0 0 0 3cm},clip]{pics/raw.pdf}
\end{figure}
\end{frame}

% Slide 2: Why use Gaussian Mixtures?
\begin{frame}{Why Use Gaussian Mixtures?}
Gaussian Mixture Models (GMMs) are widely used because of their:
\begin{itemize}
    \item \textbf{Flexibility:} Ability to model complex data distributions.
    \item \textbf{Multimodality:} Handles datasets with multiple clusters or modes.
    \item \textbf{Interpretability:} Each Gaussian component represents a sub-population with interpretable parameters.
    \item \textbf{Clustering Applications:} GMMs are a natural probabilistic method for clustering.
\end{itemize}

\textbf{Special Case:}
For simplicity, in clustering, we often assume \( \Sigma_k = \Sigma \) for all \( k \).
\end{frame}

\begin{frame}
\frametitle{Mixture of Gaussians as a latent variable model}
Recall: \textcolor{blue}{$p(x)\;=\;\sum_{k=1}^K \pi_k N_m(x|\mu_k,\Sigma_k)$}.\\[.3cm]
\begin{itemize}
	\item Consider a latent variable $z$ with $K$ states $z\in \{1,\ldots,K\}$. 
%	\item For simplicity encode it in a 1-to-K representation:
%	$$
%	z\in \{e_1,\ldots,e_K\},\quad  \mbox{where } e_i=(0,\ldots,0,1,0,\ldots,0).
%	$$
	\item The distribution of $z$ given by the mixing coefficients: $$p(z=k)=\pi_k.$$
	\item Specify the conditional as $p(x|z=k)=N_m(x|\mu_k,\Sigma_k)$ with joint: $$p(x,z=k)\;=\;p(z=k)p(x|z=k)\;=\;\pi_k N_m(x|\mu_k,\Sigma_k).$$
	\item Then the marginal $p(x)$ satisfies $$\textcolor{blue}{p(x)=\sum_{k=1}^K p(x,z=k)\;=\;\sum_{k=1}^K \pi_k N_m(x|\mu_k,\Sigma_k)}.$$
\end{itemize}
%\begin{figure}
%\includegraphics[page=3,width=4.8in,trim={0 0 0 3cm},clip]{./raw.pdf}
%\end{figure}
\end{frame}

\begin{frame}{Yet another illustration}
The quantities $p(z|x)$ are called responsibilities.
\begin{figure}
\includegraphics[page=8,width=4.8in,trim={0 0 0 3cm},clip]{pics/raw.pdf}
\end{figure}
\end{frame}

\begin{frame}{The Likelihood function}
Parameters: $\boldsymbol\pi=(\pi_1,\ldots,\pi_K)$, $\boldsymbol\mu=(\mu_1,\ldots,\mu_K)$, $\boldsymbol\Sigma=(\Sigma_1,\ldots,\Sigma_K)$.

Recall: $\textcolor{blue}{p(x|\boldsymbol\pi,\boldsymbol\mu,\boldsymbol\Sigma)=\sum_{k=1}^K \pi_k N_m(x|\mu_k,\Sigma_k)}$
\medskip

\begin{itemize}
	\item Represent the dataset $\{x_1,\ldots,x_N\}$ as $\boldsymbol X\in \mathbb R^{N\times m}$.
	\item The latent variable is represented by a vector $\boldsymbol z\in \mathbb R^N$.
	\item The log-likelihood takes the form $$\log p(\boldsymbol X|\boldsymbol\pi,\boldsymbol\mu,\boldsymbol\Sigma)=\sum_{n=1}^N\log\left(\sum_{k=1}^K\pi_k N_m(x_n|\mu_k,\Sigma_k)\right)$$
\end{itemize}
%\begin{figure}
%\includegraphics[page=9,width=1.8in,trim={16.5cm 3.5cm 0cm 8cm},clip]{./raw.pdf}
%\end{figure}
\end{frame}

\begin{frame}{Maximum Likelihood ($\boldsymbol \mu$)}
Recall: $\log p(\boldsymbol X|\boldsymbol\pi,\boldsymbol\mu,\boldsymbol\Sigma)=\sum_{n=1}^N\log\left(\sum_{k=1}^K\pi_k N_m(x_n|\mu_k,\Sigma_k)\right)$.
\begin{itemize}
%	\item The log-likelihood takes the form $$\log p(\boldsymbol X|\boldsymbol\pi,\boldsymbol\mu,\boldsymbol\Sigma)=\sum_{n=1}^N\log\left(\sum_{k=1}^K\pi_k N_m(x_n|\mu_k,\Sigma_k)\right)$$
	\item Differentiating wrt $\mu_k$ and setting to zero gives:
	\begin{eqnarray*}
		0&=&\sum_{n=1}^N\frac{\pi_k N(x_n|\mu_k,\Sigma_k)}{\sum_j \pi_j N(x_n|\mu_j,\Sigma_j)}\Sigma_k^{-1}(x_n-\mu_k)\;=\; \sum_{n=1}^N p(z_n=k|x_n)\Sigma_k^{-1}(x_n-\mu_k)\\[2mm]\pause
		&=&\Sigma_k^{-1}\left(\sum_{n=1}^N p(z_n=k|x_n)x_n-\mu_k\textcolor{red!80!green!20!blue}{\sum_{n=1}^N p(z_n=k|x_n)}\right).
	\end{eqnarray*} 	
	\item Equivalently (as $\Sigma_k$ is positive definite) $$\mu_k\;=\;\sum_n \frac{p(z=k|x_n)}{N_k}x_n,\qquad N_k=\textcolor{red!80!green!20!blue}{\sum_n p(z=k|x_n)}.$$
	\item Simple interpretation: the MLE given by the weighted mean of the data weighted by the posterior $p(z=k|x_n)$.
\end{itemize}
%\begin{figure}
%\includegraphics[page=9,width=1.8in,trim={16.5cm 3.5cm 0cm 8cm},clip]{./raw.pdf}
%\end{figure}
\end{frame}



%\begin{frame}
%\frametitle{Maximum Likelihood}
%\begin{figure}
%\includegraphics[page=10,width=4.8in,trim={0 0 0 3cm},clip]{./raw.pdf}
%\end{figure}
%\end{frame}

%\begin{frame}
%\frametitle{Maximum Likelihood}
%\begin{figure}
%\includegraphics[page=11,width=4.8in,trim={0 0 0 3cm},clip]{./raw.pdf}
%\end{figure}
%\end{frame}

\begin{frame}{Maximum Likelihood ($\boldsymbol \Sigma,\boldsymbol \pi$)}
Recall: $\log p(\boldsymbol X|\boldsymbol\pi,\boldsymbol\mu,\boldsymbol\Sigma)=\sum_{n=1}^N\log\left(\sum_{k=1}^K\pi_k N_m(x_n|\mu_k,\Sigma_k)\right)$.
\begin{itemize}
	\item Differentiating wrt $\Sigma_k$ and setting to zero gives:
	$$\Sigma_k\;=\;\sum_n \frac{p(z=k|x_n)}{N_k}(x_n-\mu_k)(x_n-\mu_k)^\top.$$
	\item Again data points weighted by posterior probabilities.\\[.4cm]
	\item Finally, for the weights $\pi_k$ the MLE is
$$
\pi_k\;=\;\frac{N_k}{\sum_{j=1}^K N_j}\;=\;\frac{N_k}{N},\qquad N_k=\sum_n p(z=k|x_n).
$$
\end{itemize}
%\begin{figure}
%\includegraphics[page=9,width=1.8in,trim={16.5cm 3.5cm 0cm 8cm},clip]{./raw.pdf}
%\end{figure}
\end{frame}


%\begin{frame}
%\frametitle{Maximum Likelihood}
%\begin{figure}
%\includegraphics[page=12,width=4.8in,trim={0 0 0 3cm},clip]{./raw.pdf}
%\end{figure}
%\end{frame}

%\begin{frame}
%\frametitle{Maximum Likelihood}
%\begin{figure}
%\includegraphics[page=13,width=4.8in,trim={0 0 0 3cm},clip]{./raw.pdf}
%\end{figure}
%\end{frame}

\begin{frame}{Motivating the EM algorithm}
\begin{itemize}
	\item The MLE \textcolor{red}{does not have a closed form solution.}
	\item The estimates depend on the posterior probabilities $p(z=k|x_n)$, which themselves depend on those parameters.
	\item Indeed, recall that $$p(z=k|x_n)\;=\;\frac{\pi_k N_m(x_n|\mu_k,\Sigma_k)}{\sum_{j=1}^K \pi_j N_m(x_n|\mu_j,\Sigma_j)}.
$$
\item Iterative solution (EM algorithm):
\begin{itemize}
\item Initialize the parameters to some values.
\item [\textcolor{red}{E-step}] Update the posteriors $p(z=k|x_n)$.
\item [\textcolor{red}{M-step}] Update model parameters $\boldsymbol\pi,\boldsymbol\mu,\boldsymbol\Sigma$.
\item Repeat.
\end{itemize}
\end{itemize}
%\begin{figure}
%\includegraphics[page=14,width=4.8in,trim={0 0 0 3cm},clip]{./raw.pdf}
%\end{figure}
\end{frame}

%\begin{frame}
%\frametitle{Maximum Likelihood}
%\begin{figure}
%\includegraphics[page=15,width=4.8in,trim={0 0 0 3cm},clip]{./raw.pdf}
%\end{figure}
%\end{frame}

\begin{frame}[label=GMEM]{EM algorithm for Gaussian mixtures}
	\begin{itemize}
		\item Initialize $\boldsymbol\pi,\boldsymbol\mu,\boldsymbol\Sigma$.
		\item \textcolor{red}{E-step}: for each $k,n$ compute the posterior probabilities $$p(z=k|x_n)\;=\;\frac{\pi_k N_m(x_n|\mu_k,\Sigma_k)}{\sum_{j=1}^K \pi_j N_m(x_n|\mu_j,\Sigma_j)}.$$
		\item \textcolor{red}{M-step}: Re-estimate model parameters
		\begin{align*}
			\mu_k^{\rm new} & =\sum_{n=1}^N \frac{p(z=k|x_n)}{N_k}x_n,\qquad N_k=\sum_{n=1}^N p(z=k|x_n),\\
			\Sigma_k^{\rm new} & = \sum_{n=1}^N \frac{p(z=k|x_n)}{N_k}(x_n-\mu_k^{\rm new})(x_n-\mu_k^{\rm new})^\top,\\
			\pi_k^{\rm new} & = \frac{N_k}{N}.
		\end{align*}
		\item Evaluate the log-likelihood and check for convergence. 
	\end{itemize}
\end{frame}

%\begin{frame}
%\frametitle{EM algorithm}
%\begin{figure}
%\includegraphics[page=16,width=4.8in,trim={0 0 0 3cm},clip]{./raw.pdf}
%\end{figure}
%\end{frame}

\begin{frame}{Visualization of EM Algorithm}
\begin{figure}
\includegraphics[page=17,width=4.8in,trim={0 0 0 4.7cm},clip]{pics/raw.pdf}
\end{figure}
\end{frame}


\begin{frame}{The General EM algorithm}
Consider a general setting with latent variables.
\begin{itemize}
	\item Observed dataset $\boldsymbol{X}\in \mathbb R^{N\times D}$, latent variables $\boldsymbol{Z}\in \mathbb R^{N\times K}$.
	\end{itemize}
Maximize the log-likelihood  \textcolor{purple}{$\log p(\boldsymbol{X}|\theta)=\log\left(\sum_{\boldsymbol Z}p(\boldsymbol X,\boldsymbol Z|\theta)\right)$}.
\begin{itemize}
	\item Initialize parameters $\theta^{\rm old}$.
%	\item Our knowledge on $z$ carried by the posterior $p(z|x,\theta)$. 
	 \item \textbf{E-step}: use $\theta^{\rm old}$ to compute the posterior $p(\boldsymbol{Z}|\boldsymbol{X},\theta^{\rm old})$.
 	\item \textbf{M-step}: $\theta^{\rm new}=\arg\max_\theta Q(\theta,\theta^{\rm old})$, where
$$
	Q(\theta,\theta^{\rm old})\;=\;\sum_{\boldsymbol{Z}}p(\boldsymbol{Z}|\boldsymbol{X},\theta^{\rm old})\log p(\boldsymbol{X},\boldsymbol{Z}|\theta)\;=\;\mathbb E\Big(\log p(\boldsymbol{X},\boldsymbol{Z}|\theta)\Big|\boldsymbol{X},\theta^{\rm old}\Big)
$$
which is tractable in many applications.
\item Replace $\theta^{\rm old}\leftarrow\theta^{\rm new}$. Repeat until convergence.
\end{itemize}
%\begin{figure}
%\includegraphics[page=19,width=4.8in,trim={0 0 0 3cm},clip]{./raw.pdf}
%\end{figure}
\end{frame}

\begin{frame}{Example: Gaussian mixture}
\begin{itemize}
%	\item We confirm earlier formulas for the Gaussian mixture.
	\item If $z$ was observed, the MLE would be trivial
	$${\small \log p(\boldsymbol X,\boldsymbol Z|\theta)=\sum_{n=1}^N \log p(x_n,z_n|\theta)=\sum_{n=1}^N\sum_{k=1}^K \textcolor{blue}{1\!\!1(z_n\!=\!k)}\log\left(\pi_kN(x_n|\mu_k,\Sigma_k)\right).}$$
\end{itemize}
For the E-step: $p(\boldsymbol{Z}|\boldsymbol{X},\theta)=\prod_{n=1}^N p(z_n|\boldsymbol{X},\theta)$ we have
$$
p(z_n=k|\boldsymbol{X},\theta)=p(z_n=k|x_n,\theta)=\frac{\pi_k N_m(x_n|\mu_k,\Sigma_k)}{\sum_{j=1}^K \pi_j N_m(x_n|\mu_j,\Sigma_j)}.
$$
For the M-step: ${\small \mathbb E(1\!\!1(z_n=k)|\boldsymbol{X},\theta^{\rm old})=p(z_n=k|\boldsymbol{X},\theta^{\rm old})}$ and so 
$$
{\small \mathbb E\Big(\log p(\boldsymbol X,\boldsymbol Z|\theta)\Big|\boldsymbol{X},\theta^{\rm old}\Big)\;=\;\sum_{n=1}^N\sum_{k=1}^K \textcolor{blue}{p(z_n=k|\boldsymbol{X},\theta^{\rm old})}\log\left(\pi_kN(x_n|\mu_k,\Sigma_k)\right).}
$$
Maximizing gives the formulas on Slide~\ref{GMEM}.
%\begin{figure}
%\includegraphics[page=24,width=4.8in,trim={0 0 0 3cm},clip]{./raw.pdf}
%\end{figure}
\end{frame}



\end{document}

