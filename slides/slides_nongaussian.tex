\documentclass[11pt,handout,aspectratio=169]{beamer}


%%%%%%%%% GENERAL PACKAGES

\usepackage{pdfpages}
\usetheme[progressbar=frametitle]{metropolis}
\setbeamercolor{background canvas}{bg=white}
\usepackage{appendixnumberbeamer}
\usepackage{booktabs}
\usepackage[scale=2]{ccicons}
\usepackage{pgfplots}
\usepgfplotslibrary{dateplot}
\usepackage{xspace}
\newcommand{\themename}{\textbf{\textsc{metropolis}}\xspace}
\usepackage[absolute,overlay]{textpos}

%%%%%%%%% COLOR THEME

% Define some colors:
\definecolor{DarkFern}{HTML}{407428}
\definecolor{DarkCharcoal}{HTML}{4D4944}
\definecolor{AlertColor}{RGB}{89,124,158}
\definecolor{HighLight}{RGB}{96,95,134}
\definecolor{Important}{RGB}{234,122,133}
\definecolor{Yellow}{HTML}{00539C}
\colorlet{Fern}{DarkFern!85!white}
\colorlet{Charcoal}{DarkCharcoal!85!white}
\colorlet{LightCharcoal}{Charcoal!50!white}
\colorlet{HighLight2}{AlertColor}
\colorlet{DarkRed}{red!70!black}
\colorlet{DarkBlue}{blue!70!black}
\colorlet{DarkGreen}{green!70!black}
\definecolor{RoyalBlue}{HTML}{00539C}
\definecolor{Peach}{HTML}{EEA47F}
\definecolor{ForestGreen}{HTML}{2C5F2D}
\definecolor{MossGreen}{HTML}{E8FCC9}
% Use the colors:
\setbeamercolor{title}{fg=Fern}
\setbeamercolor{frametitle}{fg=MossGreen,bg=ForestGreen}
\setbeamercolor{normal text}{fg=Charcoal!70!black}
\setbeamercolor{block title}{fg=black,bg=Fern!25!white}
\setbeamercolor{block body}{fg=black,bg=Fern!10!white}
\setbeamercolor{block title alerted}{fg=black,bg=DarkRed!25!white}
\setbeamercolor{block body alerted}{fg=black,bg=DarkRed!10!white}
\setbeamercolor{alerted text}{fg=DarkRed}
\setbeamercolor{itemize item}{fg=Charcoal}



%%%%%%%%% OTHER COMMANDS
\newcommand{\Perp}{\perp\!\!\! \perp}
\newcommand{\Ep}[2]{\ensuremath{E_{#1}\left[{#2}\right]}}
\def\hpY{\mathbf{\bar{\beta}}}
\newcommand{\gaus}[2]{\mathcal{N}\left({#1};\,{#2}\right)}
\newcommand{\comment}[1]{}

\newcommand{\trace}{\text{trace}}
\def\T{\top}
%\newcommand{\det}{\text{det}}
\newcommand{\variance}{\mathrm{Var}}
\def\reals{{\mathbb R}}
\newcommand{\prob}{\mathrm{Pr}}
\newcommand{\zeroVec}{\mathbf{0}}
\newcommand{\zeroMat}{\mathbf{0}}
\newcommand{\onesVec}{\mathbf{1}}
\newcommand{\ident}{\mathbf{I}}
\newcommand{\deriv}{\mathrm{d}}
\newcommand{\transpose}{\top}
\newcommand{\costDeriv}[1]{\overline{#1}}
\newcommand{\lossDeriv}{\costDeriv}
\newcommand{\normal}{\mathcal{N}}
\newcommand{\data}{\mathcal{D}}
%\newcommand{\bm}{{\mathbf{m}}}
\newcommand{\loss}{{\cal L}}
\newcommand{\cG}{{\cal G}}
\newcommand{\cV}{{\cal V}}
\newcommand{\cE}{{\cal E}}
\newcommand{\cP}{{\cal P}}
\newcommand{\X}{{\cal X}}
\newcommand{\Y}{{\cal Y}}
\newcommand{\bK}{\mathbf{K}}
\newcommand{\bX}{\mathbf{X}}
\newcommand{\bY}{\mathbf{Y}}
\newcommand{\bk}{\mathbf{k}}
\newcommand{\bx}{\mathbf{x}}
\newcommand{\by}{\mathbf{y}}
\newcommand{\bhy}{\hat{\mathbf{y}}}
\newcommand{\bty}{\tilde{\mathbf{y}}}
\newcommand{\bG}{\mathbf{G}}
\newcommand{\bI}{\mathbf{I}}
\newcommand{\bg}{\mathbf{g}}
\newcommand{\bS}{\mathbf{S}}
\newcommand{\bs}{\mathbf{s}}
\newcommand{\bM}{\mathbf{M}}
\newcommand{\bw}{\mathbf{w}}
\newcommand{\eye}{\mathbf{I}}
\newcommand{\bU}{\mathbf{U}}
\newcommand{\bV}{\mathbf{V}}
\newcommand{\bW}{\mathbf{W}}
\newcommand{\bn}{\mathbf{n}}
\newcommand{\bv}{\mathbf{v}}
\newcommand{\bq}{\mathbf{q}}
\newcommand{\bR}{\mathbf{R}}
\newcommand{\bi}{\mathbf{i}}
\newcommand{\bj}{\mathbf{j}}
\newcommand{\bp}{\mathbf{p}}
\newcommand{\bt}{\mathbf{t}}
\newcommand{\bJ}{\mathbf{J}}
\newcommand{\bu}{\mathbf{u}}
\newcommand{\bB}{\mathbf{B}}
\newcommand{\bD}{\mathbf{D}}
\newcommand{\bz}{\mathbf{z}}
\newcommand{\bP}{\mathbf{P}}
\newcommand{\bC}{\mathbf{C}}
\newcommand{\bA}{\mathbf{A}}
\newcommand{\bZ}{\mathbf{Z}}
\newcommand{\bff}{\mathbf{f}}
\newcommand{\bF}{\mathbf{F}}
\newcommand{\bo}{\mathbf{o}}
\newcommand{\bc}{\mathbf{c}}
\newcommand{\bm}{\mathbf{m}}
\newcommand{\bT}{\mathbf{T}}
\newcommand{\bQ}{\mathbf{Q}}
\newcommand{\bL}{\mathbf{L}}
\newcommand{\bl}{\mathbf{l}}
\newcommand{\ba}{\mathbf{a}}
\newcommand{\bE}{\mathbf{E}}
\newcommand{\bH}{\mathbf{H}}
\newcommand{\bN}{\mathbf{N}}
\newcommand{\bd}{\mathbf{d}}
\newcommand{\br}{\mathbf{r}}
\newcommand{\be}{\mathbf{e}}
\newcommand{\bb}{\mathbf{b}}
\newcommand{\bh}{\mathbf{h}}
\newcommand{\bhh}{\hat{\mathbf{h}}}

\newcommand{\graph}{{\cal H}}
\newcommand{\bayes}{{\cal B}}
\newcommand{\cx}{{\cal X}}
\newcommand{\cg}{{\cal G}}
\newcommand{\cm}{{\cal M}}
\newcommand{\ci}{{\cal I}}
\newcommand{\ct}{{\cal T}}
\newcommand{\co}{{\cal O}}
\newcommand{\ck}{{\cal K}}
\newcommand{\cu}{{\cal U}}
\newcommand{\cv}{{\cal V}}
\newcommand{\ce}{{\cal E}}
\newcommand{\cf}{{\cal F}}
\newcommand{\cb}{{\cal B}}
\newcommand{\cq}{{\cal Q}}
\newcommand{\cd}{{\cal D}}

\newcommand{\btheta}{\boldsymbol{\theta}}
\newcommand{\bpi}{\boldsymbol{\pi}}
\newcommand{\bphi}{\boldsymbol{\phi}}
\newcommand{\bPhi}{\boldsymbol{\Phi}}
\newcommand{\bmu}{\boldsymbol{\mu}}
\newcommand{\bSigma}{\boldsymbol{\Sigma}}
\newcommand{\bGamma}{\boldsymbol{\Gamma}}
\newcommand{\bbeta}{\boldsymbol{\beta}}
\newcommand{\bomega}{\boldsymbol{\omega}}
\newcommand{\blambda}{\boldsymbol{\lambda}}
\newcommand{\bkappa}{\boldsymbol{\kappa}}
\newcommand{\btau}{\boldsymbol{\tau}}
\newcommand{\balpha}{\boldsymbol{\alpha}}
\def\bgamma{\boldsymbol\gamma}

\newcommand{\argmin}{\operatornamewithlimits{argmin}}

%\newcommand{\animal}[2]{\item[\bf #1] {\em #2}}
 \newcommand{\ikron}[1] {\bI\otimes #1}
  %\newcommand{\val}{\bar{\bx}}
    \newcommand{\train}[1]{{\phi(\bx_{#1})}}
    \newcommand{\ikronval}[1]{(\ikron{\phi(\val_{#1}))}}
\newcommand{\ikronvalT}[1]{(\ikron{\phi(\val_{#1})^T)}}
\newcommand{\ikrontrainT}{(\ikron{\train{i}^T)}}
\newcommand{\ikrontrain}[1]{(\ikron{\train{#1})}}
\newcommand{\ikrontrainAT}{(\ikron{\phi(\bx)^T)}}
\newcommand{\ikrontrainA}{(\ikron{\phi(\bx))}}
  \newcommand{\half}{\frac{1}{2}}
  \newcommand{\con}{C^{(c)}}
    \newcommand{\ig}{\frac{1}{\gamma}}
      \newcommand{\Bi}{\bB^{-1}}
 \newcommand{\kernel}{\hat{\bK}}    
 \newcommand{\ikrontestT}{(\ikron{\test^T)}}
   \newcommand{\test}{\phi(\bx_*)}

% partial derivatives
 \newcommand{\pardev}[2]{\frac{\partial #1}{\partial #2}}
  \newcommand{\dev}[2]{\frac{d #1}{d #2}}
  \newcommand{\dw}{\delta\bw}
  
    \newcommand{\lab}{\mathcal{L}}
      \newcommand{\unlab}{\mathcal{U}}
      
      
  \newcommand{\ind}{1{\hskip -2.5 pt}\hbox{I}}
  
 \newcommand{\ff}[2]{   \cf_{\prec (#1 \rightarrow #2)}}
 \newcommand{\vv}[2]{   \cv_{\prec (#1 \rightarrow #2)}}
  \newcommand{\dd}[2]{   \delta_{#1 \rightarrow #2}}
    \newcommand{\ld}[2]{   \lambda_{#1 \rightarrow #2}}
    \newcommand{\en}[2]{  \bD(#1|| #2)}
       \newcommand{\ex}[3]{  \bE_{#1 \sim #2}\left[ #3\right]} 
       \newcommand{\exd}[2]{  \bE_{#1 }\left[ #2\right]} 
  
%  \newtheorem{theorem}{Theorem}
%\newtheorem{proposition}{Prop}
%\newtheorem{lemma}{Lemma}
%\newtheorem{lemma-ap}{Lemma}
%\newtheorem{definition}{Definition}
%\newtheorem{corollary}{Corollary}
%\newtheorem{claim}{Claim}
%\newtheorem{claim-ap}{Claim}
%\newcommand{\argmin}[1]{\underset{#1}{\mathrm{argmin}} \:}
\newcommand{\argmax}[1]{\underset{#1}{\mathrm{argmax}} \:}
\DeclareMathOperator*{\Max}{max}
\def\eop {{\noindent\framebox[0.5em]{\rule[0.25ex]{0em}{0.75ex}}}}

\newcommand{\tr}[1]{\ensuremath{\mathrm{tr}\left(#1\right)}}
\def\Xdim{{d}}
\def\Ydim{{D}}
\def\Zdim{{S}}

\setbeamertemplate{itemize subitem}{\tiny\raise1.5pt\hbox{\donotcoloroutermaths$\blacktriangleright$}}
\setbeamertemplate{itemize subsubitem}{\tiny\raise1.5pt\hbox{\donotcoloroutermaths$\blacktriangleright$}}
\setbeamertemplate{enumerate item}{\insertenumlabel.}
\setbeamertemplate{enumerate subitem}{\insertenumlabel.\insertsubenumlabel}
\setbeamertemplate{enumerate subsubitem}{\insertenumlabel.\insertsubenumlabel.\insertsubsubenumlabel}
\setbeamertemplate{enumerate mini template}{\insertenumlabel}

\newcommand{\book}[1]{{\it{#1}}}

\newcommand{\high}[1]{{\color{blue}{#1}}}
\newcommand{\raquel}[1]{{\color{red}{#1}}}

\newcommand{\Real}{\mathbb{R}}
\newcommand{\TODO}[1]{{\color{red}{[TODO: #1]}}}

\newcommand{\dataGenDist}{p_{\rm sample}}
\newcommand{\trainDist}{p_{\rm dataset}}
\DeclareMathOperator*{\Var}{Var}
\newcommand{\given}{\,|\,}
\newcommand{\expect}{\mathbb{E}}

\newcommand{\dataIdx}{i}
\newcommand{\featIdx}{j}
\newcommand{\dimIdx}{\featIdx}
\newcommand{\paramIdx}{\dimIdx}
\newcommand{\hidIdx}{i}
\newcommand{\classIdx}{k}
\newcommand{\outputIdx}{k}
\newcommand{\classIdxTwo}{\ell}
\newcommand{\featIdxTwo}{j^\prime}
\newcommand{\nfeat}{D}
\newcommand{\ndim}{\nfeat}
\newcommand{\ndata}{N}
\newcommand{\numClasses}{K}
\newcommand{\nout}{\numClasses}
\newcommand{\layerIdx}{\ell}
\newcommand{\numLayers}{L}
\newcommand{\nhid}{M}
\newcommand{\timeIdx}{t}
\newcommand{\ntime}{T}
\newcommand{\contextLen}{K}


\newcommand{\inputIJ}[2]{x^{(#1)}_{#2}}
\newcommand{\inputI}[1]{{\bf x}^{(#1)}}
\newcommand{\inputJ}[1]{x_{#1}}
\newcommand{\inputVec}{{\bf x}}
\newcommand{\inputVecT}[1]{\inputVec^{(#1)}}
\newcommand{\inputVecI}[1]{\inputVec^{(#1)}}
\newcommand{\inputUni}{x}
\newcommand{\inputUniI}[1]{x^{(#1)}}
\newcommand{\inputUniT}[1]{x^{(#1)}}
\newcommand{\inputMatrix}{\mathbf{X}}
\newcommand{\inputMatrixT}[1]{\inputMatrix^{(#1)}}
\newcommand{\targetI}[1]{t^{(#1)}}
\newcommand{\target}{t}
\newcommand{\targetK}[1]{\target_{#1}}
\newcommand{\targets}{\mathbf{t}}
\newcommand{\prediction}{y}
\newcommand{\predictionI}[1]{y^{(#1)}}
\newcommand{\predictionK}[1]{y_{#1}}
\newcommand{\predictionT}[1]{y^{(#1)}}
\newcommand{\predictions}{\mathbf{y}}
\newcommand{\predictionMatrix}{\mathbf{Y}}
\newcommand{\predictionMatrixT}[1]{\predictionMatrix^{(#1)}}
\newcommand{\intermediate}{z}
\newcommand{\intermediateI}[1]{\intermediate^{(#1)}}
\newcommand{\intermediateT}[1]{\intermediate^{(#1)}}
\newcommand{\intermediateK}[1]{\intermediate_{#1}}
\newcommand{\intermediates}{\mathbf{z}}
\newcommand{\intermediateMatrix}{\mathbf{Z}}
\newcommand{\intermediateMatrixT}[1]{\intermediateMatrix^{(#1)}}
\newcommand{\outIntermediate}{r}
\newcommand{\outIntermediateT}[1]{r^{(#1)}}
\newcommand{\outIntermediateK}[1]{\outIntermediate_{#1}}
\newcommand{\outIntermediates}{\mathbf{r}}
\newcommand{\outIntermediateMat}{\mathbf{R}}
\newcommand{\outIntermediateMatrix}{\mathbf{R}}
\newcommand{\outIntermediateMatrixT}[1]{\outIntermediateMatrix^{(#1)}}
\newcommand{\hiddenI}[1]{h_{#1}}
\newcommand{\hiddenT}[1]{h^{(#1)}}
\newcommand{\hiddenIT}[2]{h_{#1}^{(#2)}}
\newcommand{\hiddenLI}[2]{h_{#2}^{(#1)}}
\newcommand{\hiddens}{\mathbf{h}}
\newcommand{\hiddensL}[1]{\hiddens^{(#1)}}
\newcommand{\hiddensT}[1]{\hiddens^{(#1)}}
\newcommand{\hiddenMatrix}{\mathbf{H}}
\newcommand{\hiddenMat}{\hiddenMatrix}
\newcommand{\hiddenMatrixT}[1]{\hiddenMatrix^{(#1)}}
\newcommand{\hiddenMatL}[1]{\hiddenMat^{(#1)}}
\newcommand{\weights}{{\bf w}}
\newcommand{\weightsLS}{{\bf w}^{\text{LS}}}
\newcommand{\weightsMLE}{{\bf w}^{\text{MLE}}}
\newcommand{\weightsHat}{\hat{\weights}}
\newcommand{\weightsL}[1]{\weights^{(#1)}}
\newcommand{\weightJ}[1]{w_{#1}}
\newcommand{\weightLIJ}[3]{w^{(#1)}_{#2 #3}}
\newcommand{\weightLKI}[3]{w^{(#1)}_{#2 #3}}
\newcommand{\weightLJ}[2]{w^{(#1)}_{#2}}
\newcommand{\weightKJ}[2]{w_{#1 #2}}
\newcommand{\weightIJ}{\weightKJ}
\newcommand{\weightUni}{w}
\newcommand{\weightMat}{\mathbf{W}}
\newcommand{\weightMatL}[1]{\weightMat^{(#1)}}
\newcommand{\bias}{b}
\newcommand{\biasLI}[2]{\bias^{(#1)}_{#2}}
\newcommand{\biasLK}{\biasLI}
\newcommand{\biasL}[1]{\bias^{(#1)}}
\newcommand{\biasK}[1]{\bias_{#1}}
\newcommand{\biasJ}[1]{\bias_{#1}}
\newcommand{\biases}{\mathbf{b}}
\newcommand{\biasesL}[1]{\biases^{(#1)}}
\newcommand{\threshold}{r}
\newcommand{\featureJ}[1]{\psi_{#1}}
\newcommand{\featureVec}{{\boldsymbol \psi}}
\newcommand{\lossI}[1]{\loss^{(#1)}}
\newcommand{\zeroOneLoss}{\loss_{\rm 0-1}}
\newcommand{\squaredErrorLoss}{\loss_{\rm SE}}
\newcommand{\crossEntropyLoss}{\loss_{\rm CE}}
\newcommand{\logisticCrossEntropyLoss}{\loss_{\rm LCE}}
\newcommand{\softmaxCrossEntropyLoss}{\loss_{\rm SCE}}
\newcommand{\hingeLoss}{\loss_{\rm H}}
\newcommand{\cost}{\mathcal{J}}
\newcommand{\regularizer}{\mathcal{R}}
\newcommand{\lrate}{\alpha}
\newcommand{\learningRate}{\lrate}
\newcommand{\featureMap}{{\boldsymbol \psi}}
\newcommand{\featureMapJ}[1]{\psi_{#1}}
\newcommand{\sigmoid}{\sigma}
\newcommand{\logistic}{\sigmoid}
\newcommand{\activationFunction}{\phi}
\newcommand{\activationFunctionL}[1]{\activationFunction^{(#1)}}
\newcommand{\activationFunctionTwo}{\psi}
\newcommand{\parityFunction}{f_{\rm par}}
\newcommand{\function}{f}
\newcommand{\functionL}[1]{\function^{(#1)}}
\newcommand{\indicatorOf}[1]{\mathbbm{1}_{#1}}
\newcommand{\softmax}{\mathrm{softmax}}
\newcommand{\weightCost}{\lambda}
\newcommand{\genCost}{\mathcal{C}}
\newcommand{\momentumVec}{\mathbf{p}}
\newcommand{\momentumJ}[1]{p_{#1}}
\newcommand{\momentumParam}{\mu}
\newcommand{\genParams}{{\boldsymbol \theta}}
\newcommand{\genParamJ}[1]{\theta_{#1}}
\newcommand{\pData}{p_{\mathcal{D}}}
\newcommand{\bestPrediction}{\prediction_\star}

\newcommand{\obs}{\mathbf{x}}
\newcommand{\obsJ}[1]{x_{#1}}
\newcommand{\obsI}[1]{\obs^{(#1)}}
\newcommand{\pfn}{\mathcal{Z}}
\newcommand{\happiness}{H}
\newcommand{\latents}{\mathbf{z}}

\newcommand{\state}{\mathbf{s}}
\newcommand{\stateT}[1]{\state_{#1}}
\newcommand{\act}{\mathbf{a}}
\newcommand{\actT}[1]{\act_{#1}}
\newcommand{\reward}{r}
\newcommand{\policy}{\pi}
\newcommand{\policyParams}{\boldsymbol{\theta}}
\newcommand{\policyTh}{{\policy_{\policyParams}}}
\newcommand{\MDP}{\mathcal{M}}
\newcommand{\rollout}{\tau}
\newcommand{\expectedReturn}{R}

\newcommand{\discReturn}{G}
\newcommand{\discFactor}{\gamma}
\newcommand{\valueFunc}{V}
\newcommand{\valueFuncPi}{\valueFunc^{\policy}}
\newcommand{\valueFuncPiTh}{\valueFunc^{\policyTh}}
\newcommand{\qFunc}{Q}
\newcommand{\qFuncPi}{\qFunc^{\policy}}
\newcommand{\optPolicy}{\policy^*}
\newcommand{\optQ}{\qFunc^*}

\newcommand{\subspace}{\mathcal{S}}
\newcommand{\projectedInput}{\tilde{\inputVec}}
\newcommand{\projectedInputI}[1]{\projectedInput^{(#1)}}
\newcommand{\codeVec}{\mathbf{z}}
\newcommand{\codeVecI}[1]{\codeVec^{(#1)}}
\newcommand{\dataMean}{\boldsymbol{\mu}}
\newcommand{\dataCov}{\boldsymbol{\Sigma}}
\newcommand{\pcaVec}{\mathbf{u}}

\newcommand{\featureMatrix}{{\boldsymbol \Psi}}
\newcommand{\smootherMatrix}{{\boldsymbol \Omega}}
\newcommand{\smootherMatrixEntry}{\Omega}
\newcommand{\hypothesis}{\mathcal{H}}
\newcommand{\priorMean}{\mathbf{m}}
\newcommand{\priorCov}{\mathbf{S}}
\newcommand{\priorVar}{\eta}
\newcommand{\postMean}{\boldsymbol{\mu}}
\newcommand{\postCov}{\boldsymbol{\Sigma}}
\newcommand{\predMean}{\mu_{\rm pred}}
\newcommand{\predVar}{\sigma^2_{\rm pred}}
\newcommand{\predStd}{\sigma_{\rm pred}}

\newcommand{\R}{\mathbb R}
\newcommand{\E}{\mathbb E}
\renewcommand{\P}{\mathbb P}

\newcommand{\One}[1]{{\mathbb I}{\{#1\}}}
\newcommand{\norm}[1]{\left\Vert#1\right\Vert}
\newcommand{\expLoss}{\loss_{\rm E}}
\newcommand{\eqdef}{\triangleq}
\newcommand{\param}{\theta}
\newcommand{\params}{{\boldsymbol \theta}}
\newcommand{\coinParam}{\param}
\newcommand{\numHeads}{{N_H}}
\newcommand{\numTails}{{N_T}}
\newcommand{\likelihood}{L}
\newcommand{\loglik}{\ell}
\newcommand{\mean}{\mu}
\newcommand{\std}{\sigma}
\newcommand{\coinParamML}{\hat{\theta}_{\rm ML}}
\newcommand{\coinParamPred}{\theta_{\rm pred}}
\newcommand{\coinParamMAP}{\hat{\theta}_{\rm MAP}}
\newcommand{\priorStd}{\std_{\rm pri}}
\newcommand{\postStd}{\std_{\rm post}}
\newcommand{\meanML}{\hat{\mu}_{\rm ML}}
\newcommand{\stdML}{\hat{\sigma}_{\rm ML}}
\newcommand{\meanMAP}{\hat{\mu}_{\rm MAP}}
\newcommand{\stdMAP}{\hat{\sigma}_{\rm MAP}}
\newcommand{\paramsML}{\hat{\params}_{\rm ML}}
\newcommand{\paramsMAP}{\hat{\params}_{\rm MAP}}
\def\bW{{\mathbf{u}}}
\renewcommand{\bm}{\mathbf{m}}
\renewcommand{\bA}{\mathbf{A}}
\renewcommand{\bb}{\mathbf{b}}
\renewcommand{\bU}{\mathbf{U}}
\renewcommand{\bW}{\mathbf{W}}
\renewcommand{\bw}{\mathbf{w}}
\newcommand{\beps}{\mathbf{\epsilon}}
\DeclareMathOperator*{\Cov}{Cov}

\newcommand\cceq{\stackrel{\mathclap{\tiny\mbox{for $c=1$}}}{=}}

\newsavebox{\zerobox}
\newenvironment{nospace}
{\par\edef\theprevdepth{\the\prevdepth}\nointerlineskip
  \setbox\zerobox=\vtop to 0pt\bgroup
  \hrule height0pt\kern\dimexpr\baselineskip-\topskip\relax
}
{\par\vss\egroup\ht\zerobox=0pt \wd\zerobox=0pt \dp\zerobox=0pt
  \box\zerobox}

\usepackage{soul}
\makeatletter
\let\HL\hl
\renewcommand\hl{%
  \let\set@color\beamerorig@set@color
  \let\reset@color\beamerorig@reset@color
  \HL}
  \makeatother


\title[STA437-Week1]{STA 437/2005: \\ Methods for Multivariate Data}
\subtitle[]{Week 5: Non-Gaussian Distributions}
\author[Piotr Zwiernik]{Piotr Zwiernik}
\institute[UofT]{University of Toronto}
\date{}


%\usepackage{Sweave}

\begin{document}

\maketitle

\begin{frame}{Table of contents}
\setbeamertemplate{section in toc}[sections numbered]
\tableofcontents%[hideallsubsections]
\end{frame}

\section{Elliptical distributions}

\begin{frame}{}
	\begin{center}
		{\Huge \alert{Elliptical distributions}}
	\end{center}
\end{frame}

\subsection{Spherical distributions}

\begin{frame}{Why Study Elliptical Distributions?}
    \begin{itemize}
        \item Generalize the multivariate normal distribution.\\[5mm]
        \item Model data with heavy tails or outliers.\\[5mm]
        \item Maintain symmetry and linear correlation structures.\\[5mm]
        \item Applications in finance, insurance, and environmental studies.
    \end{itemize}
\end{frame}

% Slide: Spherical Distributions Definition
\begin{frame}{Spherical Distributions}
\textbf{Orthogonal Matrices:} $O(m) = \{ U \in \mathbb{R}^{m \times m} : U^\top U = I_m \}$.
%  \begin{itemize}
%    \item $U \in O(m)$ implies $U^{-1} = U^\top$.
%    \item Determinants satisfy $|\det(U)| = 1$.
%  \end{itemize}
\begin{alertblock}{Spherical distribution}
A random vector $X \in \mathbb{R}^m$ has a \emph{spherical distribution} if for any $U \in O(m)$:
  \begin{equation*}
    X \overset{d}{=} U X.
  \end{equation*}	
\end{alertblock}
Characteristic function satisfies: $\psi_X(\bs t)=\psi_{UX}(\bs t)=\psi_X(U^\top \bs t)$ and so \textbf{equivalently} $\psi_X(\bs t)$ depends only on $\|\bs t\|$. Thus, the same applies to the density: $$f_X(\x)\;=\;h(\|\bs x\|)\qquad\mbox{for some }h\mbox{ (generator)}.$$
%e.g. $X\sim N_m(\bs 0_m,I_m)$ then $h(s)=\tfrac{1}{(2\pi)^{m/2}}e^{-\tfrac12 s}$
  \end{frame}

% Slide: Examples of Spherical Distributions
\begin{frame}{Examples of Spherical Distributions}
Standard normal distribution $Z \sim N_m(0, I_m)$ is a simple example.
\bigskip

\begin{block}{Spherical scale mixture of normals}
	If $Z \sim N_m(0, I_m)$ and a random variable $\tau > 0$ is independent of $Z$, then:
      \begin{equation*}
        X = \frac{1}{\sqrt{\tau}} Z
      \end{equation*}
      has a spherical distribution.
\end{block}
  \textbf{Indeed:} Let $U \in O(m)$, then 
$$    UX \;=\; \frac{1}{\sqrt{\tau}} UZ \;\overset{d}{=}\; \frac{1}{\sqrt{\tau}} Z \;=\; X.$$
\end{frame}

% Slide: Moment Structure of Spherical Distributions
\begin{frame}{Moment Structure of Spherical Distributions}
Spherical symmetry implies:
      \begin{align*}
        \mathbb{E}[X] &= 0, \\
        \var(X) &= c I_m, \quad \mbox{for some }c \geq 0.
      \end{align*}  
      \textbf{Indeed:} $\var(X)=\var(UX)=U \var(X)U^\top$ for any $U\in O(m)$ 
\bigskip

For $X = \frac{1}{\sqrt{\tau}} Z$ with $Z \sim N(0, I_m)$, $\tau>0$, $\tau\indep Z$:
      \begin{equation*}
        \var(X) = \mathbb{E}[\tau^{-1}] I_m.
      \end{equation*}
\textbf{Indeed:} $\E[X]=\E[\tfrac{1}{\sqrt{\tau}}Z]=\E[\tfrac{1}{\sqrt{\tau}}]\E[Z]=\bs 0_m$ and so 
$$\var(X)\;=\;\E XX^\top -\E[X]\E[X]^\top\;=\;\E[\tfrac{1}{\tau}ZZ^\top]\;=\;\E[\tfrac{1}{\tau}]\E[ZZ^\top]\;=\;\E[\tfrac{1}{\tau}]I_m$$
\end{frame}

% Slide: Independence of Norm and Direction
\begin{frame}{Independence of $\|X\|$ and $\frac{X}{\|X\|}$}
\begin{alertblock}{Key Property}
	If $X$ is spherical, the norm $\|X\| = \sqrt{X^\top X}$ is independent of the direction $\frac{X}{\|X\|}$.
\end{alertblock}
  \textbf{Proof Sketch:}
Let $U \in O(m)$. Then:
$$
        \frac{X}{\|X\|} \;\overset{d}{=}\; \frac{UX}{\|UX\|} \;=\; U \frac{X}{\|X\|}.
$$
The vector $\frac{X}{\|X\|}$ is rotationally invariant $\Longrightarrow$ has uniform distribution on the unit sphere (independent of what $\|X\|$ is).
\bigskip

A formal proof uses polar coordinates, see the notes.
\end{frame}

%% Slide: Polar Coordinates
%\begin{frame}{Polar Coordinates}
%  \textbf{Definition:} In $\mathbb{R}^m$, polar coordinates represent $\mathbf{x}$ as:
%  \begin{equation*}
%    \mathbf{x} = r \mathbf{u}(\boldsymbol{\theta}),
%  \end{equation*}
%  where $r = \|\mathbf{x}\|$ is the radial coordinate, and $\boldsymbol{\theta}$ are angular coordinates.
%  \vspace{0.5cm}
%  \textbf{Jacobian Determinant:}
%  \begin{equation*}
%    J(r, \boldsymbol{\theta}) = r^{m-1} \prod_{i=2}^{m-1} \sin^{m-i}(\theta_{i-1}).
%  \end{equation*}
%  \vspace{0.5cm}
%  \textbf{Implication:} If $f(\mathbf{x}) = g(\|\mathbf{x}\|^2)$, then:
%  \begin{equation*}
%    f(\mathbf{x}) d\mathbf{x} = g(r^2) r^{m-1} dr d\boldsymbol{\theta}.
%  \end{equation*}
%\end{frame}

\subsection{Elliptical distributions}

% Slide: Elliptical Distributions
\begin{frame}{Elliptical Distribution $E(\mu,\Sigma)$}
Recall that $Z\sim N_m(\bs 0_m,I_m)$ then $X=\mu+\Sigma^{1/2}Z\sim N_m(\mu,\Sigma)$.
\begin{block}{Elliptical distribution}
A random vector $X \in \mathbb{R}^m$ has an elliptical distribution if:
  \begin{equation*}
    X = \mu + \Sigma^{1/2} Z,
  \end{equation*}
  where $Z$ is a spherical random vector.	
\end{block}
The density of $X\sim E(\mu,\Sigma)$ is of the form $$f_X(\x)\;=\;c_m \sqrt{\det{\Sigma^{-1}}}h((\x-\mu)^\top\Sigma^{-1}(\x-\mu)).$$
The generator $g$ controls the shape of the distribution (and its tails in particular). 
\end{frame}

% Slide: Why Elliptical Distributions?
\begin{frame}{Again: Why Elliptical Distributions?}
    \begin{itemize}
        \item Generalize the multivariate normal distribution.\\[5mm]
        \item Model data with heavy tails or outliers.\\[5mm]
        \item Maintain symmetry and linear correlation structures.\\[5mm]
        \item Applications in finance, insurance, and environmental studies.
    \end{itemize}
\end{frame}

% Slide: Scale Mixtures of Normals
\begin{frame}{Scale Mixtures of Normals}
Scale mixture of normals is a special class of elliptical distributions. 
\bigskip

  \textbf{Stochastic representation:}
  \begin{equation*}
    X = \mu + \frac{1}{\sqrt{\tau}} \Sigma^{1/2} Z,
  \end{equation*}
  where $Z \sim N_m(0, I_m)$ and $\tau > 0$ is independent of $Z$.
  \vspace{0.5cm}
  \begin{block}{Special Cases of Scale Mixture of Normals}
  	  \begin{itemize}
    \item $\tau \equiv 1$: Multivariate normal.
    \item $\tau \sim \frac{1}{k} \chi^2_k$: Multivariate $t$-distribution with $k$ degrees of freedom.
    \begin{itemize}
    \item Smaller $k$ means heavier tails. Gaussian is the limit $k\to \infty$.
    \end{itemize}
    \item $\tau \sim \text{Exp}(1)$: Multivariate Laplace.
  \end{itemize}
    \end{block}
\end{frame}

% Slide: Covariance and Correlation
\begin{frame}{Covariance and Correlation in Elliptical Distributions}
$\Sigma$ is called the \textbf{scale matrix}. It is generally not equal to the covariance matrix.
\medskip
      \begin{equation*}
        \mathrm{Var}(X) = c \Sigma, \quad c > 0.
      \end{equation*}

Correlation structure is still governed by $\Sigma$:
$$
R_{ij}\;=\;\frac{c\Sigma_{ij}}{\sqrt{c\Sigma_{ii}c\Sigma_{jj}}}\;=\;\frac{\Sigma_{ij}}{\sqrt{\Sigma_{ii}\Sigma_{jj}}}.
$$
Similarly, if $X\sim E(\mu,\Sigma)$ and $X=(X_A,X_B)$ then 
$$
\E(X_A|X_B=\x_B)\;=\;\E(X_A)-\Sigma_{A,B}\Sigma_{B,B}^{-1}(\x_B-\mu_B)
$$
exactly as in the Gaussian case.
\end{frame}

\section{Copula models}
\begin{frame}{}
	\begin{center}
		{\Huge \alert{Copula models}}
	\end{center}
\end{frame}


\begin{frame}{What is a Copula?}
\begin{itemize}
    \item A \textbf{copula} is a function that captures the \textbf{dependence structure} between random variables, separate from their marginal distributions.
    \item It is a multivariate cumulative distribution function (CDF) with uniform marginals on $[0, 1]$.
    \item Why use copulas?
    \begin{itemize}
        \item To model non-Gaussian dependencies.
        \item To analyze dependence independently of marginal behaviors.
    \end{itemize}
    \item Formal Definition: \newline
    A copula $C : [0, 1]^m \to [0, 1]$ satisfies the properties of a multivariate CDF with uniform marginals.
\end{itemize}
\end{frame}

% Slide 2: Sklar's Theorem
\begin{frame}{Sklar's Theorem}
\begin{block}{Theorem (Sklar, 1959)}
Let $X = (X_1, \ldots, X_m)$ be a random vector with joint CDF $F$ and marginals $F_1, \ldots, F_m$. There exists a unique copula $C$ such that:
\[ F(x_1, \dots, x_m) = C(F_1(x_1), \dots, F_m(x_m)). \]
Conversely, given marginals $F_1, \ldots, F_m$ and a copula $C$, the joint CDF $F$ is defined by the same formula.
\end{block}
\begin{itemize}
    \item $C$ captures \textbf{dependence structure}.
    \item $F_1, \ldots, F_m$ capture marginal behaviors.
\end{itemize}
\end{frame}

% Slide 3: Understanding Sklar's Theorem
\begin{frame}{Understanding Sklar's Theorem}
\begin{itemize}
    \item When $m = 1$, $C(u) = u$, the identity function on $[0, 1]$.
    \item If $X$ is continuous with CDF $F$, then $F(X) \sim U(0, 1)$. \newline
    \textbf{Proof:} \newline
    \[
    \mathbb{P}(F(X) \leq u) = \mathbb{P}(X \leq F^{-1}(u)) = F(F^{-1}(u)) = u.
    \]
    \item In higher dimensions:
    \[ F(x_1, \ldots, x_m) = C(F_1(x_1), \ldots, F_m(x_m)). \]
\end{itemize}
\end{frame}

% Slide 4: Copulas from Uniform Marginals
\begin{frame}{Copulas from Uniform Marginals}
\begin{itemize}
    \item Let $X = (X_1, \ldots, X_m)$ with CDF $F$.
    \item Define $U_i = F_i(X_i)$, where $F_i$ are the marginal CDFs.
    \item The transformed variables $U = (U_1, \ldots, U_m)$ have uniform marginals.
    \[ \mathbb{P}(U_1 \leq u_1, \ldots, U_m \leq u_m) = C(u_1, \ldots, u_m). \]
    \item Sklar's theorem ensures $C$ is unique for continuous distributions.
\end{itemize}
\end{frame}

% Slide 5: Example of a Copula
\begin{frame}{Simple Example of a Copula}
\begin{itemize}
    \item Joint CDF:
    \[ F_{X,Y}(x, y) =
    \begin{cases} 
    0 & x < 0 \text{ or } y < 0, \\
    x^2 y^2 & 0 \leq x, y \leq 1, \\
    1 & x > 1 \text{ and } y > 1, \\
    \min(x^2, y^2) & \text{otherwise}.
    \end{cases} \]
    \item Marginal CDFs:
    \[ F_X(x) = x^2, \quad F_Y(y) = y^2 \quad \text{for } 0 \leq x, y \leq 1. \]
    \item Copula:
    \[ C(u, v) = uv \quad \text{if } u, v \leq 1. \]
\end{itemize}
\end{frame}

% Slide 6: Invariance under Transformations
\begin{frame}{Invariance under Monotone Transformations}
\begin{itemize}
    \item Let $Y_i = f_i(X_i)$, where $f_i$ are strictly increasing transformations.
    \item The copula remains unchanged.
    \item Proof outline:
    \begin{itemize}
        \item Marginals transform: $G_i(y_i) = F_i(f_i^{-1}(y_i))$.
        \item Copula representation remains:
        \[ C(u_1, \ldots, u_m) = F(F_1^{-1}(u_1), \ldots, F_m^{-1}(u_m)). \]
    \end{itemize}
\end{itemize}
\end{frame}

% Slide 7: Copula Density
\begin{frame}{Density of a Copula}
\begin{itemize}
    \item The PDF of a copula $C$ is obtained by differentiating its CDF:
    \[ c(\mathbf{u}) = \frac{\partial^m C(\mathbf{u})}{\partial u_1 \cdots \partial u_m}. \]
    \item Using the chain rule:
    \[ c(\mathbf{u}) = \frac{f(\mathbf{x})}{\prod_{i=1}^m f_i(x_i)}, \]
    where $f$ is the joint density and $f_i$ are marginal densities.
\end{itemize}
\end{frame}

% Slide 8: Gaussian Copula
\begin{frame}{Gaussian Copula}
\begin{itemize}
    \item Derived from the multivariate normal distribution.
    \item Simplify to standard normal marginals:
    \[ c(\mathbf{u}; \Sigma) = \det(\Sigma)^{-1/2} \exp\left(-\frac{1}{2} \mathbf{x}^\top (\Sigma^{-1} - I_m) \mathbf{x}\right), \]
    where $x_i = \Phi^{-1}(u_i)$ and $\Phi$ is the standard normal CDF.
\end{itemize}
\end{frame}

% Slide 9: Applications of Copulas
\begin{frame}{Applications of Copulas}
\begin{itemize}
    \item \textbf{Finance:} Modeling dependencies in asset returns.
    \item \textbf{Insurance:} Understanding risks in correlated claims.
    \item \textbf{Environmental Science:} Joint modeling of extreme events (e.g., floods).
    \item \textbf{Medical Statistics:} Modeling dependence in survival times.
\end{itemize}
\end{frame}


\end{document}

