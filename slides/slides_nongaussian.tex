\documentclass[11pt,handout,aspectratio=169]{beamer}


%%%%%%%%% GENERAL PACKAGES
%\usepackage{xcolor}
%\usepackage{pdfpages}
%\usetheme[progressbar=frametitle]{metropolis}
%\setbeamercolor{background canvas}{bg=white}
%\usepackage{appendixnumberbeamer}
%\usepackage{booktabs}
%\usepackage[scale=2]{ccicons}
%\usepackage{pgfplots}
%\usepgfplotslibrary{dateplot}
%\usepackage{xspace}
%\newcommand{\themename}{\textbf{\textsc{metropolis}}\xspace}
%\usepackage[absolute,overlay]{textpos}

%%%%%%%%% COLOR THEME

% Define some colors:
\definecolor{DarkFern}{HTML}{407428}
\definecolor{DarkCharcoal}{HTML}{4D4944}
\definecolor{AlertColor}{RGB}{89,124,158}
\definecolor{HighLight}{RGB}{96,95,134}
\definecolor{Important}{RGB}{234,122,133}
\definecolor{Yellow}{HTML}{00539C}
\colorlet{Fern}{DarkFern!85!white}
\colorlet{Charcoal}{DarkCharcoal!85!white}
\colorlet{LightCharcoal}{Charcoal!50!white}
\colorlet{HighLight2}{AlertColor}
\colorlet{DarkRed}{red!70!black}
\colorlet{DarkBlue}{blue!70!black}
\colorlet{DarkGreen}{green!70!black}
\definecolor{RoyalBlue}{HTML}{00539C}
\definecolor{Peach}{HTML}{EEA47F}
\definecolor{ForestGreen}{HTML}{2C5F2D}
\definecolor{MossGreen}{HTML}{E8FCC9}
% Use the colors:
\setbeamercolor{title}{fg=Fern}
\setbeamercolor{frametitle}{fg=MossGreen,bg=ForestGreen}
\setbeamercolor{normal text}{fg=Charcoal!70!black}
\setbeamercolor{block title}{fg=black,bg=Fern!25!white}
\setbeamercolor{block body}{fg=black,bg=Fern!10!white}
\setbeamercolor{block title alerted}{fg=black,bg=DarkRed!25!white}
\setbeamercolor{block body alerted}{fg=black,bg=DarkRed!10!white}
\setbeamercolor{alerted text}{fg=DarkRed}
\setbeamercolor{itemize item}{fg=Charcoal}



%%%%%%%%% OTHER COMMANDS
\newcommand{\indep}{\perp\!\!\! \perp}
\newcommand{\comment}[1]{}
\newcommand{\bs}{\boldsymbol}
\newcommand{\tr}{\text{trace}}
\newcommand{\sgn}{{\rm sgn}}
\def\T{\top}
%\newcommand{\det}{\text{det}}
\newcommand{\var}{\mathrm{var}}
\newcommand{\cC}{{\cal C}}
\newcommand{\cG}{{\cal G}}
\newcommand{\cV}{{\cal V}}
\newcommand{\cE}{{\cal E}}
\newcommand{\cM}{{\cal M}}
\newcommand{\cP}{{\cal P}}
\newcommand{\cX}{{\cal X}}
\newcommand{\cY}{{\cal Y}}
\newcommand{\X}{\mathbf{X}}
\newcommand{\Y}{\mathbf{Y}}
\newcommand{\x}{\mathbf{x}}
\newcommand{\y}{\mathbf{y}}
\newcommand{\z}{\mathbf{z}}

\newcommand{\argmin}{\operatornamewithlimits{argmin}}
\newcommand{\eps}{\varepsilon}
\newcommand{\<}{\langle}
\renewcommand{\>}{\rangle}


\setbeamertemplate{itemize subitem}{\tiny\raise1.5pt\hbox{\donotcoloroutermaths$\blacktriangleright$}}
\setbeamertemplate{itemize subsubitem}{\tiny\raise1.5pt\hbox{\donotcoloroutermaths$\blacktriangleright$}}
\setbeamertemplate{enumerate item}{\insertenumlabel.}
\setbeamertemplate{enumerate subitem}{\insertenumlabel.\insertsubenumlabel}
\setbeamertemplate{enumerate subsubitem}{\insertenumlabel.\insertsubenumlabel.\insertsubsubenumlabel}
\setbeamertemplate{enumerate mini template}{\insertenumlabel}

\newcommand{\TODO}[1]{{\color{red}{[TODO: #1]}}}


\newcommand{\R}{\mathbb R}
\newcommand{\E}{\mathbb E}
\renewcommand{\P}{\mathbb P}


\DeclareMathOperator*{\cov}{cov}


\newsavebox{\zerobox}
\newenvironment{nospace}
{\par\edef\theprevdepth{\the\prevdepth}\nointerlineskip
  \setbox\zerobox=\vtop to 0pt\bgroup
  \hrule height0pt\kern\dimexpr\baselineskip-\topskip\relax
}
{\par\vss\egroup\ht\zerobox=0pt \wd\zerobox=0pt \dp\zerobox=0pt
  \box\zerobox}

\usepackage{soul}
\makeatletter
\let\HL\hl
\renewcommand\hl{%
  \let\set@color\beamerorig@set@color
  \let\reset@color\beamerorig@reset@color
  \HL}
  \makeatother


\title[STA437-Week1]{STA 437/2005: \\ Methods for Multivariate Data}
\subtitle[]{Week 5: Non-Gaussian Distributions}
\author[Piotr Zwiernik]{Piotr Zwiernik}
\institute[UofT]{University of Toronto}
\date{}


%\usepackage{Sweave}

\begin{document}

\maketitle

\begin{frame}{Table of contents}
\setbeamertemplate{section in toc}[sections numbered]
\tableofcontents%[hideallsubsections]
\end{frame}

\section{Elliptical distributions}

\begin{frame}{}
	\begin{center}
		{\Huge \alert{Elliptical distributions}}
	\end{center}
\end{frame}

\subsection{Spherical distributions}

\begin{frame}{Why Study Elliptical Distributions?}
    \begin{itemize}
        \item Generalize the multivariate normal distribution.\\[5mm]
        \item Model data with heavy tails or outliers.\\[5mm]
        \item Maintain symmetry and linear correlation structures.\\[5mm]
        \item Applications in finance, insurance, and environmental studies.
    \end{itemize}
\end{frame}

% Slide: Spherical Distributions Definition
\begin{frame}{Spherical Distributions}
\textbf{Orthogonal Matrices:} $O(m) = \{ U \in \mathbb{R}^{m \times m} : U^\top U = I_m \}$.
%  \begin{itemize}
%    \item $U \in O(m)$ implies $U^{-1} = U^\top$.
%    \item Determinants satisfy $|\det(U)| = 1$.
%  \end{itemize}
\begin{alertblock}{Spherical distribution}
A random vector $X \in \mathbb{R}^m$ has a \emph{spherical distribution} if for any $U \in O(m)$:
  \begin{equation*}
    X \overset{d}{=} U X.
  \end{equation*}	
\end{alertblock}
Characteristic function satisfies: $\psi_X(\bs t)=\psi_{UX}(\bs t)=\psi_X(U^\top \bs t)$ and so \textbf{equivalently} $\psi_X(\bs t)$ depends only on $\|\bs t\|$. Thus, the same applies to the density: $$f_X(\x)\;=\;h(\|\bs x\|)\qquad\mbox{for some }h\mbox{ (generator)}.$$
%e.g. $X\sim N_m(\bs 0_m,I_m)$ then $h(s)=\tfrac{1}{(2\pi)^{m/2}}e^{-\tfrac12 s}$
  \end{frame}

% Slide: Examples of Spherical Distributions
\begin{frame}{Examples of Spherical Distributions}
Standard normal distribution $Z \sim N_m(0, I_m)$ is a simple example.
\bigskip

\begin{block}{Spherical scale mixture of normals}
	If $Z \sim N_m(0, I_m)$ and a random variable $\tau > 0$ is independent of $Z$, then:
      \begin{equation*}
        X = \frac{1}{\sqrt{\tau}} Z
      \end{equation*}
      has a spherical distribution.
\end{block}
  \textbf{Indeed:} Let $U \in O(m)$, then 
$$    UX \;=\; \frac{1}{\sqrt{\tau}} UZ \;\overset{d}{=}\; \frac{1}{\sqrt{\tau}} Z \;=\; X.$$
\end{frame}

% Slide: Moment Structure of Spherical Distributions
\begin{frame}{Moment Structure of Spherical Distributions}
Spherical symmetry implies:
      \begin{align*}
        \mathbb{E}[X] &= 0, \\
        \var(X) &= c I_m, \quad \mbox{for some }c \geq 0.
      \end{align*}  
      \textbf{Indeed:} $\var(X)=\var(UX)=U \var(X)U^\top$ for any $U\in O(m)$ 
\bigskip

For $X = \frac{1}{\sqrt{\tau}} Z$ with $Z \sim N(0, I_m)$, $\tau>0$, $\tau\indep Z$:
      \begin{equation*}
        \var(X) = \mathbb{E}[\tau^{-1}] I_m.
      \end{equation*}
\textbf{Indeed:} $\E[X]=\E[\tfrac{1}{\sqrt{\tau}}Z]=\E[\tfrac{1}{\sqrt{\tau}}]\E[Z]=\bs 0_m$ and so 
$$\var(X)\;=\;\E XX^\top -\E[X]\E[X]^\top\;=\;\E[\tfrac{1}{\tau}ZZ^\top]\;=\;\E[\tfrac{1}{\tau}]\E[ZZ^\top]\;=\;\E[\tfrac{1}{\tau}]I_m$$
\end{frame}

% Slide: Independence of Norm and Direction
\begin{frame}{Independence of $\|X\|$ and $\frac{X}{\|X\|}$}
\begin{alertblock}{Key Property}
	If $X$ is spherical, the norm $\|X\| = \sqrt{X^\top X}$ is independent of the direction $\frac{X}{\|X\|}$.
\end{alertblock}
  \textbf{Proof Sketch:}
Let $U \in O(m)$. Then:
$$
        \frac{X}{\|X\|} \;\overset{d}{=}\; \frac{UX}{\|UX\|} \;=\; U \frac{X}{\|X\|}.
$$
The vector $\frac{X}{\|X\|}$ is rotationally invariant $\Longrightarrow$ has uniform distribution on the unit sphere (independent of what $\|X\|$ is).
\bigskip

A formal proof uses polar coordinates, see the notes.
\end{frame}

%% Slide: Polar Coordinates
%\begin{frame}{Polar Coordinates}
%  \textbf{Definition:} In $\mathbb{R}^m$, polar coordinates represent $\mathbf{x}$ as:
%  \begin{equation*}
%    \mathbf{x} = r \mathbf{u}(\boldsymbol{\theta}),
%  \end{equation*}
%  where $r = \|\mathbf{x}\|$ is the radial coordinate, and $\boldsymbol{\theta}$ are angular coordinates.
%  \vspace{0.5cm}
%  \textbf{Jacobian Determinant:}
%  \begin{equation*}
%    J(r, \boldsymbol{\theta}) = r^{m-1} \prod_{i=2}^{m-1} \sin^{m-i}(\theta_{i-1}).
%  \end{equation*}
%  \vspace{0.5cm}
%  \textbf{Implication:} If $f(\mathbf{x}) = g(\|\mathbf{x}\|^2)$, then:
%  \begin{equation*}
%    f(\mathbf{x}) d\mathbf{x} = g(r^2) r^{m-1} dr d\boldsymbol{\theta}.
%  \end{equation*}
%\end{frame}

\subsection{Elliptical distributions}

% Slide: Elliptical Distributions
\begin{frame}{Elliptical Distribution $E(\mu,\Sigma)$}
Recall that $Z\sim N_m(\bs 0_m,I_m)$ then $X=\mu+\Sigma^{1/2}Z\sim N_m(\mu,\Sigma)$.
\begin{block}{Elliptical distribution}
A random vector $X \in \mathbb{R}^m$ has an elliptical distribution if:
  \begin{equation*}
    X = \mu + \Sigma^{1/2} Z,
  \end{equation*}
  where $Z$ is a spherical random vector.	
\end{block}
The density of $X\sim E(\mu,\Sigma)$ is of the form $$f_X(\x)\;=\;c_m \sqrt{\det{\Sigma^{-1}}}h((\x-\mu)^\top\Sigma^{-1}(\x-\mu)).$$
The generator $g$ controls the shape of the distribution (and its tails in particular). 
\end{frame}

% Slide: Why Elliptical Distributions?
\begin{frame}{Again: Why Elliptical Distributions?}
    \begin{itemize}
        \item Generalize the multivariate normal distribution.\\[5mm]
        \item Model data with heavy tails or outliers.\\[5mm]
        \item Maintain symmetry and linear correlation structures.\\[5mm]
        \item Applications in finance, insurance, and environmental studies.
    \end{itemize}
\end{frame}

% Slide: Scale Mixtures of Normals
\begin{frame}{Scale Mixtures of Normals}
Scale mixture of normals is a special class of elliptical distributions. 
\bigskip

  \textbf{Stochastic representation:}
  \begin{equation*}
    X = \mu + \frac{1}{\sqrt{\tau}} \Sigma^{1/2} Z,
  \end{equation*}
  where $Z \sim N_m(0, I_m)$ and $\tau > 0$ is independent of $Z$.
  \vspace{0.5cm}
  \begin{block}{Special Cases of Scale Mixture of Normals}
  	  \begin{itemize}
    \item $\tau \equiv 1$: Multivariate normal.
    \item $\tau \sim \frac{1}{k} \chi^2_k$: Multivariate $t$-distribution with $k$ degrees of freedom.
    \begin{itemize}
    \item Smaller $k$ means heavier tails. Gaussian is the limit $k\to \infty$.
    \end{itemize}
    \item $\tau \sim \text{Exp}(1)$: Multivariate Laplace.
  \end{itemize}
    \end{block}
\end{frame}

% Slide: Covariance and Correlation
\begin{frame}{Covariance and Correlation in Elliptical Distributions}
$\Sigma$ is called the \textbf{scale matrix}. It is generally not equal to the covariance matrix.
\medskip
      \begin{equation*}
        \mathrm{Var}(X) = c \Sigma, \quad c > 0.
      \end{equation*}

Correlation structure is still governed by $\Sigma$:
$$
R_{ij}\;=\;\frac{c\Sigma_{ij}}{\sqrt{c\Sigma_{ii}c\Sigma_{jj}}}\;=\;\frac{\Sigma_{ij}}{\sqrt{\Sigma_{ii}\Sigma_{jj}}}.
$$
Similarly, if $X\sim E(\mu,\Sigma)$ and $X=(X_A,X_B)$ then 
$$
\E(X_A|X_B=\x_B)\;=\;\E(X_A)-\Sigma_{A,B}\Sigma_{B,B}^{-1}(\x_B-\mu_B)
$$
exactly as in the Gaussian case.
\end{frame}

\section{Copula models}
\begin{frame}{}
	\begin{center}
		{\Huge \alert{Copula models}}
	\end{center}
\end{frame}

\begin{frame}{Cumulative Distribution Function (CDF)}
	Let $X=(X_1,\ldots,X_m)$ be a random vector.  Its \textbf{CDF} is: $$F(x_1,\ldots,x_m)=\P(X_1\leq x_1,X_2\leq x_2,\ldots,X_m\leq x_m).$$
	Marginal CDF: $F_1(x_1)=\P(X_1\leq x_1)=\lim_{x_2\to \infty}\cdots\lim_{x_m\to\infty}F(x_1,x_2,\ldots,x_m)$.
	(similar for any other margin)\\[5mm]	
	\begin{block}{}
	If $f$ is the corresponding density, then:
	$$
	f(x_1,\ldots,x_m)\;=\;\frac{\partial^m}{\partial x_1\cdots\partial x_m}F(x_1,\ldots,x_m)
	$$
	
	$$
	F(x_1,\ldots,x_m)\;=\;\int_{-\infty}^{x_1}\cdots \int_{-\infty}^{x_m} f(y_1,\ldots,y_m){\rm d}y_1\cdots {\rm d} y_m. 
	$$		
	\end{block}
		If $U\sim U[0,1]$ then $F(u)=u$ $u\in [0,1]$.

\end{frame}


\begin{frame}{What is a Copula?}
\begin{itemize}
    \item A \textbf{copula} is a function that captures the \textbf{dependence structure} between random variables, separate from their marginal distributions.
\end{itemize}
\begin{alertblock}{Definition}
	A function $C : [0, 1]^m \to [0, 1]$ is a \textbf{copula} if it is a CDF with uniform marginals, that is, $C_1(u_1)=u_1$, \ldots, $C_m(u_m)=u_m$, where $C_i$ are the marginal CDF's.
\end{alertblock}
For example, the copula $C(\bs u)=u_1\cdots u_m$ corresponds to a $m$ independent $U[0,1]$.
\begin{block}{Why use copulas?}
	\begin{itemize}
        \item To model non-Gaussian dependencies.
        \item To analyze dependence independently of marginal behaviors.
    \end{itemize}
\end{block}
\end{frame}

% Slide 2: Sklar's Theorem
\begin{frame}{Sklar's Theorem}
\begin{block}{Theorem (Sklar, 1959)}
Let $X = (X_1, \ldots, X_m)$ be a random vector with joint CDF $F$ and marginals $F_1, \ldots, F_m$. There exists a unique copula $C$ such that:
\[ F(x_1, \dots, x_m) = C(F_1(x_1), \dots, F_m(x_m)). \]
Conversely, given marginals $F_1, \ldots, F_m$ and a copula $C$, the joint CDF $F$ is defined by the same formula.
\end{block}
\begin{itemize}
    \item $C$ captures \textbf{dependence structure}.
    \item $F_1, \ldots, F_m$ capture marginal behaviors.
\end{itemize}
\end{frame}

% Slide 3: Understanding Sklar's Theorem
\begin{frame}{Understanding Sklar's Theorem}
\begin{itemize}
    \item When $m = 1$, $C(u) = u$, the identity function on $[0, 1]$.\\[5mm]
    \item If $X$ is continuous with CDF $F$, then $F(X) \sim U(0, 1)$. \\[3mm]
    \textbf{Proof:} If $X$ is continuous, $F$ is strictly increasing on the support. Hence
    \[
    \mathbb{P}(F(X) \leq u) = \mathbb{P}(X \leq F^{-1}(u)) = F(F^{-1}(u)) = u.
    \]
%    \item In higher dimensions:
%    \[ F(x_1, \ldots, x_m) = C(F_1(x_1), \ldots, F_m(x_m)). \]
\end{itemize}
\end{frame}

% Slide 4: Copulas from Uniform Marginals
\begin{frame}{}
\begin{itemize}
    \item Let $X = (X_1, \ldots, X_m)$ with CDF $F$.
    \item Define $U_i = F_i(X_i)$, where $F_i$ are the marginal CDFs.
    \item The transformed variables $U = (U_1, \ldots, U_m)$ have uniform marginals.
    \[ \mathbb{P}(U_1 \leq u_1, \ldots, U_m \leq u_m) = C(u_1, \ldots, u_m). \]
    \item Also note that 
\begin{equation}\label{eq:cop1}
    \mathbb{P}(U_1 \leq u_1, \ldots, U_m \leq u_m) = F(F_1^{-1}(u_1),\ldots,F_m^{-1}(u_m))	
\end{equation}
    and so the copula can be computed explicitly.
    \item Sklar's theorem ensures $C$ is unique for continuous distributions.
\end{itemize}
\begin{alertblock}{}
	Given copula $C$ and margins $F_i$ we easily get the corresponding  distribution using \eqref{eq:cop1}:
	$$
	F(x_1,\ldots,x_m)\;=\;C(F_1(x_1),\ldots,F_m(x_m)).
	$$
\end{alertblock}
\end{frame}

\begin{frame}{Sampling}
	Fix a copula $C(\bs u)$
\end{frame}

% Slide 5: Example of a Copula
\begin{frame}{Simple Example of a Copula}
\begin{itemize}
    \item Joint CDF:
    \[ F_{X,Y}(x, y) =
    \begin{cases} 
    0 & x < 0 \text{ or } y < 0, \\
    x^2 y^2 & 0 \leq x, y \leq 1, \\
    1 & x > 1 \text{ and } y > 1, \\
    \min(x^2, y^2) & \text{otherwise}.
    \end{cases} \]
    \item Marginal CDFs:
    \[ F_X(x) = x^2, \quad F_Y(y) = y^2 \quad \text{for } 0 \leq x, y \leq 1. \]
    \item Copula:
    \[ C(u, v) = uv \quad \text{if } u, v \leq 1. \]
\end{itemize}
\end{frame}

% Slide 6: Invariance under Transformations
\begin{frame}{Invariance under Monotone Transformations}
\begin{itemize}
    \item Let $Y_i = f_i(X_i)$, where $f_i$ are strictly increasing transformations.
    \item The copula remains unchanged.
    \item Proof outline:
    \begin{itemize}
        \item Marginals transform: $G_i(y_i) = F_i(f_i^{-1}(y_i))$.
        \item Copula representation remains:
        \[ C(u_1, \ldots, u_m) = F(F_1^{-1}(u_1), \ldots, F_m^{-1}(u_m)). \]
    \end{itemize}
\end{itemize}
\end{frame}

% Slide 7: Copula Density
\begin{frame}{Density of a Copula}
\begin{itemize}
    \item The PDF of a copula $C$ is obtained by differentiating its CDF:
    \[ c(\mathbf{u}) = \frac{\partial^m C(\mathbf{u})}{\partial u_1 \cdots \partial u_m}. \]
    \item Using the chain rule:
    \[ c(\mathbf{u}) = \frac{f(\mathbf{x})}{\prod_{i=1}^m f_i(x_i)}, \]
    where $f$ is the joint density and $f_i$ are marginal densities.
\end{itemize}
\end{frame}

% Slide 8: Gaussian Copula
\begin{frame}{Gaussian Copula}
\begin{itemize}
    \item Derived from the multivariate normal distribution.
    \item Simplify to standard normal marginals:
    \[ c(\mathbf{u}; \Sigma) = \det(\Sigma)^{-1/2} \exp\left(-\frac{1}{2} \mathbf{x}^\top (\Sigma^{-1} - I_m) \mathbf{x}\right), \]
    where $x_i = \Phi^{-1}(u_i)$ and $\Phi$ is the standard normal CDF.
\end{itemize}
\end{frame}

% Slide 9: Applications of Copulas
\begin{frame}{Applications of Copulas}
\begin{itemize}
    \item \textbf{Finance:} Modeling dependencies in asset returns.
    \item \textbf{Insurance:} Understanding risks in correlated claims.
    \item \textbf{Environmental Science:} Joint modeling of extreme events (e.g., floods).
    \item \textbf{Medical Statistics:} Modeling dependence in survival times.
\end{itemize}
\end{frame}


\end{document}

